\chapter{Poisson Processes}
A Poisson process of intensity, or rate, $\lambda > 0$ is an integer-valued stochastic process ${X_t; t \ge 0}$ for which:
\begin{enumerate}
  \item $X_0 = 0$
	\item For any time points $t_0 = 0 < t_1 < t_2 < \cdots < t_n$, the process increments
	$$X_{t_1}-X_{t_0}, X_{t_2}-X_{t_1}, \cdots, X_{t_n}-X_{t_{n-1}}$$
	are independent random variables, i.e. knowing the number of new arrivals doesn't help to know the next arrivals
	\item For $s \ge 0$ and $t > 0$, the random variable $X_{t+s} - X_s$ has the Poisson distribution
	\begin{equation}\label{p_dist}
	  \prob[X_{t+s} - X_s = n] = \frac{(\lambda t)^ne^{-\lambda t}}{n!}
	\end{equation}
	where
	\begin{equation}\label{p_dist_conseq}
	  \begin{split}
	    \prob[X_n \ge 1] &= \lambda n + o(n) \\
			\prob[X_n \ge 1] &= o(n)
	  \end{split}
	\end{equation}
	\gls{kt} at page 226 shows that assuming \eq{p_dist} you can get \eq{p_dist_conseq} and viceversa.
\end{enumerate}

Interarrival times in a \gls{pp} are iid exponentials with parameter $\lambda$

\begin{tikzpicture}
	\begin{axis}[
		y = 1.5cm,
		hide y axis,
		axis x line = bottom,
		xtick={0,1,2,3,4},
		xticklabels={,,$s_0$,$s_1$,$s_2$,$\cdots$}
	]
	\end{axis}
\end{tikzpicture}


\begin{equation}
	\prob[S_0 > t] = \prob[0 ~arrivals ~in ~[0,t]] = \frac{(\lambda t)^0 e^{-\lambda t}}{0!} = e^{-\lambda t}
\end{equation}
\begin{equation}\label{poiss_independence}
	\prob[S_1 > t | S_0 = s] = \prob[0 ~arrivals ~in ~[s, s+t]] = \prob[S_1 > t] = e^{-\lambda t}
\end{equation}

Where \eq{poiss_independence} is possible because $S_0$ and $S_1$ are disjoint and so the condition on $S_0$ is independent on the $S_1$ slot.
In general,
\begin{equation}
	\prob[S_n > t | S_i = s_i, ~i=0,\cdots,n-1] = e^{-\lambda t}
\end{equation}

%LAW OF RARE EVENTS: rare events can be aproximated with a poisson process? (non ho ben capito)
\section{Properties}
\begin{itemize}
  \item Consider the sum of two independent Poisson process with parameters $\lambda_1$ and $\lambda_2$, what is the outgoing process?
        \begin{figure}[H]
          \centering
          \begin{tikzpicture}
  	         \draw (0,0) circle [radius = 0.5];
             \draw [->] (-2,1.2) -- (-0.5,0.5);
             \draw [->] (-2,-1.2) -- (-0.5,-0.5);
             \draw [->] (0.6, 0) -- (2.6,0);
             \node at (-1.25, 1.3) {$\lambda_1$};
             \node at (-1.25, - 1.3) {$\lambda_2$};
             \node at (1.6 , 0.3) {$\lambda$};
          \end{tikzpicture}
        \end{figure}
          Given X $\sim Poi(\mu)$, Y $\sim Poi(\nu)$
          \begin{equation}
            \begin{split}
  	           \prob[X + Y = n] &=  \sum\limits_{k=0}^n \prob[X = k, Y = n - k] \\
               &=  \sum\limits_{k=0}^n \prob[X = k] \prob[Y = n - k] \\
               &=  \sum\limits_{k=0}^n \frac{e^{-\mu} \mu^k}{k!} \frac{e^{-\nu}\nu^{n-k}}{(n-k)!}\\
               &=  \frac{e^{-\mu + \nu}}{n!}\sum\limits_{k=0}^n \frac{n!}{k!(n-k)!}\mu^k\nu^{n-k}\\
               &=  \frac{e^{-\mu + \nu}}{n!}(\mu + \nu)^n
             \end{split}
           \end{equation}
           That is poisson distributed with parameter $\mu + \nu$ (the sum of the two parameters). \\
           Therefore the outgoing process is posson distributed with parameter $\lambda = \lambda_1 + \lambda_2$.
     \item Now considering the opposite case: two processes generated by the splitting of one (into process $X_1(t)$ with probability $\rho$ and into process $X_2(t)$ with probability $1-\rho$)
           \begin{figure}[H]
                  \centering
                    \begin{tikzpicture}
                      \draw (0,0) circle [radius = 0.5];
                      \draw [->] (0.5, 0.5) -- (2 , 1.2);
                      \draw [->] (0.5,-0.5) -- (2, -1.2);
                      \draw [->] (-2.6, 0) -- (-0.6, 0);
                      \node at (1.25, 1.3) {$\lambda_1$};
                      \node at (1.25, - 1.3) {$\lambda_2$};
                      \node at (-1.6 , 0.3) {$\lambda$};
                      \node at (2.5, 1.2) {$X_1(t)$};
                      \node at (2.5, -1.2) {$X_2(t)$};
                      \node at (-3.1, 0) {$X(t)$};
                    \end{tikzpicture}
                  \end{figure}
            The resulting processes are independent with parameters $\lambda_1=\lambda \rho$ and $\lambda_2 = \lambda (1 - \rho)$.
            \begin{proof}
              We already know that $\lambda_1 = \lambda \rho$ and $\lambda_2 = \lambda (1 - \rho)$ (?).  We need to prove that the two processes are poisson distributed and independent.
              \begin{equation}
                \begin{split}
                  \prob[X_1(t) = n, X_2(t)=m] &= \prob[X_1(t)=n, X(t)= n +m] \\
                  &= \prob[X_1(t)=n | X(t)=n+m]\prob[X(t)=n+m]\\
                  &= {{n+m}\choose{n}}p^n(1-p)^m \frac{e^{-\lambda t}(\lambda t)^{m+n}}{(m+n)!}\\
                  &= \frac{(n+m)!}{n!m!}p^n(1-p)^m e^{-\lambda p t}e^{-\lambda (1-p)t}\frac{(\lambda t)^m (\lambda t)^n}{(n+m)!}\\
                  &= \frac{e^{-\lambda p t}(\lambda p t)^n}{n!}\frac{e^{-\lambda (1-p) t}(\lambda (1-p) t)^m}{m!}
                \end{split}
              \end{equation}
              I can factorize the two distributions and separate them, this shows the indipendence of the two poisson distributions.
            \end{proof}
            For the result to be proved in general the proof should be done for every possible interval of time. Idea of the proof:
            \begin{itemize}
              \item disjoint intervals: arrivals already independent $\rightarrow$ trivial
              \item $x_1$ inside $x_2$: I can apply the result I've just proved to $x_1$ first and then to the parts in $x_2$ that are not in $x_1$
              \item $x_1$ and $x_2$ overlapping just on one side: same as before, first apply the previous result to the overlapping part, then to the others
            \end{itemize}



  \end{itemize}
\section{Distribution related to Poisson Process}
Defining:
\begin{itemize}
	\item $X(t)$ as the number of arrivals up to time t;
	\item $W_n$ the instant of $n_{th}$ arrival;
	\item $S_n$= $W_{n+1}$-$W_n$ the inter-arrival time between $(n+1)_{th}$ and $n_{th}$ arrival;
\end{itemize}


	\begin{theorem}[5.4, 	\textbf{$W_n$$ \sim$ Gamma Distribution} ]
		Being $W_{1}$,$W_{2}$,..$W_{n}$ the arrival time in a Poisson Process $N(t)$ with rate ${\lambda}>0$ the probability density function of $W_n$ is the following:

			\begin{equation}
			f_{W_n} (t )=\frac{\lambda^n t^{n-1}} {(n-1)!}e^{-\lambda t} \\
			\end{equation}
	\end{theorem}

\begin{theorem}[5.7, \textbf{Joint probability of [$W_{1}$,..$W_{n}$]} ]
	Being $W_{1}$,$W_{2}$,..$W_{n}$ the arrival time in a Poisson Process $N(t)$ with rate ${\lambda}>0$. Given $N(t)=n$, the joint distribution of $w_{1}$,$w_{2}$,..$w_{n}$ variables is:
	\begin{equation}
	f_{W_1,W_2,..W_n|n} (w_1,w_2,..w_n) = P (W_1=w_1,...W_n=w_n|N(t)=n) = n!/t^n
	\end{equation}
	\begin{proof}
		\begin{enumerate}
			\item Excluding Simultaneous arrivals: \newline
		$	P[N(t+h)-N(t)=0]= 1 - \lambda h+ o(h)$ \newline
		$    P[N(t+h)-N(t)=1]= \lambda h+ o(h)$     \newline
		$    P[N(t+h)-N(t)>=0]= o(h)$, negligible as h$\rightarrow$0  \newline
		\item
	 The probability of the event $ \{w_i<W_i \leq w_i+\Delta w \}$ for $i=1,...,n$ and $N(t)=n$
	  correspond to the following events:
	  \newline $E_1=$\{0 arrivals out of$\Delta$ $w_i$ intervals\}=$\{N((0,w_1])=0,...,N((w_n+\Delta w,t])=0 \} $
	  \newline and
	  \newline $E_2$=\{1 arrival in each of $\Delta$$w_i$ intervals\}=$\{N((w_1,w_1+\Delta w_1])=1,...,N((w_n,w_n+\Delta w])=1\}$
	 \newline Compute the probability of these two disjoint events:
	  \begin{equation}
	  \begin{split}
	  P[N((0,w_1])=0,...,N((w_n+\Delta w_n,t])=0] =\\
	 = e^{-\lambda w_1} \cdot e^{-\lambda (w_2-(w_2+\Delta w_2))} ... \cdot e^{-\lambda (t-(w_n+\Delta w_n))}\\
	  =e^{-\lambda t} (1+o(max\{\Delta w_i\}))
	  \end{split}
       \end{equation}
       \newline
       \begin{equation}
      \begin{split}
     	P[N((w_1,,w_1+\Delta w_1])=1,...,N((w_n,w_n+\Delta w])=1] =\\
        = (\lambda \Delta w_1) \cdot ... \cdot(\lambda \Delta w_n) (1+o(max\{\Delta w_i\}))
      \end{split}
	  \end{equation}
	  \item So the final probability is computed as the product of these two independent events:
	  	\begin{equation}
	  	\begin{split}
	   f_{W_1,W_2,..W_n|n} (w_1,w_2,..w_n)\cdot( \Delta w_1\Delta w_2..\Delta w_n) = \\
	  =P[E_1 \cap E_2 ]\\
	  =P[w_i\leq W_i \leq w_i+\Delta w_i, i=1,2,..,n| \textnormal{n arrivals}]=\\
	  =\frac{P[w_i\leq W_i \leq w_i+\Delta w_i, i=1,2,..,n,\textnormal{n arrivals}}{\textnormal{n arrivals}}\\
	  =\frac{(\lambda \Delta w_1 e^{-\lambda w_1}\cdot..\lambda \Delta w_n e^{-\lambda w_n})e^{-\lambda t}}{\frac{(\lambda t)^{n}e^{-\lambda t}} {n!}}\\
	  =\frac{ (\Delta w_1 \cdot \Delta w_2\cdot.. \Delta w_n) n!}{t^n}
	  	\end{split}
	  \end{equation}
	  \item
	  Dividing both members for ($\Delta w_1 \cdot \Delta w_2 \cdot .. \Delta w_n$) we obtain:
	   \begin{equation}
         f_{W_1,W_2,..W_n|n} (w_1,w_2,..w_n) =n!/t^n
	  \end{equation}
	  (Fixing the value of t and n this is a uniform density function)
		\end{enumerate}
	\end{proof}
\end{theorem}
\begin{theorem}[5.5,\textbf{Exponential Inter-arrival} ]
	The Inter-arrival times $S_0,S_1, S_{n-1}$ are independent r.v. with exponential density function:
	\begin{equation}
	f_{S_k}(s)=\lambda e^{-\lambda s}, s \ge 0
	\end{equation}

\end{theorem}

\begin{theorem}[5.6,\textbf{Conditioned Arrivals} ]
  (Enunciate given in an intuitive but not so formal way)
  \newline
  Being $X(t)$ a Poisson Process with $\lambda>0$, $u$ and $t$ two temporal variables such that $0<u<t$, $n$ and $k$ two variables which describes the number of arrivals such that $0\leq k\leq n$ . The probability of the event  "up to instant $u$, $k$ arrivals have been occurred conditioned by the fact that $n$ arrivals occurs up to time $t$" is described by a binomial distribution with $n$ attempts and $u/t$ success probability.
  Formally:

  \begin{equation}
  P[X(u)=k|X(t)=n]= \frac{n!}{k!(n-k)!}(\frac{u}{t})^k(1-\frac{u}{t})^{n-k}
  \end{equation}
  \begin{proof}
  	It is a direct consequence of Theorem 5.7.
  	Being $X(t)=n$, the position of each arrival along interval $t$ can be modeled as i.i.d. uniform random variable. This corresponds to the realization of n experiments with  $p_{succ}=u/t$  thus the conditioned probability can be modeled as binomial $Bin(n,\frac{u}{t})$.
  \end{proof}

\end{theorem}

\section{M/G/$\infty$}
$X(t)= \#$ of arrivals in $ [ 0, t ]$ \\
$M(t)= \#$ of users in the system at time $t$ \\
A user is still in the system at time t if $W_k + Y_k \geq t$. Therefore $M(t) = \sum\limits_{k=1}^{X(t)} \mathds{1}\{W_k + Y_k \geq t\} $.
\begin{equation}
  \begin{split}
      P[M(t)=m|X(t)=n] &= P[\sum\limits_{k=1}^{X(t)} \mathds{1}\{W_k + Y_k \geq t\} = m|X(t)=n] \\
      &= P[\sum\limits_{k=1}^{n} \mathds{1}\{U_k + Y_k \geq t\}] = {{n}\choose{m}}p^{m}(1-p)^{n-m}
  \end{split}
\end{equation}
Where $p = P[U_k + Y_k \geq t]$, we have to compute this probability.\\
\begin{equation}
  \begin{split}
    p = P[U_k + Y_k \geq t] &= \frac{1}{t}\int_0^t P[Y_k \geq t -u] \mathrm{d}u \\
    &= \frac{1}{t}\int_0^t [1 - G(t-u)] \mathrm{d}u
    &= \frac{1}{t}\int_0^t [1 - G(z)] \mathrm{d}z
  \end{split}
\end{equation}
Now we can compute
\begin{equation}
  \begin{split}
      P[M(t)=m] &= \sum\limits_{n=0}^{\infty} P[M(t)=m|X(t)=n]P[X(t)=n] \\
      &= \sum\limits_{n=m}^{\infty} \frac{n!}{m!(n-m)!}p^m(1-p)^{n-m}\frac{e^{-\lambda t} (\lambda t)^n}{n!} \\
      &= \frac{e^{-\lambda t} p^m (\lambda p t)^m}{m!}\sum\limits_{n=m}^{\infty} \frac{(1-p)^{n-m}(\lambda t)^{n-m}}{(n-m)!}\\
      &= \frac{e^{\lambda p t}(\lambda p t)^m}{m!}
  \end{split}
\end{equation}
Where $\lambda p t = E[M(t)] = \lambda t \frac{1}{t} \int_0^t [1 - G(z)] \mathrm{d}z = \lambda \int_0^t [1 - G(z)] \mathrm{d}z$. \\
As $t \rightarrow \infty$: $\lambda \int_0^t [1 - G(z)] \mathrm{d}z = \lambda E[Y] (=p)$.

%shot noise example? should we add it?

\section{BG 275: Slotted multi-access}
Assumptions:
\begin{itemize}
  \item[1.] Slotted system: constant packet duration, synchronized users
  \item[2.] Poisson arrivals: with \textbf{total} arrival rate $\lambda$
  \item[3.] Collision or perfect reception
  \item[4.] Immediate $ \{ 0,1, e\}$ feedback
  \item[5.] Retransmission until success (backlogged)
  \item[6a.] No buffering $\rightarrow$ upper bound to the performance of a realistic system
  \item[6b.] Infinite set of nodes $\rightarrow$ lower bound
\end{itemize}
\subsection{ALOHA}
Slot time = 1\\
G = average $\#$ of attempts per slot\\
S = average $\#$ of successes per slot \\
$\lambda$ = average rate of arrivals (new packet per slot)

The success probability is $P_{succ} = e^{-T \cdot G} = e^{-2 \cdot G}$, as the vulnerability interval is T=2.
Now suppose to consider the interval $d\tau$, we have
\begin{equation}\begin{split}
  \text{Are stationary processes:} \begin{cases}
    \prob[\text{one TX starts in $d\tau$}] &\approx G  \cdot d\tau\\
    \exp{\text{\# of successes in $d\tau$}} &= G \cdot e^{-2 \cdot G} d\tau \\
  \end{cases}
  \exp{\text{\# of successes in $[0;t]$}} &= \int\limits_0^t G \cdot e^{-2 \cdot G} = t \cdot G \cdot e^{-2 \cdot G}\\
  \text{S = throughput  }&=\frac{1}{t}\cdot  \int\limits_0^t G \cdot e^{-2 \cdot G} =  G \cdot e^{-2 \cdot G}\\
  \implies S_{max} = \frac{1}{2e}& \approx 18.7\% \quad G_{max} = \frac{1}{2}
\end{split}\end{equation}

If the user transmits only at the end of the (universal) time: \textbf{Slotted ALOHA}.
With S.A. the vulnerability interval becomes T=1. $\implies S_{max})=\frac{1}{e} \approx 37\% \quad G_{max} = 1$.

\subsubsection{Suppose m users}
Let the state of the system be n, which considers the number of backlogged users.
The traffic per user is $\frac{\lambda}{m}$ and let the probability that a backlogged
user transmits in a slot be $q_r$. Suppose the retransmission probability density function be a geometric,

then a non-backlogged user transmits with probability $q_a = 1-e^{-\frac{\lambda}{m}}$.

As the arrival are poissonian and the retransmission are memoryless, we can study this case with \gls{mc}.

\begin{equation}\begin{split}
  Q_a(i,n) &= \prob[\text{i new arrivals per TX | state n}] = \binom{m-n}{i} \cdot q_a^i \cdot (1-q_a)^{m-n-i} \\
  Q_r(i,n) &= \prob[\text{i ReTX | state n}] = \binom{n}{i} \cdot q_r^i \cdot (1-q_r)^{n-i} \\
\end{split}\end{equation}
Now we want to define the transition probability $P_{n,n+1}$, and the calculations provide the following cases:

\begin{equation}
  P_{n,n+1} = \begin{cases}
    Q_a(i,n) & 2 \le i \le m-n \\
    Q_a(1,n) \cdot [1-Q_r(0,n)] & i=1 \quad \text{1 arrival, no departures}\\
    Q_a(0,n) \cdot [1-Q_r(1,n)] + Q_a(1,n)\cdot Q_r(0,n) & i=0 \quad \text{no arrivals, 1 departure (successful reTX) or 1 arrival successfully transmitted }\\
    Q_a(0,n)\cdot Q_r(1,n) & i = -1 \quad \text{1 departure, no arrivals}\\
    0 & i < -1 \quad \text{no more than one departure}
  \end{cases}
\end{equation}
We can conclude that the resulting matrix is an almost-upper-triangular.

\subsubsection{Drift Analysis}
\begin{equation}\begin{split}
  D_n &= \exp[\text{change in state} | n] = \exp[X_{t+1} - X_t | X_t = n] \\
  &= \exp[\text{arrivals - departures} | n ] = (m-n) \cdot q_a - P_{succ} \\
  P_{succ}&= \prob[\text{exactly 1 TX}|n] = Q_a(1,n) \cdot Q_r(0,n) + Q_a(0,n) \cdot Q_r(1,n) \\
  &= (m-n)\cdot(1-q_a)^{n-m-1} \cdot q_a \cdot(1-q_r)^n + (1-q_a)^{m-n}\cdot n \cdot q_r \cdot (1-q_r)^(n-1) \\
  &= (1-q_1)^{m-n} \cdot (1-q_r)^n \cdot [\frac{m-n}{1-qa}\cdot q_a + \frac{n \cdot q_r}{1-q_r}]
\end{split}\end{equation}
we can then approximate $(1+x)^n \approx e^{-n \cdot x}$ if $x \ll 1$
\begin{equation}
  \implies P_{succ} \approx e^{-(m-n)\cdot q_a} \cdot e^{-n\cdot q_r} \cdot \overbrace{[(m-n) \cdot q_a + n \cdot q_r]}^{G(n)}
  \implies P_{succ} \approx e^{-G(n)}
\end{equation}

We found out that the success probability depends on the state, so we would like to know how the states evolve in time.
The process can be approximated as a Poisson Process and the description is given by $D_n = (m-n)\cdot q_a - G(n) \cdot e^{-G(n)}$
with G(n) as given before.

------ INSERIRE GRAFICO Dn

When $D_n>0$ the system tends to reduce the backlogging, wherease if $D_n<0$ the backlogging
increases and the system becomes unstable.

We would like to improve the system, such that the probability of the system to become
unstable, is small and the delay to be as low as possible; to do so we can
\begin{enumerate}
  \item[increase $q_r$] The delay gets smaller, but the stability get worse
  \item[decrease $q_r$] the delay increases, but the system is more stable
\end{enumerate}

Considering the limit, when $m \to +\infty$, $q_a \to 0 \implies q_a \approx \frac{\lambda}{m}$
As $G=\lambda + n \cdot q_r$ we'd like to maximize the throughput and the success probability.
$G_{max}=1 = \lambda + n  \cdot q_r \implies q_r = \frac{1-\lambda}{n} \implies q_r = q_r(n)$

We don't know n, so we need to extimate $q_r(n)$ to best fit. We will call the maximized G version
of ALOHA as the \textbf{STABILIZED ALOHA}.

\begin{lemma}[Pakes' lemma (B.G. p.264)]
  Suppose $D_i < +\infty \forall i$ and that $\exists \delta > 0 , i_0$ index s.t.
  $D_i \le - \delta \forall i > i_0$ ($D_i$ is bounded away from zero)\\
  Then the  \gls{mc} is \textbf{stable} (or positive recurrent).
  \begin{proof}
    Let $\beta = \max_{i \le i_0} D_i$ . Then $\forall i$:
    \begin{equation}\begin{split}
      \exp[X_n | X_0 = i] -i &= \sum\limits_{k=1}^n \exp[X_k - X_{k-1}|X_0 = i] \\
      &= \sum\limits_{k=1}^n \sum\limits_{j=0}^{+\infty} \exp[X_k - X_{k-1}|X_{k-1} = j , X_0 = i ] \cdot P_{i,j}^{(k-1)}\\
      & \le \sum\limits_{k=1}^{n} \left[\beta \cdot \sum\limits_{j=0}^{+\infty} P_{i,j}^{(k-1)} - \delta \cdot (1- \sum\limits_{j=0}^{i_0} P_{i,j}^{(k-1)}\right]\\
       &= (\beta + \delta) \cdot \sum\limits_{k=1}^n \sum\limits_{j=0}^{i_0} P_{i,j}^{(k-1)} - n \cdot \delta \\
       \implies & 0 \le \exp[X_n | X_0 = i] \le (\beta + \delta) \cdot \sum\limits_{k=1}^n \sum\limits_{j=0}^{i_0} P_{i,j}^{(k-1)} - n \cdot \delta +i\\
       & \quad \text{as it's valid $\forall n , i$ we can say that} \\
       \exp[X_n | X_0 = i] &= n \cdot \left[(\beta + \delta) \cdot \sum\limits_{j=0}^{i_0} \frac{1}{n} \cdot \sum\limits_{k=1}^{n} P_{i,j}^{(k-1)} - \delta \right] +i\\
    \text{As $n \to +\infty$ } & \text{ we can write}\\
    0 \le \lim_{n \to +\infty} &(\beta + \delta) \cdot\sum\limits_{j=0}^{i_0} \underbrace {\frac{1}{n} \cdot \sum\limits_{k=1}^{n} P_{i,j}^{(k-1)}}_{\pi_j} - \delta + \underbrace{\frac{i}{n}}_{\to 0}\\
    0 \le &\underbrace{(\beta + \delta)}_{>0} \cdot \sum\limits_{j=0}^{i_0} \pi_j - \delta \quad \text{\textbf{CONTRADDICTION }if } \pi_j = 0 \forall j \\
    \implies & \text{it must be positive recurrent}
    \end{split}\end{equation}
  \end{proof}
\end{lemma}
\begin{lemma}[Kaplan's lemma]

\end{lemma}
