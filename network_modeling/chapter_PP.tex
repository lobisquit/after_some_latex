\chapter{Poisson Processes}
A Poisson process of intensity, or rate, $\lambda > 0$ is an integer-valued stochastic process ${X_t; t \ge 0}$ for which:
\begin{enumerate}
  \item $X_0 = 0$
	\item For any time points $t_0 = 0 < t_1 < t_2 < \cdots < t_n$, the process increments
	$$X_{t_1}-X_{t_0}, X_{t_2}-X_{t_1}, \cdots, X_{t_n}-X_{t_{n-1}}$$
	are independent random variables, i.e. knowing the number of new arrivals doesn't help to know the next arrivals
	\item For $s \ge 0$ and $t > 0$, the random variable $X_{t+s} - X_s$ has the Poisson distribution
	\begin{equation}\label{p_dist}
	  \prob[X_{t+s} - X_s = n] = \frac{(\lambda t)^ne^{-\lambda t}}{n!}
	\end{equation}
	where
	\begin{equation}\label{p_dist_conseq}
	  \begin{split}
	    \prob[X_n \ge 1] &= \lambda n + o(n) \\
			\prob[X_n \ge 1] &= o(n)
	  \end{split}
	\end{equation}
	\gls{kt} at page 226 shows that assuming \eq{p_dist} you can get \eq{p_dist_conseq} and viceversa.
\end{enumerate}

Interarrival times in a \gls{pp} are iid exponentials with parameter $\lambda$

\begin{tikzpicture}
	\begin{axis}[
		y = 1.5cm,
		hide y axis,
		axis x line = bottom,
		xtick={0,1,2,3,4},
		xticklabels={,,$s_0$,$s_1$,$s_2$,$\cdots$}
	]
	\end{axis}
\end{tikzpicture}


\begin{equation}
	\prob[S_0 > t] = \prob[0 ~arrivals ~in ~[0,t]] = \frac{(\lambda t)^0 e^{-\lambda t}}{0!} = e^{-\lambda t}
\end{equation}
\begin{equation}\label{poiss_independence}
	\prob[S_1 > t | S_0 = s] = \prob[0 ~arrivals ~in ~[s, s+t]] = \prob[S_1 > t] = e^{-\lambda t}
\end{equation}

Where \eq{poiss_independence} is possible because $S_0$ and $S_1$ are disjoint and so the condition on $S_0$ is independent on the $S_1$ slot.
In general,
\begin{equation}
	\prob[S_n > t | S_i = s_i, ~i=0,\cdots,n-1] = e^{-\lambda t}
\end{equation}
