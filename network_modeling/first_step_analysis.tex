\section{First step analysis}
\begin{definition}[Absorbing state]
	 A state $i$ is defined $absorbient$ if it is valid the following condition:$p_{ij}=0$ $\forall j\neq i$ and $p_{ii}=1$. In practice, when a chain incur in the state $i$ no transition to other different states is longer possible. Therefore  the chain will remain "absorbed" in it.
\end{definition}
\begin{definition}[Absorption time]
	Being $A$ the set of the absorbing states of a particular markov chain, the absorption time is defined in the following way:
	\begin{equation}
	T= \min \{n\geq0: X_n \in A \}
	\end{equation}
\end{definition}
``First step analysis'' is a ``standard procedure'' applied to resolve some particular Markov Chain in which there is one of more absorbing state.
 We are interested in computing two fundamental quantities:
\begin{itemize}
	\item The probability of absorption given a starting point \\$i$ :  $ u_i=P[X_T\in A | X_0=i ]$
	\item The expected absorption time given a starting point $i$ :\\ $v_i= E[T|X_0=i]$
\end{itemize}
\subsection{First Step Analysis: General Method}
\begin{itemize}
	\item Reorganize the markov chain according to the following rules
	\begin{itemize}
		\item The states from $0,..r-1$ transient;
		\item The states from $r,...,N$ absorbing.
	\end{itemize}

	\item  Reorganize the transition matrix according to the following rules :
     \begin{equation} P=\begin{pmatrix}
     P_{00} & P_{01} & \cdots & P_{0r}& P_{0(r+1)} &\cdots & \cdots &P_{0N} \\
     P_{00} & P_{01} & \cdots & P_{1r}& P_{1(r+1)}& \cdots & \cdots & P_{1N}\\
     \vdots & \vdots & \ddots  \\
     P_{r0}=0 & P_{r1}=0& \cdots &P_{rr}=1& 0 & \cdots & \cdots &0\\
     P_{(r+1)0}=0 & P_{(r+1)1}=0& \cdots &P_{(r+1)r}=0& P_{(r+1)(r+1)}=1& 0&  \cdots&  0\\
     \vdots & \vdots & \ddots  \\
     P_{N0}=0 & P_{N1}=0& \cdots &P_{Nr}=0& P_{N(r+1)}=0& 0&  \cdots&  P_{NN}=1\\

     \end{pmatrix}
     \end{equation}
     As shown in the figure above the matrix is composed by three parts:
     \begin{itemize}
     	\item In the upper part there are the transitions probability from the not-absorbient states.
     	\item In the low left corner there are the transition probability from  the absorbing states to the not-absorbing states: these can be have only 0 for the definition of absorbing state.
     	\item In the low-right corner there are the transition probabilities from  the absorbing states to the absorbing states. This is a $(N-r) \times(N-r)$ identity submatrix: once a chain enters in one absorbing state, it can only remain in it and cannot go away

     \end{itemize}
    \item In order to compute the the probability of absorption given a starting point $i$ it is possible to write the following:
    \begin{equation}
    \begin{split}
      u_{ik} &= P[X_T=k|X_0=i]= \sum\limits_{j=0}^N u_{jk}P_{ij}=\\
      &\sum\limits_{j=0}^N P[X_T=k|X_0=i,X_1=j]P[X_1=j|X_0=i]=\\
      &= P_{ik}+\sum\limits_{j=0}^N 0*P_{ij} +\sum\limits_{j=0}^{r-1} u_{jk}P_{ij}\\
      &=P_{ik}+\sum\limits_{j=0}^{r-1} u_{jk}P_{ij}
      \end{split}
     \end{equation}

       this is valid for $i= 0,1,..r-1$
    Intuitively the probability of being absorbed in a state $k$ starting in $i$ is equal to the transient probability of the transition $i\rightarrow k$ plus the probabilities of being absorbed from another not-absorbing state ($j \in (0,...r-1)$) weighted by the probability of doing the transition  $i\rightarrow j$ ($P_{ij}$).
    In this case the unknown values are the $u_{jk}$ thus
    a system of $ r-1$ has to be solved in order to compute properly $u_{ik}$ probability.
     \newline Trivially for $i= r,.. , N$, $u_{ik}=0$ for $i\neq k$ and $u_{ik}=1$ $i=k$.

 \item In order to compute the expected absorption time given a starting point $i$ : $v_i= E[T|X_0=i]$ a very similar method is applied:
 \begin{equation}
 v_i= 1 + \sum\limits_{j=0}^{r-1} P_{ij} v_j
  \end{equation}
   Intuitively the expected absorption time from i is equal to 1 with probability 1 (at least one transition has to be done) plus the  the expected absorption time starting from j weighted by the probability of doing the transition  $i\rightarrow j$ ($P_{ij}$) where j (of course) is a not-absorbing state.

  \item Remember that all the equations written above are recursive because they will have the following form :
  \begin{equation}
     u_{ik}= P_{ik}+\sum\limits_{j=0}^{r-1} u_{jk}P_{ij} =  P_{ik}+ u_{0k}P_{i0} + .... + u_{ik}P_{ii} + ... +  u_{(r-1)k}P_{i(r-1)}
   \end{equation}
\end{itemize}

%\subsection{First passage time}
%Another application of first step analysis is the computation of so-called "First-passage-time" define in the following way:
%	\newline
%	$\theta_{ij}$= \# of transitions to reach j starting from i
%% TO BE DONE ...
