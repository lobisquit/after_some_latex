\chapter{Introduction}
\section{Topics of the course}
\begin{itemize}
  \item Review of probability theory
  \item Markov chains
  \item Poisson processes
  \item Example of applications protocols
  \item Random access protocols
\end{itemize}
\section{Review of probability theory}
Let $X_n, X_t$ be two different instances of a random variable (r.v.). We will use
\textit{t} as a subscript if the r.v. is continuous ($t \in \mathbb{R}$), \textit{n} if the r.v. is discrete ($ n \in \mathbb{Z}$)

Let \textit{A},\textit{B} be two events, then:
\begin{itemize}
  \item if $A \cap B = \emptyset$, then A and B are disjoint
  \item $P[A \cup B] = P[A]+P[B] - P[A \cap B]$
  \item $0\le P[A] \le 1$
  \item $P[A \cup B] \le P[A]+P[B]$
  \item $ P[\Omega]=1 $ and $P[\emptyset]=0$ with $\Omega$ the Universe set
\end{itemize}
\subsection{Total probability law}
Given $A_i \cap A_j = \emptyset \;\; \forall i \neq j $, then $\bigcup\limits_{i=1}^{+\infty} A_i \, = \Omega$.
Moreover $P[B]=\sum\limits_{i=1}^{+\infty} P[B \cap A_i]$.

If $P[A \cap B] = P[A]\cdot P[B] \implies A,B$ \textit{are} \textbf{independent}.

\subsection{Distribution function (PMD)}\label{sec:pmd}
$F(x) = P[X \le x]$ is called the distribution function and has the following properties:
\begin{enumerate}
  \item $\lim\limits_{x \to -\infty} F(x) = 0$ \quad and \quad $\lim\limits_{x \to +\infty} F(x) = 1$
  \item F is monothonic non-decreasing $\implies$ if $x_1 > x_2 \implies F(x_1)\ge F(x_2)$
  \item F(x) is continuous from the right
\end{enumerate}
We define $f(x)=F'(x)$ the probability density function \textit{PDF}.

\subsection{Moment}
We define  the moment as:
$E[X^m]=
\begin{cases}
    \sum\limits_{i} {x{_i}^{m} \cdot P[X=x_i]} & \text{if X is discrete } \\
    \int_{-\infty}^{+\infty} {x^{m} \cdot f(x) dx }  & \text{if X is continuous}
\end{cases}$
In particular if m=1, we obtain the mean, if m=2 we get the power (and in particular if $E[X]=0$ we obtain the variance).\\
Defining $\mu = E[X]$, we say that $E[(X-\mu)^m]$ is the center moment.\\
We will denote $E[g(x)]=\int_{-\infty}^{+\infty} g(x)\, d F_X(x)$

\subsection{Joint probability}
We define the joint probability of two variables as
\begin{equation*}
  $$F_{XY}(x,y)=P[X\le x , Y \le y] \stackrel{\text{cont. case}}{=} \int_{-\infty}^{+\infty}\int_{-\infty}^{+\infty} f_{XY}(\eta,\xi) d\eta d\xi$$
\end{equation*}

From what we obtain in \ref{sec:pmd}, we can write \\
\begin{equation*}
  $$F_X(x)=F_{XY}(x,+\infty)=\lim\limits_{y \to +\infty} F_{XY}(x,y)$$
  $$F_Y(y)=F_{XY}(+\infty,y)=\lim\limits_{x \to +\infty} F_{XY}(x,y)$$
\end{equation*}
If X and Y are independent ($X \indep Y$), we have
\begin{itemize}
  \item $F_{XY}(x,y)=F_X(x)\cdot F_Y(y) \; \forall x,y$
  \item $cov(X,Y) = E[(X - \mu_X)\cdot (Y - \mu_Y)] = E[X \cdot Y]-\mu_X \cdot \mu_Y$
\end{itemize}
If X and Y are uncorrelated ($X \bot Y$) $\implies cov(X,Y)=0$.

\subsection{Conditional probability}
\begin{equation}
  $$P[A | B]= \frac{P[A \cap B]}{P[B]} \; \text{with $P[B]\ne 0$}$$ \\
  $$\implies P[A \cap B ] = P[A | B ]\cdot P[B]$$
\end{equation}\\
From the total probability law we can write\\
$$P[A]= \sum\limits_{i=0}^{+\infty}P[A|B]\cdot P[B_i]$$

Suppose we have $X \indep Y \text{ and } Z=X+Y$. Then $F_Z(z) = P[Z\le z] = P[X+Y \le z]$
Calculating F(z) can be more difficult with two variables. We will proceed this way:
\\
\begin{equation}
  $$
  \begin{split}
    E_{Y}[P[X+Y \le z | Y]]  &\stackrel{\text{if X,Y are cont.}}{=} E_Y [P[X \le Z - Y | Y]] = E_Y[F_X(z-Y)]\\
    &=\int_{-\infty}^{+\infty}F_x(z-y) \cdot dF_Y(y) \\
    &\implies f_Z(z) = F'_z(z) = \int_{-\infty}^{+\infty}{f_X(z-y)\cdot f_Y(y) dy} \\
    &=\int_{0}^{z}{f_X(z-y) \cdot f_Y(y) dy}
  \end{split}
  $$
\end{equation}

Test
\subsection{K.T. pp. 10-13}
\begin{equation}
  $$\phi(t) = \int_{-\infty}^{+\infty}e^{\imath \lambda t} dF(\lambda) = E[e^{\imath t x}]\stackrel{\text{ x cont.}}{=}
  \int_{-\infty}^{+\infty}e^{\imath \lambda t} \cdot p(\lambda) d\lambda
  \\ \implies$$
\end{equation}
\begin{enumerate}
  \item $\phi(t)$ fully describes the statistics of X (the Fourier transform is a  one-to-one map )
  \item The characteristic function of the sum of independent variables is the product
  \item $\phi'(t)=\frac{d}{dt} E[e^{\imath \cdot t \cdot x}]=E[\imath \cdot x \cdot e^{\imath \cdot t \cdot x}]$ \\
  if t=0 $\implies E[\imath \cdot x]=\imath \cdot E[ X] = \phi'(0)$
  \item $\phi^{(k)}(0) = (\imath)^k \cdot E[X^k] \implies \phi(t)$ is the \textbf{moment-generating function}
\end{enumerate}


For integer and non negative r.v. :
$$g(s)=\sum\limits_{k=0}^{+\infty}p_k \cdot s^k = E[S^X]$$

$$\phi(t) = g(e^{\imath \cdot t})$$

Doing the derivative of g(s) we obtain:
\begin{equation}
   $$
  \begin{split}
    & g'(s) = \sum\limits_{k=0}^{+\infty} p_k \cdot k \cdot s^{k-1}
    \implies g'(1) = \sum\limits_{k=0}^{+\infty} k \cdot p_k \cdot = E[X] \\
    & g''(s) = \sum\limits_{k=0}^{+\infty} p_k \cdot k \cdot (k-1) \cdot s^{k-2}
    \implies g''(1) = \sum\limits_{k=0}^{+\infty} k \cdot p_k = E[X^2]-E[X] \\
    &\implies E[X^2] = g''(1)+g'(1) \; \; var(X) = g''(1)+g'(1)-(g'(1))^2
  \end{split}
  $$
\end{equation}

Now, let $X_1, X_2, \dotsc \; \text{ and }\; N \in \mathbb{N}$ a random number. Then $R=\sum\limits_{i=1}^{N}x_i$
is a random, because the sum is made of random $x_i$ random terms and $N$ is a random number.

\begin{equation}$$
  \begin{split}
    g_R(s)&=E[s^R]=E[s^{X_1+X_2+\dotsc+X_N}]=\sum\limits_{n=0}^{+\infty} P[N=n]\cdot E[s^{\sum\limits_{i=0}^{n}X_i}|N=n]\\
    &=\sum\limits_{n=0}^{+\infty} P[N=n]\cdot E[g(s)]^n=g_N(g(s))
    \\ & \implies g_R(s) = g_N(g(s))
  \end{split}$$
\end{equation}
The mean and the variance of the moment generating functions can be calculated as follows:
\begin{equation}
$$
\begin{split}
  g'_R(s) &= g'_N(g(s)) \cdot g'_(s) \\
  g'_R(1) &= g_N'(1)\cdot g'(1) = E[N]\cdot E[X]=E[R]\\
  &=E[N^2-N]\cdot (E[X])^2 + E[N]\cdot E[X^2-X] \\
  &=E[N^2]\cdot E[X]^2 - E[N] \cdot E[X]^2 +E[N]\cdot E[X^2] - E[N]\cdot E[X]\\
  & \implies var(R) = E[R^2]-E[R]^2 = g''_R(1)+g'_R(1)-[g'_R(1)]^2\\
  &=E[N^2] \cdot E[X]^2 + E[N] \cdot var(X) - E[N]\cdot E[X] + E[N] \cdot E[X] - (E[N] \cdot E[X])^2 \\
  &= E[X]^2 \cdot (E[N^2]-E[N]^2) + E[N] \cdot var(X)
\end{split}
$$
\end{equation}

\textbf{SO}
\begin{itemize}
  \item $E[R] = E[N] \cdot E[X]$
  \item $var(R) = E[N] \cdot var(X) + var(N) \cdot E[X]^2$
\end{itemize}

What we saw just above is used in queueing systems when transmitting: queue length is N, random,
and the queue has packets of length X\textsubscript{i}.

\section{Distributions}
\subsection{Bernoulli}
Let $X \in \{0,1\}, \, p = P[X=1]$, then $E[X]=p, \, var(X)=p \cdot (1-p)$
We define $\chi(A)$ as the indicator function.

\subsection{Binomial}
Let n be the number of i.i.d experiments with $P[success]=p$. Let Y be the counter of the successes:
$Y=\sum\limits_{i=1}^{n} x_i$. We get $P[Y=k]= \frac{n! \cdot p^k \cdot (1-p)^{n-k}}{k! \cdot (n-k)!}$

\subsection{Geometric}
Let Z be the counter of failure prior to the first success. We have $P[Z=k] = p \cdot (1-p)^{k-1}$ with $k=0,1,\dots$, \\
We get $E[Z]=\frac{1-p}{p} \; var(Z)=\frac{1-p}{p^2}$.
Sometimes books refer to $Z'=Z+1$ as the number of failure with the first success. In that case we have:\\
$E[Z']=\frac{1}{p} \; var(Z')=\frac{1-p}{p^2}=var(Z)$

\subsection{Poisson}
Let $X$ be a Poisson r.v. with parameter $\lambda$. Then $P[X=k] = \lambda^k \cdot \frac{e^{-\lambda}}{k!}$ with $k=0,1,\dots$
\\
$E[X]=\sum\limits_{k=0}^{+\infty}k \cdot \frac{\lambda^k \cdot e^{-\lambda}}{k!}= \lambda \sum\limits_{k=1}^{+\infty}\frac{\lambda^{k-1}\cdot e^{-\lambda}}{(k-1)!}=\lambda$
and $var(X)=\lambda$

We also can say that if $n \to +\infty$, the binomial can be approximated as a Poisson r.v.

\subsection{Exponential}
Let $T$ be an exponential r.v. in the form of
$$f_T(t)=
\begin{cases}
  \lambda \cdot e^{-\lambda \cdot t} & t \ge 0 \\
  0 & t <0
\end{cases}
$$
\\
Then $E[X]= \frac{1}{\lambda} \; var(X)=\frac{1}{\lambda^2}$

What is the value of $P[T-t > x | T>t]$?

\begin{equation}
  $$P[T-t > x | T>t]=\frac{P[T-t > x , T>t]}{P[T>t]}=\frac{P[T > t+x ]}{P[T>t]}=\frac{1-P[T \le t+x]}{1-P[T\le t]}=\frac{e^{-\lambda \cdot (t+x)}}{e^{-\lambda \cdot t}} = e^{-\lambda \cdot x}$$
\end{equation}
We proved that the exponential r.v. is memoryless

\subsection{Uniform}
$E[\mathbb{U}]=\frac{a+b}/2 \, var(\mathbb{U})=\frac{(b-a)^2}{12}$

\#TODO\# grafico uniforme

\subsection{Gamma}
Let $\alpha >0 , \lambda >0$. Then the PDF of a gamma-distributed r.v. is
\begin{equation}
  $$f(x) = \frac{\lambda}{\Gamma(\alpha)}\cdot (\lambda \cdot x)^{\alpha -1} \cdot e^{-\lambda \cdot x} \text{ with } x\ge 0$$
\end{equation}

For the expectation and the variance, we have: $E[G]=\frac{\alpha}{\lambda} \, var(G)=\frac{\alpha}{\lambda^2}$
\\ If $\alpha \in \mathbb{N} \implies \Gamma(\alpha) = (\alpha-1)!$ and the gamma
distribution is the sum of $\alpha$ i.i.d exponential r.v. of parameter $\lambda$.

\begin{equation} $$
  \begin{split}
  E[X]&\stackrel{discrete}{=} \sum\limits_{k=0}^{+\infty} k \cdot P[X=k]
  \stackrel{\text{can be shown  that}}{=} \sum\limits_{k=0}^{+\infty} P[X>k]\\
  &=0 \cdot p(0) + 1 \cdot p(1) + 2 \cdot p(2) + \dots \\
  &= p(1) + p(2) + p(3)+ \dots \qquad P[X>0]\\
  & \qquad \qquad + p(2) + p(3) +\dots \quad P[X>1]\\
  & \qquad \qquad \qquad \qquad + p(3)+ \dots \quad P[X>2] \\
  & \dots
  \end{split}$$
\end{equation}
where $P[X=k]=p(k)$ and $P[X>k]$ is the complementary distribution

In the continuous case
\begin{equation}$$
  \begin{split}
    E[X] &= \int_{0}^{+\infty} 1-F(x) dx = \int_{0}^{+\infty} x \cdot f(x) dx
    &=\int_{0}^{+\infty} e^{-\lambda \cdot x} dx = \frac{1}{\lambda}
  \end{split}$$
\end{equation}

\section{Chapter 2 K. T. }
We saw the Total Probability theorem and how to use it to find a probability with subsetting.\\
Let's now extend it to the composite PMD. Let $X,Y$ be two indipendent r.v, with $P[Y=y]\neq 0$.
Then, $\forall x,y$
\begin{equation}$$
  \begin{split}
    P[X\le x, Y \le y] &= \sum \limits_{\eta \le y} P[X \le x , Y \le \eta] \\
    &\stackrel{discrete}{=} \sum \limits_{\eta \le y} F_{X|Y}(x | \eta) \cdot P[Y=\eta]\\
    &\stackrel{continuous}{=} \int_{\eta \le y} F_{X|Y}(x|\eta) dF_Y(\eta)
  \end{split}$$
\end{equation}
We can try to find the expectation of $g(x)$ which is a function of Y using the conditioning. So we obtain:
\begin{equation}
  $$E[g(x)]=E[E[g(x)|Y]] = \int_{-\infty}^{+\infty}E[g(x)|Y=\eta] dF_Y(\eta)$$
\end{equation}

\textbf{SO}
\begin{equation}
  \begin{split}
   &\forall x,y \\
  &P[X\le x , Y \le y] = \int_{\eta \le y} F_{X|Y}(x|\eta) dF_Y(\eta)\\
  &\forall x \quad P[X \le x] = \int_{-\infty}^{+\infty}F_{X|Y}(x|\eta) dF_Y(\eta) \\
  &E[g(x)]=\int_{-\infty}^{+\infty}
  E[g(x)|Y=\eta] dF_Y(\eta)
  \end{split}
\end{equation}

\subsection{Example binomial}
EMPTY

\subsubsection{Transform approach}
\begin{equation}
  \begin{split}
  g_N(s)=\sum \limits_{n=1}^{+\infty} \beta \cdot (1-\beta)^{n-1}\cdot s^n = \frac{\beta \cdot s}{1-(1-\beta)\cdot s}\\
  \phi_T=E[e^{\jmath \cdot t \cdot x}] = \int_{0}^{+\infty} e^{\jmath \cdot t \cdot x} \cdot \lambda \cdot e^{ \lambda \cdot x} dx = \frac{\lambda}{\lambda - \jmath \cdot t} \\
  \phi_Z = g_N(\phi_T(t))=\frac{\beta \cdot \frac{\lambda}{\lambda - \jmath \cdot t}}{1-(1-\beta)\cdot \frac{\lambda}{\lambda - \jmath \cdot t}} =
   \frac{\beta \cdot \lambda}{\lambda - \jmath \cdot t -\lambda - \beta \cdot \lambda} = \frac{\beta \cdot \lambda}{\beta \cdot \lambda - \jmath \cdot t}
  \end{split}
\end{equation}
In the last row we can see how the the composition changed only the parameter: $\lambda$ becomes $\beta \cdot \lambda$.

We recall that $f_{X|Y} = \frac{f_{XY}(x,y)}{f_Y(y)}$ with $f_Y(y)\neq 0$ in the considered interval.
We can then derive
\begin{equation}
  \begin{split}
  F_{X|Y}(x|y) = \int_{-\infty}^{x}f_{X|Y}(\xi,y)d\xi\\
  \forall x,y \quad P[X \le x , Y \le y] &= \int_{-\infty}^{x}d\xi \int_{-\infty}^{y}f_{XY}(\xi,\eta) d\eta\\
  &=\int_{-\infty}^{y}f_Y(\eta) \int_{-\infty}^{x}f_{X|Y}(\xi,\eta) d\xi d\eta \\
  =&\int_{-\infty}^{y}F_{X|Y}(x,\eta) dF_Y(\eta)
  E[g(x)] &= \int_{-\infty}^{+\infty} \int_{-\infty}^{+\infty}g(\eta) f_{XY}(\xi,\eta) d\eta  d\xi \\
  &= \int_{-\infty}^{+\infty} f_Y(\eta) \int_{-\infty}^{+\infty}g(\eta) f_{X|Y}(\xi,\eta)   d\xi d\eta \\
  &= \int_{-\infty}^{+\infty}e[g(x)|Y=\eta] dF_Y(\eta)
  \end{split}
\end{equation}
