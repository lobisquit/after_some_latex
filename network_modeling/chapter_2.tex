%%%
\chapter{Markov Chains}

\section{Long Run Behaviour}

\subsection{Steady-state probabilities}

\begin{definition}[Regular \gls{mc}]
	A regular \gls{mc} is a \gls{mc} with the following property:

	\beq \lim_{n \to \infty} P_{ij}^{(n)} = \lim_{ n \to \infty} Prob[ x_n=j | x_0 =i] = \pi_j > 0 \quad \forall i, j \eeq
\end{definition}

This tell us that:
\begin{enumerate}
\item Limit exists (not obvious)
\item Limit is indipendent of the initial state
\item Limit is strictly positive
\end{enumerate}


%%%%%
\begin{theorem}
For a regular \gls{mc} with states 0,1,...,N the limit distribution $\bm\pi = (\pi_0,\pi_1,\cdots,\pi_N)$ is the unique solution of the following system of equations:

\begin{align}
	&\pi_j = \sum_{k=0}^N \pi_k P_{k j} , \qquad ~for ~j = 0,1, \cdots, N \\
	&\sum_{k=0}^N \pi_k = 1, \qquad \qquad \pi_k \ge 0 \quad \forall k
\end{align}

\end{theorem}

---
\begin{proof} Let's start easy by proving the \textbf{existence}:

\begin{equation}
  P_{i j}^{(n)} = \sum_{k=0}^N P_{ik}^{(n-1)} P_{k j}
	\qquad with ~\sum_{k=0}^N P_{ik}^{(n)} = 1 ~\forall n
\end{equation}
it's the developing of $\bm P^n = \bm P^{n-1} \bm P$

Now let's study what happens as $ n \to \infty $:
\begin{equation}
\begin{split}
	&\pi_j = \lim_{n \to \infty} P_{ij}^{(n)} = \lim_{n \to \infty} \sum_{k=0}^N P_{ik}^{(n-1)} P_{k j
	} =\\
	&= \sum_{k=0}^N \lim_{n \to \infty} P_{ik}^{(n-1)} P_{k j
	} = \sum_{k=0}^N \pi_k P_{kj}
\end{split}
\end{equation}
Here the limit and the sum can be switched since the sum is finite.
This shows that the system have a solution.

Now let's prove \textbf{uniqueness}:


\begin{itemize}

\item Let $x_j$ be a solution
\item $x_j = \sum_{k=0}^N x_k P_{kj} $ / by construction
\item $x_l = \sum_{j=0}^N x_j P_{jl} =  \sum_{j=0}^N ( \sum_{k=0}^N x_k P_{kj} ) P_{jl} =  \sum_{k=0}^N x_k \sum_{j=0}^N P_{kj} P_{jl} = \sum_{k=0}^N x_k P_{kl}^{(2)}$

If we apply this trick again n times we can prove by induction  that:

\item $x_j$ also satisfy $ x_j = \sum_{k=0}^N x_k P_{kj}^n $ \quad  $\forall n $

Now, as $n \to \infty$ :

\item  $x_j = \sum_{k=0}^N x_k \pi_j = \pi_j (\sum_{k=0}^N x_k) = \pi_j  \Rightarrow $ Solution is unique

\end{itemize}

\end{proof}
%%

\subsection{Classes of states}
\begin{definition}[Accessible State]

State $j$ is {\bfseries accessible} from state $i$ ($i \rightarrow j$) if $\exists n \geq 0$ such that $P_{ij}^n > 0$
\end{definition}

\begin{definition}[Communicant States]

States $i$ and $j$ are said to {\bfseries communicate} ( $ i \leftrightarrow j$ ) if

 $i \rightarrow j$ and $j \rightarrow i$
\end{definition}
%%%%

{\bfseries Proprieties:}
\begin{enumerate}
\item Reflexivity: \quad $i \leftrightarrow i$
\item Symmetry: \quad if $i \leftrightarrow j$ then $j \leftrightarrow i$
\item Transitivity: \quad if $i \leftrightarrow j$ and $j \leftrightarrow k \Rightarrow i \leftrightarrow k$
 \end{enumerate}
Communication among states is an equivalence relation.

\begin{definition}[Periodicity]
The period of state i, $d(i)$, is the GCD (greatest common denominator) of set $S_i = \{ s>0 : P_{ii}^{(s)} >0 \}$
\end{definition}

\begin{theorem} Periodicity is a class of propriety:
\begin{itemize}
\item if $i \leftrightarrow j$, then $d(i) = d(j)$
\end{itemize}
\end{theorem}
--
\begin{proof}
given $$S_i = \{ s>0 : P_{ii}^{(s)} >0 \}$$

let $$n, m >0 : P_{ij}^{(m)} >0, P_{ij}^{(m)} > 0$$

we have that $$\forall s \in S_i : P_{ii}^{(s)} > 0$$

now, for total probability theorem$$P_{jj}^{(n+s+m)} = \sum_{h, k} P_{jh}^{(n)} P_{hk}^{(s)} P_{kj}^{(m)} \geq P_{ji}^{(n)} P_{ii}^{(s)} P_{ij}^{(m)} >0 $$

$$P_{jj}^{(n+s+m)} >0 \Rightarrow n+s+m \in S_j$$

$$P_{jj}^{(n+2s+m)} \geq P_{ji}^{(n)} P_{ii}^{(s)} P_{ii}^{(s)} P_{ij}^{(m)} >0 \Rightarrow n+2s+m \in S_j$$

Now let's call $d(j) =$ g.c.d. of $S_j$

$ \Rightarrow n+s+m$ and $n+2s+m$ are integer multiples of $d(j)$

Therefore s is an integer multple of $d(j)$ ?? bo raga non capisco dai miei appunti

$\forall s \in S_i$ is integer multiple of $d(j)$ $\Rightarrow d(j)$ is a common divisor of $S_i$ and so it divides the g.c.d. of $S$, $d(i)$

$\Rightarrow d(i)$ is an integer multiple of $d(j)$.

Doing this proof again switching role of i and j we prove that $d(j)$ is an integer multiple of $d(i)$

Therefore, $d(i) = d(j)$

\end{proof}
--

\begin{definition}[Return Time]
the return time to state $i := R_i$
\end{definition}

So the probability of ever going back to state i can be written as
$$f_{ii} = \sum_{n=1}^\infty f_{ii}(n) $$

\begin{definition}[Recurrent state]
a state is recurrent if $f_{ii} = 1$
\end{definition}

\begin{definition}[Transient state]
a state is transient if  $f_{ii} < 1$
\end{definition}

\begin{definition}
$M$ is the number of returns to state $i$
\end{definition}

$$\mathbb{E}[M | x_0 = i] = \sum_{k=1}^\infty \mathbb{P}[M\geq k | x_0 = i] = \sum_{k=1}^\infty f_{ii}^k = \begin{cases}
\frac{f_{ii}}{1-f_ii}, & \mbox{if transient} \\
\infty, & \mbox{if recurrent}
\end{cases}$$

\begin{definition}[Proper r.v]
given X a finite value r.v. it is called proper if $$\lim_{x\to \infty} \mathbb{P}[x\geq a] = 0$$
\end{definition}

\begin{definition}[Improper r.v]
given X a r.v. it is called improper if  $$\lim_{x\to \infty} \mathbb{P}[x\geq a] = p_\infty \ne 0$$
\end{definition}


\begin{theorem}
$$i  \mbox{ is recurrent} \iff \sum_{n=1}^\infty P_{ii}^{(n)} = \infty$$
\end{theorem}
--
\begin{proof}
Number of time state i has been visited: $$M = \sum_{n=1}^\infty \mathds{1}\{x_n =i\} $$
The following swap from expected value and infinite sum is allowed by Fubini's Theorem$$ \mathbb{E}[M | x_0 = i] = \mathbb{E}[\sum_{n=1}^\infty \mathds{1}\{x_n = i\} | x_0 = i] = \sum_{n=1}^\infty \mathbb{E}[\mathds{1}\{ x_n = i\} | x_0 = i] = \sum_{n=1}^\infty P_{ii}^{(n)} $$
So we have shown that $\mathbb{E}[M | x_0=i] = \sum_{n=1}^\infty P_{ii}^{(n)}$.\\
Now, remembering that$$ \mathbb{E}[M | x_0 = i] = \begin{cases}
\frac{f_{ii}}{1-f_ii}, & \mbox{if transient} \\
\infty, & \mbox{if recurrent}
\end{cases}$$
the proof is concluded.
\end{proof}
%%%%%%%%%%%%%%%%%%%%%%

\begin{theorem}
(not to confuse with theorem on periodicity)
$$\mbox{if } i\leftrightarrow j \mbox{and i is recurrent} \Rightarrow \mbox{ j is also recurrent}$$
\end{theorem}
--
\begin{proof}
$$\mbox{Since } i\leftrightarrow j : \exists m,n \mbox{ such that } P_{ij}^{(n)} \mbox{ and } P_{ji}^{(m)} > 0$$
$$\mbox{let } r>0 : P_{jj}^{(m+r+n)} = \sum_{r, k} P_{jh}^{(m)} P_{hk}^{(r)} P_{kj}^{(n)} \geq P_{ji}^{(m)}  P_{ii}^{(r)}  P_{ij}^{(n)}$$
$$\sum_{l=1}^\infty P_{jj}^{(l)} \geq \sum_{r=1}^\infty P_{jj}^{(m+r+n)} \geq \sum_{r=1}^\infty P_{ji}^{(m)}  P_{ii}^{(r)}  P_{ij}^{(n)} = P_{ji}^{(m)} P_{ij}^{(n)} \sum_{r=1}^\infty P_{ii}^{(r)}$$
but since \begin{itemize}
\item$i$ is recurrent $\Rightarrow \sum_{n=1}^\infty P_{ii}^{(n)} = \infty$
\item $P_{ij}^{(n)} \mbox{ and } P_{ji}^{(m)} > 0$
\end{itemize}
$$\sum_{l=1}^\infty P_{jj}^{(l)} = \infty \Rightarrow j \mbox{ is recurrent}$$

\end{proof}



\begin{theorem}
(not to confuse with theorem on periodicity)
$$\mbox{if } i\leftrightarrow j \mbox{and i is recurrent} \Rightarrow \mbox{ j is also recurrent}$$
\end{theorem}
--
\begin{proof}
$$\mbox{Since } i\leftrightarrow j : \exists m,n \mbox{ such that } P_{ij}^{(n)} \mbox{ and } P_{ji}^{(m)} > 0$$
$$\mbox{let } r>0 : P_{jj}^{(m+r+n)} = \sum_{r, k} P_{jh}^{(m)} P_{hk}^{(r)} P_{kj}^{(n)} \geq P_{ji}^{(m)}  P_{ii}^{(r)}  P_{ij}^{(n)}$$
$$\sum_{l=1}^\infty P_{jj}^{(l)} \geq \sum_{r=1}^\infty P_{jj}^{(m+r+n)} \geq \sum_{r=1}^\infty P_{ji}^{(m)}  P_{ii}^{(r)}  P_{ij}^{(n)} = P_{ji}^{(m)} P_{ij}^{(n)} \sum_{r=1}^\infty P_{ii}^{(r)}$$
but since \begin{itemize}
\item$i$ is recurrent $\Rightarrow \sum_{n=1}^\infty P_{ii}^{(n)} = \infty$
\item $P_{ij}^{(n)} \mbox{ and } P_{ji}^{(m)} > 0$
\end{itemize}
$$\sum_{l=1}^\infty P_{jj}^{(l)} = \infty \Rightarrow j \mbox{ is recurrent}$$

\end{proof}

\begin{theorem}[Basic limit theorem on MC]
Consider an irreducible aperiodic recurrent MC (an aperiodic recurrent class), we have
$$ \lim_{n\to \infty} P_{ii}^{(n)} = \frac{1}{m_i} = \pi_i = \lim_{n\to\infty} P_{ji}^{(n)} \qquad \forall j$$
\end{theorem}
--
%%
%DA SISTEMARE LA TABELLA

\begin{center}
    \begin{tabular}{|l*{6}{|c}r}
        \hline
    state $i$ & $f_{ii} = \sum_{n=1}^\infty f_{ii}^{(n)}$  & $\lim_{k \to \infty } \mathbb{P}[M \geq k | x_0=i]$ & $\mathbb{E}[n|x_0=i]$ & $m_i = \sum_n f_{ii}^{(n)}$ & $\pi_i = \frac{1}{m_i}$  \\ \hline

    transient & $<1$ & 0 & $\frac{f_{ii}}{1-f_{ii}}$ & $\infty$ & 0 \\ \hline
    Null recurrent & 1 & 1 & $\infty$ & $\infty$ & 0 \\ \hline
    Positive recurrent & 1 & 1 & $\infty$ & $<\infty$ & $>1$ \\ \hline
        \end{tabular}
\end{center}

%%
\begin{theorem}
For an aperiodic positive recurrent class (irreducible MC), $\pi_j$ is the unique solution of

$$\begin{cases}
\pi_j = \sum_{i=0}^\infty \pi_i P_{ij} \\
\sum_{i=0}^\infty \pi_i = 1 \\
\pi_i \geq 0 \quad \forall i
\end{cases}$$
\end{theorem}
--
\begin{proof} The funny thing is that this proof is easier than the book:
\begin{enumerate}
\item we first want to show that the $\pi_j$ satisfy the system (\textbf{\textit{Existence of the solution}})
$$ \forall m,n \qquad 1=\sum_{j=0}^\infty P_{ij}^{(n)} > \sum_{j=0}^m P_{ij}^{(n)}$$
$$ \lim_{n\to\infty} \sum_{j=0}^m P_{ij}^{(n)} = \sum_{j=0}^m \pi_j \leq 1 \quad \forall n $$
\begin{equation}
\Rightarrow \sum_{j=0}^\infty \pi_j \leq 1
\end{equation}

\item
$$P_{ij}^{(n+m)} \geq \sum_{k=0}^m P_{ik}^{(m)} P_{kj}^{(n)} \quad \forall n, m, M$$
$$ \mbox{as } m \to \infty :\quad \pi_j \geq \sum_{k=0}^m \pi_k P_{kj}^{(n)} $$
\begin{equation}
\Rightarrow  \pi_j \geq \sum_{k=0}^m \pi_k P_{kj}^{(n)}
\end{equation}

\item
$$  \sum_{k=0}^\infty \sum_{j=0}^\infty \pi_k P_{kj}^{(n)} \geq
      \sum_{k=0}^m \sum_{j=0}^\infty \pi_k P_{kj}^{(n)}  =
      (\mbox{since } \sum_{j=0}^\infty P_{kj}^{(n)} = 1 ) =
      \sum_{k=0}^m \pi_k \quad \forall m $$
$$ \mbox{suppose } \exists j > 1 : \pi_j > \sum_{k=0}^\infty \pi_k P_{kj}^{(n)} $$
$$ \mbox{then we have that } \sum_{j=0}^\infty \pi_k > \sum_{j=0}^\infty \sum_{k=0}^\infty \pi_k P_{kj}^{(n)} > \sum_{k=0}^\infty \pi_k \quad \mbox{ABSURD.} $$
\begin{equation}
\Rightarrow \pi_j = \sum_{k=0}^\infty \pi_k P_{kj}^{(n)}
\end{equation}

\item
$$ \pi_j = \sum_{k=0}^\infty \pi_k P_{kj}^{(n)}, \qquad |P_{kj}^{(n)}| \leq 1 \quad \forall n,i,k $$
using an appropriate theorem for sliding the limit inside the infinite sum we have:
$$\pi_j = \sum_{k=0}^\infty  \pi_k \lim_{n\to\infty} P_{kj}^{(n)} = (\sum_{k=0}^\infty \pi_k) \pi_j$$
and now since $\pi_j>0$
\begin{equation}
\Rightarrow \sum_{k=0}^\infty \pi_k = 1
\end{equation}
\textbf{\textit{This concludes the existence proof.}}

\item
Let's now prove the \textbf{\textit{Uniqueness}} \\
Suppose $x_j$ is a solution

$$x_j =
 \sum_{i=0}^\infty x_i P_{ij} =
 \sum_{i=0}^\infty ( \sum_{k=0}^\infty x_k P_{ki} ) P_{ij} \geq
 \sum_{k=0}^m x_k \sum_{i=0}^\infty P_{ki} P_{ij} =
 \sum_{k=0}^m x_k P_{kj}^{(2)}
 \quad \forall m
$$
$$ \Rightarrow x_j \geq \sum_{k=0}^\infty x_k P_{kj}^{(n)}\qquad \forall n $$

This is the same result of $3^{rd}$ step, so as in $3^{rd}$ step, we can prove by contradiction that this inequality is in fact an equality.

$$\Rightarrow x_j = \sum_{k=0}^\infty x_k P_{kj}^{(n)} \quad \forall n$$
as $n \to \infty$ we have:
\begin{equation}
x_j = (\sum_{k=0}^\infty x_k ) \pi_j \Rightarrow x_j = \pi_j
\end{equation}
So it's unique.

\end{enumerate}
\end{proof}

\begin{lemma}
  If $0 < p_i < 1 ~,~ i=0,1,2.\cdots $, then:
	\begin{equation}\label{limprodpi}
	  \lim_{m \to \infty} \prod_{i=0}^{m}(1-p_i) = 0
	\end{equation}
	if and only if
	\begin{equation}\label{pitoinfty}
	  \sum_{i=0}^\infty p_i = \infty
	\end{equation}
\end{lemma}
\begin{proof}
	\begin{enumerate}
	  \item 	Assume \eq{pitoinfty} is true. \\
			Since the series expansion for $exp(-p_i)$ is an alternating series with terms decreasing in absolute value, we can write:
			\begin{equation}
			  1-p_i < 1-p_i + \frac{p_i^2}{2!} - \frac{p_i^3}{3!} + \cdots = ~exp(-p_i) \quad with ~i\ge 0
			\end{equation}
			applying the product to both members we obtain
			\begin{equation}
			  \prod_{i=0}^{k-1} (1-p_i) < e^{-\sum_{i=0}^{k-1}p_i}
			\end{equation}
			But by assumption $\sum_{i=0}^\infty p_i = \infty$ hence,
			$$ \lim_{m \to \infty} \prod_{i=0}^{m}(1-p_i) = 0 $$
		\item Let's prove the following inequality:
		$$ \prod_{i=j}^m(1-p_i) > 1-\sum_{i=j}^m p_i \quad \forall m \ge j+1$$
		We can prove it recursively:
		$$(1-p_j)(1-p_{j+1}) = 1-p_j - p_{j+1} + p_jp_{j+1} > 1-p_j - p_{j+1}$$
		Iterating we obtain:
		\begin{eqnarray*}
		\prod_{i=j}^{m+1}(1-p_i) = (1-p_{m+1})\prod_{i=j}^m(1-p_i) > (1-p_{m+1})(1-\sum_{i=j}^m p_i) = \\
		1- \sum_{i=j}^{m+1} p_i + p_{m+1}\sum_{i=j}^m p_i
		\end{eqnarray*}
		Assume now that $\sum_{i=1}^\infty p_i < \infty$, then there must exist an index $j>1$ s.t. $\sum_{i=j}^\infty p_i < 1$. \\
		Then we have:
		$$ \lim_{m \to \infty} \prod_{i=j}^m (1-p_i) > \lim_{m \to \infty} (1-\sum_{i=j}^m p_i) > 0 $$
	\end{enumerate}
\end{proof}

\begin{definition}[lezione 22/03/17]
$$\Pi_{ik}^n(c) = \mathbb{P}[x_n = k \in c, x_l \notin c, l=1, ... , n-1 | x_0 =i]$$
$c$ is an aperiodic recurrent class, $i$ is a transient state.
$$\Pi_i^n(c) = \sum_{k \in c} \Pi_{ik}^n(c); \qquad \Pi_i(c) = \sum_{n=1}^\infty \Pi_i^n(c)$$
\end{definition}

\begin{theorem}
Let $i$ be a transient state and$j \in c$, where $c$ is an aperiodic recurrent class:
$$\lim_{n\to\infty} P_{ij}^{(n)} = \Pi_i(c) \lim_{n\to\infty} P_{jj}^{(n)} = \Pi_i(c) \Pi_j$$
\end{theorem}





%\begin{theorem}
%Let $f$ be a function whose derivative exists in every point, then $f$ is
%a continuous function.
%\end{theorem}

%\begin{theorem}[Pythagorean theorem]
%\label{pythagorean}
%This is a theorema about right triangles and can be summarised in the next
%equation
%\[ x^2 + y^2 = z^2 \]
%\end{theorem}

%And a consequence of theorem \ref{pythagorean} is the statement in the next
%corollary.

%\begin{corollary}
%There's no right rectangle whose sides measure 3cm, 4cm, and 6cm.
%\end{corollary}

%You can reference theorems such as \ref{pythagorean} when a label is assigned.

%\begin{lemma}
%Given two line segments whose lengths are $a$ and $b$ respectively there is a
%real number $r$ such that $b=ra$.
%\end{lemma}

%\begin{proof}
%To prove it by contradiction try and assume that the statemenet is false,
%proceed from there and at some point you will arrive to a contradiction.
%\end{proof}
