\chapter{Markov Chains}

\section{Introduction}
A markov process is a process $X_n \text{ or } X_t$ s.t.
\begin{equation}
	\prob[X_{n+1} = j | X_0 = i_0, X_1 = i_1, \dots, X_n = i_n] = \prob[X_{n+1} = j | X_n=i_n] = P
\end{equation}

with P as the \textit{one-step probability}. From now on, the notation for that probability will be $P_{i,j}^{n,n+1}=P_{i,j} \quad \forall n$ if the MC is homogeneous, i.e. the transition probabilities are stationary.

We can suppose $n\ge 0$ and we'll have the \textbf{transition probability matrix}, which will characterize the process as
\begin{equation} P=\begin{pmatrix}
	P_{00} & P_{01} & \cdots  \\
	P_{10} & P_{11} & \cdots \\
	\vdots & \vdots & \ddots  \\
 	\end{pmatrix}
	\text{with } P_{i,j}\ge 0 \forall i,j \quad \text{ and } \sum\limits_{j=0}^{+\infty}P_{i,j} = 1 \quad \forall i
\end{equation}

For example
\begin{equation}
	\prob[X_3 = i_3 , X_2 = i_2, X_1 = i_1, X_0 = i_0]=P_{i_0}\cdot P_{i_0,i_1}\cdot P_{i_1,i_2}\cdot P_{i_2,i_3}
\end{equation}

The description of a markov process is given from the initial state probability and the transition probability matrix.

\section{Stability of a process}
Suppose a system where customers arrive independently and are queued in the system before getting served.
Suppose the service time occupies only one slot and all the slots for service time have same duration.
If the queue is empty, that time slot is wasted.
Now let the number of users arriving in the n-th slot be $\xi_n$ and the number of users in
the system in the n-th slot  be $X_n$. Then we have:
 \begin{equation}
 	\begin{split}
 	&\prob[\xi_n=k] = a_k \text{, probability of k arriving users at time n}\\
  	&X_{n+1}=
		\begin{cases}
 			X_n -1 +\xi_n &  \text{if } X_n >0 \\
 			\xi_n 			 	&  \text{if } X_n = 0
 		\end{cases}
 	\end{split}
 \end{equation}

The transition probability matrix:
\begin{equation} P=\begin{pmatrix}
	a_{0} & a_{1} & a_2 & \cdots  \\
	a_{0} & a_{1} & a_2 & \cdots  \\
	0			& a_{0} & a_{1} & \cdots  \\
	0 		& 0			& a_{0} & \cdots  \\
	\vdots & \vdots & \ddots &  \\
\end{pmatrix}
\end{equation}
\begin{definition}[Behavior of a MC]
Let $\rho=\sum\limits_{k=0}^{+\infty} k \cdot a_k$. We say that for
\begin{equation} \rho : \begin{cases}
	>1 & \text{\textbf{unstable }: the system will never be able to serve everybody}\\
	=1 & \text{\textbf{unstable \textit{unless deterministic}}: sooner or later the system will fail}\\
	<1 & \text{\textbf{stable}: the queue tends to be empty}
\end{cases}\end{equation}
\end{definition}

\section{Long Run Behaviour}

\subsection{Steady-state probabilities}

	\begin{definition}[Regular \gls{mc}]
		A regular \gls{mc} is a \gls{mc} with the following property:

		\begin{equation} \lim_{n \to \infty} P_{ij}^{(n)} = \lim_{ n \to \infty} Prob[ X_n=j | X_0 =i] = \pi_j > 0 \quad \forall i, j \end{equation}

	\end{definition}

	This tell us that:
	\begin{enumerate}
		\item Limit \textbf{exists} (not obvious)
		\item Limit is indipendent of the initial state
		\item Limit is \textbf{strictly} positive
	\end{enumerate}

	\begin{theorem}
		For a regular \gls{mc} with states 0, 1, ..., N the limit distribution $\bm\pi = (\pi_0,\pi_1,\cdots,\pi_N)$ is the unique solution of the following system of equations:\\

		$\begin{cases}
			\pi_j = \sum\limits_{k=0}^N \pi_k P_{k j} & \text{for } j = 0,1, \cdots, N \\
			\sum\limits_{k=0}^N \pi_k = 1\\
			\\ \pi_k \ge 0 & \forall k
		\end{cases}$
	\end{theorem}

	\begin{proof} of \textbf{existence}:
		\begin{equation}
  			P_{i j}^{(n)} = \sum\limits_{k=0}^N P_{ik}^{(n-1)} P_{k j}
			\qquad with ~\sum\limits_{k=0}^N P_{ik}^{(n)} = 1 ~\forall n
		\end{equation}
		it's the developing of $\bm P^n = \bm P^{n-1} \bm P$

		Now let's study what happens as $ n \to \infty $:
		\begin{equation}
			\begin{split}
				&\pi_j = \lim_{n \to \infty} P_{ij}^{(n)} = \lim_{n \to \infty} \sum\limits_{k=0}^N P_{ik}^{(n-1)} P_{k j
				} =\\
				&= \sum\limits_{k=0}^N \lim_{n \to \infty} P_{ik}^{(n-1)} P_{k j
				} = \sum\limits_{k=0}^N \pi_k P_{kj}
			\end{split}
		\end{equation}
		Here the limit and the sum can be switched since the sum is finite.
		This shows that the system have a solution.
	\end{proof}

	\begin{proof} of \textbf{uniqueness}:
		Let $X_j$ be a solution, so, by construction, $X_j = \sum\limits_{k=0}^N X_k P_{kj} ~$.

		For a given state $l$, it holds that
		\begin{equation}
				X_l = \sum\limits_{j=0}^N X_j P_{jl} =  \sum\limits_{j=0}^N \left( \sum\limits_{k=0}^N X_k P_{kj} \right) P_{jl} =  \sum\limits_{k=0}^N X_k \sum\limits_{j=0}^N P_{kj} P_{jl} = \sum\limits_{k=0}^N X_k P_{kl}^{(2)}
		\end{equation}

		If we apply this trick again $n$ times we can prove by induction that:
		$X_j$ also satisfy $ X_j = \sum\limits_{k=0}^N X_k P_{kj}^{(n)} $ \quad  $\forall n $

		Now, as $n \to \infty$, we have that
		\begin{equation}
			X_j = \sum\limits_{k=0}^N X_k \pi_j = \pi_j (\sum\limits_{k=0}^N X_k) = \pi_j  \implies
			\text{The solution is unique}
		\end{equation}
	\end{proof}

\subsection{Classes of states}
	\begin{definition}[Accessible State]
		State $j$ is {\bfseries accessible} from state $i$ if $\exists n \geq 0$ such that $P_{ij}^{(n)} > 0$. It can be written ($i \rightarrow j$).
		State it's \textbf{not} accessible if $\forall n \ge 0 \quad P_{ij}^{(n)}=0$
	\end{definition}

	\begin{definition}[Communicant States]
		States $i$ and $j$ are said to {\bfseries communicate}, written $ i \leftrightarrow j$, if
		$$\begin{cases}
			i \rightarrow j \\
			j \rightarrow i
		\end{cases}$$
	\end{definition}

	{\bfseries Proprieties}
	\begin{enumerate}
		\item \textbf{Reflexivity}: \quad $i \leftrightarrow i \quad\text{ given } P_{ii}^{(0)}=1$
		\item \textbf{Symmetry}: \quad if $i \leftrightarrow j$ then $j \leftrightarrow i$
		\item \textbf{Transitivity}: \quad if $i \leftrightarrow j$ and $j \leftrightarrow k \Rightarrow i \leftrightarrow k$
	\end{enumerate}
	This implies that communication between states is an equivalence relation.

	\begin{definition}[Periodicity]
		The period of state i, written $d(i)$, is the GCD (greatest common denominator) of set $S_i = \{ s>0 : P_{ii}^{(s)} >0 \}$.\\
		If $d(i)=1$, the state is said to be \textbf{aperiodic}.
	\end{definition}

	\begin{theorem}[Periodicity] Periodicity is a characteristic of groups (called \emph{classes}) of communicating states.
		$$\text{if } i \leftrightarrow j \text{, then } d(i) = d(j)$$
	\end{theorem}

	\begin{proof}
		Given $S_i = \{ s>0 : P_{ii}^{(s)} >0 \}$, and let $n, m > 0 : P_{ij}^{(m)} > 0, P_{ji}^{(n)} > 0$, we have that $$\forall s \in S_i : P_{ii}^{(s)} > 0$$.

		Now, for total probability theorem, it holds that
		$$ P_{jj}^{(n+s+m)} = \sum\limits_{h, k} P_{jh}^{(n)} P_{hk}^{(s)} P_{kj}^{(m)} \geq P_{ji}^{(n)} P_{ii}^{(s)} P_{ij}^{(m)} >0 $$
		$$ P_{jj}^{(n+s+m)} >0 \implies n+s+m \in S_j$$

		$$P_{jj}^{(n+2s+m)} \geq P_{ji}^{(n)} P_{ii}^{(s)} P_{ii}^{(s)} P_{ij}^{(m)} >0 \Rightarrow n+2s+m \in S_j$$

		Now let's call $d(j) =$ g.c.d. of $S_j$.

		Since both $n+s+m$ and $n+2s+m$ are integer multiples of $d(j)$, so it is their difference $s$.

		We have proved that $\forall s \in S_i$ is integer multiple of $d(j)$: this implies that $d(j)$ is a common divisor of $S_i$, and moreover it divides the g.c.d. of $S$, $d(i)$. In conclusion, $d(i)$ is an integer multiple of $d(j)$.

		Doing this proof again switching role of $i$ and $j$ we prove that $d(j)$ is an integer multiple of $d(i)$, and so $d(i) = d(j)$.
	\end{proof}
	---

	\begin{definition}[Return Time]
		The random variable $R_i$ is defined as the time it takes to return to $i$ starting from $i$ itself.
	\end{definition}

	So the probability of ever going back to state $i$ can be written as
	$$ f_{ii} = \sum\limits_{n=1}^\infty f_{ii}(n)  = \sum\limits_{n=1}^\infty \prob[R_i=n | X_0=i] $$
	where $f_{ii}(n)$ is the probability of returning to state $i$ in $n$ steps.

	\begin{definition}[Recurrent state]
		A state is recurrent if $f_{ii} = 1$
	\end{definition}
	In other words we say that a state $i$ is recurrent if and only if, after the process starts from state $i$, the probability of its returning to state i after some finite length of time is one.

	\begin{definition}[Transient state]
		A state is transient if  $f_{ii} < 1$
	\end{definition}

	In other words a state $i$ is said to be transient if there exists a non-zero probability of never coming back to it.\footnote{like Angela. Please come back Angela, I love you.}

	\begin{definition}
		$M$ is the number of returns to state $i$, starting from $i$
		$$\exp[M | X_0 = i] = \sum\limits_{k=1}^\infty \prob[M\geq k | X_0 = i] = \sum\limits_{k=1}^\infty f_{ii}^k = \begin{cases}
		\frac{f_{ii}}{1-f_{ii}}, & \text{if transient} \\
		\infty, & \text{if recurrent}
		\end{cases}$$
	\end{definition}

	\begin{definition}[Proper r.v]
		given X a finite value r.v. it is called proper if $$\lim_{a\to \infty} \prob[X\geq a] = 0$$
	\end{definition}

	\begin{definition}[Improper r.v]
		given X a r.v. it is called improper if  $$\lim_{a\to \infty} \prob[X\geq a] = p_\infty \ne 0$$
	\end{definition}

	\begin{theorem}
		$$i  \text{ is recurrent} \iff \sum\limits_{n=1}^\infty P_{ii}^{(n)} = \infty$$
	\end{theorem}
	---

	\begin{proof}
		Let the number of time state i has been visited be $$M = \sum\limits_{n=1}^\infty \mathds{1}\{X_n =i\} $$
		The following swap from expected value and infinite sum is allowed by Fubini's Theorem
		$$
			\exp[M | X_0 = i] = \exp\left[\sum\limits_{n=1}^\infty \mathds{1}\{X_n = i\} | X_0 = i\right]
			= \sum\limits_{n=1}^\infty \exp\left[\mathds{1}\{ X_n = i\} | X_0 = i\right] = \sum\limits_{n=1}^\infty P_{ii}^{(n)}
		$$
		So we have shown that $\exp[M | X_0=i] = \sum\limits_{n=1}^\infty P_{ii}^{(n)}$.\\
		Now, remembering that $$ \exp[M | X_0 = i] = \begin{cases}
		\frac{f_{ii}}{1-f_{ii}}, & \text{if transient} \\
		\infty, & \text{if recurrent}
		\end{cases}$$
		the proof is concluded.
	\end{proof}

	\begin{theorem}
		(not to confuse with theorem on periodicity)
		$$\text{if } i\leftrightarrow j \text{and i is recurrent} \Rightarrow \text{ j is also recurrent}$$
	\end{theorem}
	---
	\begin{proof}
		$$\text{Since } i\leftrightarrow j : \exists m,n \text{ such that } P_{ij}^{(n)} \text{ and } P_{ji}^{(m)} > 0$$
		$$\text{let } r>0 : P_{jj}^{(m+r+n)} = \sum\limits_{r, k} P_{jh}^{(m)} P_{hk}^{(r)} P_{kj}^{(n)} \geq P_{ji}^{(m)}  P_{ii}^{(r)}  P_{ij}^{(n)}$$
		$$\sum\limits_{l=1}^\infty P_{jj}^{(l)} \geq \sum\limits_{r=1}^\infty P_{jj}^{(m+r+n)} \geq \sum\limits_{r=1}^\infty P_{ji}^{(m)}  P_{ii}^{(r)}  P_{ij}^{(n)} = P_{ji}^{(m)} P_{ij}^{(n)} \sum\limits_{r=1}^\infty P_{ii}^{(r)}$$
		but since \begin{itemize}
		\item$i$ is recurrent $\Rightarrow \sum\limits_{n=1}^\infty P_{ii}^{(n)} = \infty$
		\item $P_{ij}^{(n)} \text{ and } P_{ji}^{(m)} > 0$
		\end{itemize}
		$$\sum\limits_{l=1}^\infty P_{jj}^{(l)} = \infty \Rightarrow j \text{ is recurrent}$$
	\end{proof}

	\begin{theorem}[Basic limit theorem on MC]
		Consider an irreducible aperiodic recurrent MC (an aperiodic recurrent class), we have
		$$ \lim_{n\to \infty} P_{ii}^{(n)} = \frac{1}{m_i} = \pi_i = \lim_{n\to\infty} P_{ji}^{(n)} \qquad \forall j$$
	\end{theorem}

	---

	Following table summarize some properties for the types of state that can be found in a \gls{mc}.

	{\renewcommand{\arraystretch}{1.2}
	\begin{center}
		\begin{tabular}{|c||c|c|c|}
			\hline
			State $i$ & Transient & Null recurrent & Positive recurrent \\ \hline
			$f_{ii} = \sum\limits_{n=1}^\infty f_{ii}^{(n)}$ & $<1$ & 1 & 1 \\ \hline
			$\lim\limits_{k \to \infty } \prob[M \geq k | X_0=i]$ & 0 & 1 & 1 \\ \hline
			$\exp[n|X_0=i]$ & $\frac{f_{ii}}{1-f_{ii}}$ & $\infty$ & $\infty$ \\ \hline
			$m_i = \sum\limits_n f_{ii}^{(n)}$ & $\infty$ & $\infty$ & $<\infty$ \\ \hline
			$\pi_i = \frac{1}{m_i}$ & 0 & 0 & $>1$ \\ \hline
		\end{tabular}
	\end{center}}

	\begin{theorem}
		For an aperiodic positive recurrent class (irreducible MC), $\pi_j$ is the unique solution of the following system.
		\begin{equation*}\begin{cases}
			\pi_j = \sum\limits_{i=0}^\infty \pi_i P_{ij} \\
			\sum\limits_{i=0}^\infty \pi_i = 1 \\
			\pi_i \geq 0 \quad \forall i
		\end{cases}\end{equation*}
	\end{theorem}

	\begin{proof} The funny thing is that this proof is easier than the book:
		\begin{enumerate}
			\item
			we first want to show that the $\pi_j$ satisfy the system (\textbf{\textit{Existence of the solution}})
			\begin{equation*}
				\begin{split}
					\forall m,n \qquad 1&=\sum\limits_{j=0}^\infty P_{ij}^{(n)} > \sum\limits_{j=0}^m P_{ij}^{(n)}\\
	 			 \lim_{n\to\infty} \sum\limits_{j=0}^m P_{ij}^{(n)} &= \sum\limits_{j=0}^m \pi_j \leq 1 \quad \forall n \\
				 &\implies \sum\limits_{j=0}^\infty \pi_j \leq 1\\
				\end{split}
			\end{equation*}

			\item
			\begin{equation*}
				\begin{split}
					P_{ij}^{(n+m)} &\geq \sum\limits_{k=0}^m P_{ik}^{(m)} P_{kj}^{(n)} \quad \forall n, m, M\\
					\text{as } m \to \infty :\quad \pi_j &\geq \sum\limits_{k=0}^m \pi_k P_{kj}^{(n)}\\
					&\implies  \pi_j \geq \sum\limits_{k=0}^m \pi_k P_{kj}^{(n)}
				\end{split}
			\end{equation*}

			\item
			\begin{equation*}
				\begin{split}
					\sum\limits_{k=0}^\infty \sum\limits_{j=0}^\infty \pi_k P_{kj}^{(n)} &\geq
					\sum\limits_{k=0}^m \sum\limits_{j=0}^\infty \pi_k P_{kj}^{(n)}  \\
					&=(\text{since } \sum\limits_{j=0}^\infty P_{kj}^{(n)} = 1 )\\
					&=\sum\limits_{k=0}^m \pi_k \quad \forall m\\
					\text{suppose } \exists j > 1 : \pi_j &> \sum\limits_{k=0}^\infty \pi_k P_{kj}^{(n)} \\
					\text{then we have that } \sum\limits_{j=0}^\infty \pi_k &> \sum\limits_{j=0}^\infty \sum\limits_{k=0}^\infty \pi_k P_{kj}^{(n)} > \sum\limits_{k=0}^\infty \pi_k \quad \text{ABSURD.} \\
					&\implies \pi_j = \sum\limits_{k=0}^\infty \pi_k P_{kj}^{(n)}
				\end{split}
			\end{equation*}

			\item
			\begin{equation*}
				\begin{split}
			 		\pi_j = \sum\limits_{k=0}^\infty \pi_k P_{kj}^{(n)}, \qquad |P_{kj}^{(n)}| &\leq 1 \quad \forall n,i,k \\
					\text{using an appropriate theorem for sliding the}&\text{ limit inside the infinite sum we have:}\\
					\pi_j = \sum\limits_{k=0}^\infty  \pi_k \lim_{n\to\infty} P_{kj}^{(n)} &= (\sum\limits_{k=0}^\infty \pi_k) \pi_j\\
					\text{and now since $\pi_j>0$}
					&\implies \sum\limits_{k=0}^\infty \pi_k = 1
				\end{split}
			\end{equation*}
			\textbf{\textit{This concludes the existence proof.}}

			\item
			Let's now prove the \textbf{\textit{Uniqueness}} \\
			Suppose $X_j$ is a solution
			\begin{equation}
				\begin{split}
					X_j =
					 \sum\limits_{i=0}^\infty X_i P_{ij} &=
					 \sum\limits_{i=0}^\infty ( \sum\limits_{k=0}^\infty X_k P_{ki} ) P_{ij} \geq
					 \sum\limits_{k=0}^m X_k \\
					 \sum\limits_{i=0}^\infty P_{ki} P_{ij} &= \sum\limits_{k=0}^m X_k P_{kj}^{(2)}
					 \quad \forall m
					 \Rightarrow X_j \geq \sum\limits_{k=0}^\infty X_k P_{kj}^{(n)}\qquad \forall n
				\end{split}
			\end{equation}

			This is the same result of $3^{rd}$ step, so as in $3^{rd}$ step, we can prove by contradiction that this inequality is in fact an equality.

			$$ \implies X_j = \sum\limits_{k=0}^\infty X_k P_{kj}^{(n)} \quad \forall n $$
			as $n \to \infty$ we have:

			\begin{equation}
				X_j = (\sum\limits_{k=0}^\infty X_k ) \pi_j \Rightarrow X_j = \pi_j
			\end{equation}

			So it's unique.
		\end{enumerate}
	\end{proof}

	\begin{lemma}
	  If $0 < p_i < 1 ~,~ i=0,1,2.\cdots $, then:
		\begin{equation}\label{limprodpi}
			\lim_{m \to \infty} \prod_{i=0}^{m}(1-p_i) = 0
		\end{equation}
		if and only if
		\begin{equation}\label{pitoinfty}
			\sum\limits_{i=0}^\infty p_i = \infty
		\end{equation}
	\end{lemma}

	\begin{proof}
		\begin{enumerate}
			\item Assume \eq{pitoinfty} is true. \\
				Since the series expansion for $e^{-p_i}$ is an alternating series with terms decreasing in absolute value, we can write:
				\begin{equation}
					1-p_i < 1-p_i + \frac{p_i^2}{2!} - \frac{p_i^3}{3!} + \cdots = ~e^{-p_i} \quad with ~i\ge 0
				\end{equation}
				applying the product to both members we obtain
				\begin{equation}
					\prod_{i=0}^{k-1} (1-p_i) < e^{-\sum\limits_{i=0}^{k-1}p_i}
				\end{equation}
				But by assumption $\sum\limits_{i=0}^\infty p_i = \infty$ hence,
				$$ \lim_{m \to \infty} \prod_{i=0}^{m}(1-p_i) = 0 $$

			\item Let's prove the following inequality:
			$$ \prod_{i=j}^m(1-p_i) > 1-\sum\limits_{i=j}^m p_i \quad \forall m \ge j+1$$
			We can prove it recursively:
			$$(1-p_j)(1-p_{j+1}) = 1-p_j - p_{j+1} + p_j p_{j+1} > 1-p_j - p_{j+1}$$
			Iterating we obtain:
			\begin{eqnarray*}
				\prod_{i=j}^{m+1}(1-p_i) = (1-p_{m+1})\prod_{i=j}^m(1-p_i) > (1-p_{m+1})(1-\sum\limits_{i=j}^m p_i) = \\
				1- \sum\limits_{i=j}^{m+1} p_i + p_{m+1}\sum\limits_{i=j}^m p_i
			\end{eqnarray*}

			Assume now that $\sum\limits_{i=1}^\infty p_i < \infty$, then there must exist an index $j>1$ s.t. $\sum\limits_{i=j}^\infty p_i < 1$.

			Then we have:
			$$ \lim_{m \to \infty} \prod_{i=j}^m (1-p_i) \ge \lim_{m \to \infty} (1-\sum\limits_{i=j}^m p_i) > 0 $$
		\end{enumerate}
	\end{proof}

	\begin{definition}[lesson 22/03/17] \label{def:falling_probability}

		Given a recurrent class $C$ and atransient state $i \notin C$, the probability of entering in that class through state $k \in C$ at step $n$ can be written as
		$$ \Pi_{ik}^{n}(C) = \prob[X_n = k \in C, x_l \notin C ~ \forall l=1, ..., n-1 | X_0 = i] $$

		Thus, the probability that the chain, starting from $i$, reaches class $C$ at step $n$ is
		$$ \pi_{i}^{n}(C) = \sum_{k \in C} \Pi_{ik}^{n}(C) $$

		and the general probability of reaching $C$ starting from $i$ is
		$$ \pi_i(C) = \sum_{n \in \mathbb{N}} \pi_i^{n}(C) $$
	\end{definition}

	\begin{theorem}[3.1, KT p. 91] \label{th:3.1}
		Given state $j \in C$, an aperiodic and recurrent class, and a transient state $i \notin C$, it holds that

		$$ \lim_{n \to \infty } P_{ij}^{(n)} = \pi_i(C) \cdot \lim_{n \to \infty } P_{jj}^{(n)} = \pi_i(C) \cdot \pi_j $$
	\end{theorem}
	---
	\begin{proof}
		The limit of theorem thesis can be expanded this way
		\begin{equation}\begin{split} \label{eq:theorem_3.1_thesis}
			& \forall \epsilon > 0, \exists \text{ class } C' \in C, N \in \mathbb{N} \text{ such that } \\
			& \forall n \ge N,~ \left| P_{ij}^{(n)} - \pi_i(C) \pi_j \right| \stackrel{(1)}{=} \\
			= & \left| P_{ij}^{(n)} - \left( \sum_{v = 1}^{N} \sum_{k \in C'} \pi_{ik}^{(v)}(C) \right) \pi_j +
				\left( \sum_{v = 1}^{N} \sum_{k \in C'} \pi_{ik}^{(v)}(C)\right)\pi_j - \pi_i(C)\pi_j \right| \stackrel{(2)}{\le} \\
			\le & \left| P_{ij}^{(n)} - \left( \sum_{v = 1}^{N} \sum_{k \in C'} \pi_{ik}^{(v)}(C) \right) \pi_j \right| +
				\left| \left( \sum_{v = 1}^{N} \sum_{k \in C'} \pi_{ik}^{(v)}(C)\right) - \pi_i(C) \right| \pi_j
		\end{split}\end{equation}
		where
		\begin{itemize}
			\item [(1)] term between parenthesis is added and subtracted
			\item [(2)] absolute value of a sum is less or equal than the sum of the absolute values
		\end{itemize}

		If we can prove that the two terms are infinitesimal (i.e. $< \epsilon$) for given $N$ and $C'$, we are done.
		\smallbreak

		First we can prove that, given class $C$ is recurrent, transition probability in $n$ steps from $i$ to $j$ can be formulated as follows.
		\begin{equation} \label{eq:n_step_in_class}
			P_{ij}^{(n)} = \sum_{v = 1}^{n} \sum_{k \in C} \pi_{ik}^{(v)}(C) ~ P_{kj}^{(n-v)}
		\end{equation}
		where path from $i$ to $j$ is splitted in two parts, before and after entering $C$.

		This formula can be written as a limit in $n$ and $C$, expliciting the inner infinite sum over class elements.
		\begin{equation}\begin{split} \label{eq:theorem_3.1_first term}
			& \forall \epsilon > 0, \exists N \in \mathbb{N} \text{ and a finite class } C' \subseteq C \text{ such that } \\
			& \forall n \ge N, \left| P_{ij}^{(n)} - \pi_j \left( \sum_{v = 1}^{n} \sum_{k \in C'} \pi_{ik}^{(v)} \right) \right| \stackrel{(1)}{=}
			\\
			= & \left|
				\left(
					\sum_{v = 1}^{n} \sum_{k \in C'} \pi_{ik}^{(v)}(C) ~ P_{kj}^{(n-v)} + \sum_{v = 1}^{n} \sum_{\substack{k \in C \\ k \notin C'}} \pi_{ik}^{(v)}(C) ~ P_{kj}^{(n-v)} \right)
				- \pi_j \left( \sum_{v = 1}^{n} \sum_{k \in C'} \pi_{ik}^{(v)}(C) \right)
				\right| \stackrel{(2)}{=}
			\\
			= & \left| \sum_{v = 1}^{n} \sum_{k \in C'} \pi_{ik}^{(v)}(C) (P_{kj}^{(n-v)} - \pi_j)
				+ \sum_{v = 1}^{n} \sum_{\substack{k \in C \\ k \notin C'}} \pi_{ik}^{(v)}(C) P_{kj}^{(n-v)} \right| \stackrel{(3)}{=}
			\\
			= & \left| \sum_{v = 1}^{N} \sum_{k \in C'} \pi_{ik}^{(v)}(C) (P_{kj}^{(n-v)} - \pi_j) +
				\sum_{v = N+1}^{n} \sum_{k \in C'} \pi_{ik}^{(v)}(C) (P_{kj}^{(n-v)} - \pi_j) +
				\sum_{v = 1}^{n} \sum_{\substack{k \in C \\ k \notin C'}} \pi_{ik}^{(v)}(C) P_{kj}^{(n-v)} \right| \stackrel{(4)}{\le}
			\\
			\le & \left| \sum_{v = 1}^{N} \sum_{k \in C'} \pi_{ik}^{(v)}(C) (P_{kj}^{(n-v)} - \pi_j) \right| +
				\left| \sum_{v = N+1}^{n} \sum_{k \in C'} \pi_{ik}^{(v)}(C) (P_{kj}^{(n-v)} - \pi_j) \right| +
				\left| \sum_{v = 1}^{n} \sum_{\substack{k \in C \\ k \notin C'}} \pi_{ik}^{(v)}(C) P_{kj}^{(n-v)} \right| \stackrel{(5)}{\le}
			\\
			\le & \left| \sum_{v = 1}^{N} \sum_{k \in C'} \pi_{ik}^{(v)}(C) (P_{kj}^{(n-v)} - \pi_j) \right| +
				2 \left( \sum_{v = N+1}^{n} \sum_{k \in C'} \pi_{ik}^{(v)}(C) \right) +
				\left( \sum_{v = 1}^{n} \sum_{\substack{k \in C \\ k \notin C'}} \pi_{ik}^{(v)}(C) \right) \stackrel{(6)}{\le}
			\\
			\le & \left| \sum_{v = 1}^{N} \sum_{k \in C'} \pi_{ik}^{(v)} (P_{kj}^{(n-v)} - \pi_j) \right| + 3 \epsilon \stackrel{(7)}{<} 4 \epsilon
		\end{split}\end{equation}since $\pi_j = \lim_{n \to +\infty} P_{kj}^{(n-v)}$ by definition, what we have here is a finite sum (over $N$ and $C'$) of infinitesimal quantities
		where
		\begin{itemize}
			\item [(1)] using equation \ref{eq:n_step_in_class}, sum is splitted in $k \in C$ into two sums in $k \in C'$ and $k \in C ~ \wedge ~ k \notin C' $
			\item [(2)] terms have been rearranged: first and last one have been merged
			\item [(3)] the first sum over all $n \in \mathbb{N}$ is splitted using $N$
			\item [(4)] the module of a sum is less or equal than the sum of the modules
			\item [(5)] given it always holds that $P_{kj}^{(n-v)} \le 1$ and $|P_{kj}^{(n-v)} - \pi_j| \le 2$
			\item [(6)] choosing $C'$ and $N$ large enough, we can make the two terms between parenthesis small as we want: they are infinitesimal
			\item [(7)] since $\pi_j = \lim_{n \to +\infty} P_{kj}^{(n-v)}$ by definition, what we have here is a finite sum (over $N$ and $C'$) of infinitesimal quantities
		\end{itemize}
		This way we have verified that first term of \ref{eq:theorem_3.1_thesis} is in fact infinitesimal.

		\bigbreak
		Now we explore the second term.
		Given definitions \ref{def:falling_probability}, it holds that
		$$ \pi_i(C) = \sum_{v = 1}^{+\infty} \sum_{k \in C} \pi_{ik}^{(v)}(C) $$

		\smallbreak
		Since the right term converges to a finite value, namely $\pi_i$, the limit implicit in the infinite sum can be expanded this way.
		\begin{equation}\begin{split} \label{eq:pi_limit_definition}
			& \forall \epsilon > 0, \exists N \in \mathbb{N} \text{ and a finite class } C' \subseteq C \text{ such that } \\
			& \forall n \ge N,~ \left| \pi_i(C) - \sum_{v = 1}^{n} \sum_{k \in C'} \pi_{ik}^{(v)}(C) \right| < \epsilon
		\end{split}\end{equation}

		\bigbreak
		Recalling thesis (equation \ref{eq:theorem_3.1_thesis}), both terms of the sum have been proven infinitesimal, so wanted limit is itself proven.
	\end{proof}

	Summarizing all we know about limiting distribution across multiple classes, we can build the following table.
	See theorem \ref{th:3.1} for last line.
	\begin{center}\begin{tabular}{c|c|c}
		Starting state $i$ & Arrival state $j$ & $\lim_{n \to +\infty} P_{ij}^{(n)}$ \\ \hline
		any & transient & 0 \\
		recurrent $\notin C_1 ,\: \in C$ & recurrent $\in C$ & 0 \\
		recurrent $\in C$ & recurrent $\in C$ & $\pi_j = 1 / m_j$ \\
		transient & recurrent $\in C$ & $\pi_i(C)\cdot\pi_j = \pi_i(C) / m_j$ \\
	\end{tabular}\end{center}

	\begin{theorem}[Property of finite \gls{mc}] \label{th:finite_MC_1}
		A finite \gls{mc} must have a positive recurrent state.

		$$ \forall \text{ finite \gls{mc}, } \exists i: \pi_i \neq 0 $$
	\end{theorem}
	---
	\begin{proof}
		Labeling \gls{mc} states from $1$ to $N$, we can always write that

		$$ \forall i, n ~ \sum_{j=0}^N P_{ij}^{(n)} = 1$$

		This holds for each value of $n$, so it must be true also in the limit.
		Suppose also that there are no positive reccurent states:
		$$ 1 = \lim_{n \to +\infty} \sum_{j=0}^N P_{ij}^{(n)} \stackrel{(*)}{=} \sum_{j=0}^N \lim_{n \to +\infty} P_{ij}^{(n)}
		= \sum_{j=0}^N \pi_j $$
		where passage marked with the (*) is possible only because sum is finite.

		Since the sum of the $\pi_j$ is not zero, one of them must be strictly positive.
		Corresponding state is then positive recurrent, proving our theorem.
	\end{proof}

	\begin{theorem}[Property of finite \gls{mc}]
		A finite \gls{mc} can't have null recurrent state.

		$$ \text{ In a finite \gls{mc}, } \nexists~ i: \pi_i = 0 $$
	\end{theorem}
	---
	\begin{proof}
		Supposing such a null recurrent exists, there must be a null recurrent class that contains it.
		Such class is finite, given the chain is finite too.

		But for theorem \ref{th:finite_MC_1} such class must have at least one positive recurrent state, which is absurd.
		So a null recurrent class cannot exist in a finite \gls{mc}.
	\end{proof}

	\begin{lemma}[Ross 2, pg. 78-82] \label{lemma:MC_irreducible_fi0}
		Given an \gls{mc} with a state set $S$ that contains state 0, it is irreducible if and only if

		$$ \forall i \neq 0, f_{i 0} = 1 $$
		where $f_{i 0}$ is the probability of reaching state $j$ starting from 0 at any time in the future.
	\end{lemma}
	---
	\begin{proof} \emph{if implication} $"\Leftarrow"$
		\begin{equation}\begin{split}
			f_{00} = P_{00} + \sum_{i \neq 0} P_{0i} f_{i 0} \stackrel{(*)}{=} \sum_{i} P_{0i} = 1
		\end{split}\end{equation}
		where (*) is due to hypothesis.

		This implies that, starting from any state $j$, the chain returns to zero eventually with probability 1.
	\end{proof}

	\begin{proof} \emph{only if implication} $"\Rightarrow"$
		Suppose chain is irreducible, but thesis is false, i.e.
		$$ \exists i \neq 0 : f_{i0} < 1 $$.

		If \gls{mc} is irreducible (i.e. it is made only of a single class) it must be that
		$$ \forall i \neq 0, \exists m : P_{0i}^{(m)} > 0 $$

		Let $n$ be the minimum value among all possible $m$  that satisfies this condition for given state $i \neq 0$: such number is the shortest path from 0 to $i$ and, by its definition, does not cross state 0 in the middle.

		$$ \prob[\forall n, X_n \neq 0 | X_0 = 0] = 1 - f_{00} \stackrel{(*)}{\ge} P_{0i}^{(n)} (1 - f_{i0}) > 0 \Rightarrow \text{ state 0 is transient} $$

		where (*) inequality holds because leaving 0 forever (first term) is a more general case of going to state $i$ via the shortest path and then not crossing 0 ever again.

		This is absurd, because chain is supposed irreducible.
	\end{proof}

	\begin{definition}
		Given an \gls{mc} with a state set $S = \{1, 2, ...\}$, quantity $Y_i(n)$ is defined as the probability of staying in $S$ along an $n$-steps path that starts from $i \in S$.

		$$ Y_i(n) = \prob[X_j \in S ~\forall j=1, 2, ..., n | X_0 = i \in S] $$
	\end{definition}

	\begin{theorem}
		$Y_i(n)$ is monothonically non-increasing on $n$.
		$$ Y_i(n) \le Y_i(n-1) $$
	\end{theorem}
	---
	\begin{proof} Base case $n=1$
		\begin{equation}\begin{split} \label{eq:Y_i-properties}
			& Y_i(1) = \prob[X_1 \in S | X_0 = i] = \sum_{j \in S} P_{ij} \\
			& Y_i(2) \stackrel{(*)}{=} \sum_{j \in S} P_{ij} Y_i(1) \le \sum_{j \in S} P_{ij} = Y_i(1)
		\end{split}\end{equation}
		where (*) holds by \emph{first step analysis}: $ Y_i(n) = \sum_{j \in S} P_{ij} Y_j(n-1) $.
	\end{proof}

	\begin{proof} Inductive step
		\begin{equation}\begin{split}
			& Y_i(n+1) \stackrel{(1)}{=} \sum_{j \in S} P_{ij} Y_i(n) \stackrel{(2)}{\le} \sum_{j \in S} P_{ij} Y_i(n-1) \stackrel{(3)}{=} Y_i(n)
		\end{split}\end{equation}
		where (1) and (3) hold by first step analysis and (2) for inductive hypothesis.
	\end{proof}

	\begin{lemma}
		Since $Y_i(n)$ is monothonically non-increasing and positive (it is a probability), we can always take the limit for $n \to +\infty$.

		$$ \exists \lim_{n \to +\infty} Y_i(n) \stackrel{.}{=} Y_i = \prob[\text{staying in } S | \text{starting from } i \in S] $$
		where $ S = \{ 1, 2, 3, ...\} $ as before.

		Using \emph{first step analysis} as in previous equation \ref{eq:Y_i-properties} and taking the limit for $n \to +\infty $, we can build the following system.
		\begin{equation} \label{eq:Yj_system}
			\forall i \neq 0, Y_i = \sum_{j \in S} P_{ij} Y_j
		\end{equation}
	\end{lemma}

	\begin{lemma}
		Let $\{Z_i, i=1, 2, ...\}$ be a solution set for system \ref{eq:Yj_system}. We focus in particular on non-divergent solutions, imposing that $ \forall i, |Z_i| \le 1 $.

		It holds that
		$$ \forall n, |Z_i| \le Y_i(n) ~ \Rightarrow ~|Z_i| \le \lim_{n \to +\infty} Y_i(n) = Y_i $$
	\end{lemma}
	---
	\begin{proof} Base case $n=1$
		$$ |Z_i| = \sum_{j \in S} P_{ij} |Z_j| \stackrel{(*)}{\le} \sum_{j \in S} P_{ij} = Y_i(1) $$
		where (*) holds because $|Z_i| < 1$.
	\end{proof}

	\begin{proof} Inductive step
		$$ |Z_i| = \sum_{j \in S} P_{ij} |Z_j| \stackrel{(*)}{\le} \sum_{j \in S} P_{ij} Y_j(n) = Y_i(n+1) $$
		where (*) is due to the inductive hypothesis.
	\end{proof}

	\begin{theorem}
		An irreducible \gls{mc} with states $S = 0, 1, 2, ...$ is transient if and only if

		$$ Z_i = \sum_{j=1}^{+\infty} P_{ij} Z_j \text{ for } i = 1, 2, ...$$

		has a non-zero bounded solution with $ |Z_i| \le 1$.
	\end{theorem}
	---
	\begin{proof}
		Let's consider a \gls{mc} and a state subset $S = \{1, 2, 3, ...\}$.

		$$ \prob[\text{leaving } S| X_0 = i \neq 0] \stackrel{(1)}{=}
			1 - Y_i \stackrel{(2)}{=} f_{i0} $$
		where (1) holds for $Y_i$ definition and (2) because, for $S$ formulation, leaving $S$ means reaching 0.

		We can now distinguish two cases, based on $Z_i$ solutions:
		\begin{itemize}
			% Z has only at most zero solutons
			\item $ \forall i, Z_i \le 0 \Leftrightarrow \forall i, Y_i = 0 \Leftrightarrow \forall i, f_{i0} = 1 \stackrel{(*)}{\Leftrightarrow} $ chain is recurrent

			% Z has a non zero bounded solution
			\item $ \exists i: Z_i > 0 \Leftrightarrow \exists i: Y_i > 0 \Leftrightarrow \exists i: f_{i0} < 1 \stackrel{(*)}{\Leftrightarrow} $ chain is transient
		\end{itemize}
		where (*) implications hold for theorem \ref{lemma:MC_irreducible_fi0}.
	\end{proof}

	\begin{theorem}[4.2, KT p. 95]
		A \gls{mc} with probability matrix $P$ is irreducible and aperiodic if

		\begin{equation}\noindent%\begin{split}
			\exists \text{ a sequence } Y_i \text{, for } i = 1,\,... :\\
			\begin{cases}
 				\sum\limits_{j=0}^{+\infty} P_{ij} Y_j \le Y_i 	& \forall i \neq 0 \\
 				\lim\limits_{i \to +\infty} Y_i = +\infty		& \forall i
			\end{cases}
	% \end{split}
	\end{equation}
	\end{theorem}
	---
	\begin{proof}
		Proof is splitted into small chunks, as usual.

		\begin{enumerate}
		\item Let's consider a \gls{mc} with probability matrix $\tilde{P}$, that concides with P except for the first line, which is $(1, 0, 0, ...)$.
		This way we make the state 0 absorbing.

		We employ this trick to write theorem condition also for $i=0$.
		$$ \sum_{j=0}^{+\infty} \tilde{P}_{ij} Y_j \le Y_i ~~ \forall i$$

		\item If a given sequence $Y_i$ is a solution, its shifted version $y_i + b$ is a solution too, for all $b$.

		Since $Y_i$ sequence diverges, we can choose a suitable $b$ that makes each member strictly positive.
		\begin{equation}
			\exists b: \forall i, Y_i + b \doteq Y_i' > 0 \text{ is a valid solution.}
		\end{equation}

		From now on, we consider only positive sequences $Y_i$.

		\item \begin{equation}\begin{split}
			\forall i, ~ Y_i & \ge \sum_{j=0}^{+\infty} \tilde{P}_{ij} Y_j \stackrel{(1)}{\ge}
			\sum_{j=0}^{+\infty} \tilde{P}_{ij} \left( \sum_{k=0}^{+\infty} \tilde{P}_{ik} Y_k \right) \stackrel{(2)}{=} \\
			& = \sum_{k=0}^{+\infty} Y_k \sum_{j=0}^{+\infty} \tilde{P}_{ij}  \tilde{P}_{ik} =
			\sum_{k=0}^{+\infty} Y_k \tilde{P}_{ij}^{(2)} = \\
			& \stackrel{(3)}{=} \ldots = \sum_{k=0}^{+\infty} Y_k \tilde{P}_{ij}^{(m)} \forall m
		\end{split}\end{equation}
		where
		\begin{itemize}
			\item [(1)] hypothesis has been applied to $Y_j$ itself. Note that for $\tilde{P}$ this property is valid for $i=0$ too.
			\item [(2)] sums can be swapped since sum in $k$ surely converges, both simply and absolutely since $Y_i$ are strictly positive. Note that convergence is guaranteed by the fact that at the left side of the current chain of decreasing inequalities we find the real number $Y_i$, that exists for hypothesis.
			\item [(3)] the same reasoning of (2) is repeated in a recursive way.
		\end{itemize}

		\item
		Since $Y_i$ serie diverges, we can write, by limit definition
		$$ \forall \epsilon > 0, \exists M \in \mathbb{N} : Y_i \ge 1 / \epsilon ~ \forall i \ge M$$

		\begin{equation}\begin{split}
			\forall i, \forall m, ~ Y_i & \ge \sum_{k=0}^{+\infty} Y_k \tilde{P}_{ij}^{(m)}
				= \sum_{k=0}^{M-1} Y_k \tilde{P}_{ij}^{(m)}
				+ \sum_{k=M}^{+\infty} Y_k \tilde{P}_{ij}^{(m)} \ge \\
			& \ge \sum_{k=0}^{M-1} Y_k \tilde{P}_{ij}^{(m)}
				+ \frac{1}{\epsilon} \sum_{k=M}^{+\infty} \tilde{P}_{ij}^{(m)} \\
			& = \sum_{k=0}^{M-1} Y_k \tilde{P}_{ij}^{(m)}
				+ \frac{1}{\epsilon} \left( 1 - \sum_{k=0}^{M-1} Y_k \tilde{P}_{ij}^{(m)} \right) \\
		\end{split}\end{equation}

		\item
		Our aim is now to prove following statement
		$$ \forall j \neq 0, \lim_{m \to \infty} \tilde{P}_{ij}^{(m)} = 0 $$

		We can distinguish two cases,
		\begin{itemize}
			\item if $P$ is recurrent, we certantly reach state 0 at some time ($f_{i0} = 1$), but given $\tilde{P}$ strucure we will be stuck there forever.
			\item if $P$ is transient, it must be that $\lim_{m \to \infty} P_{ij}^{(m)} = 0 $, so our lemma would be proven if
			$$ \forall i, j \neq 0, ~ \tilde{P}_{ij}^{(m)} \le P_{ij}^{(m)} $$
		\end{itemize}

		This possibility can be more widely explained this way, splitting $\tilde{P}_{ij}^{(m)}$ and $P_{ij}^{(m)}$ across random walks	 that contain state 0 or not.
		\begin{equation}
			\left\{ \begin{split}
				& P_{ij}^{(m)} = \sum_{l=1}^{m-1} f_{i0}(l) P_{0j}^{(m-l)} +
					& \prob[X_m = j, X_l \neq 0 ~ \forall l = 1, ..., m-1 | X_0 = i] \\
				& \tilde{P}_{ij}^{(m)} =
					& \prob[X_m = j, X_l \neq 0 ~ \forall l = 1, ..., m-1 | X_0 = i] \\
			\end{split} \right.
		\end{equation}
		Note that, in $\tilde{P}$, state $0$ is a trap state, so the first term of the sum disappear because $j \neq 0$ can no more be reached.

		\item
		Now we assume that chain is transient? and we proceed to an absurd statement, which proves our theorem.

		By definition,
		$$ f_{i0} = \lim_{m \to +\infty} \tilde{P}_{i0}^{(m)} = \prob[\exists k : X_k = 0 | X_0 = i] = \tilde{\pi}_i(C_0)$$
		where $C_0$ is the class that contains only state 0.

		For what has been proved previously,
		\begin{equation}\begin{split}
			Y_i & = \lim_{m \to +\infty} Y_i = \\
				& = \lim_{m \to +\infty} \left[ \sum_{k=0}^{M-1} Y_k \tilde{P}_{ij}^{(m)}
					+ \frac{1}{\epsilon} \left( 1 - \sum_{k=0}^{M-1} Y_k \tilde{P}_{ij}^{(m)} \right) \right] \stackrel{(*)}{=} \\
			& = \tilde{\pi}_i(C_0) Y_0 + \frac{1}{\epsilon} \left( 1 - \tilde{\pi}_i(C_0) \right) ~ \forall \epsilon > 0
		\end{split}\end{equation}
		where (*) holds because 0 is a trap state for $\tilde{P}$, so for $m \to +\infty$ only paths ending in that state have a non-zero probability of being taken.

		Terms can be rearranged this way

		\begin{equation*}\begin{split}
			\forall \epsilon > 0, ~ & Y_i - \tilde{\pi}_i(C_0) Y_0 = \frac{1}{\epsilon} \left( 1 - \tilde{\pi}_i(C_0) \right) \\
			& \epsilon \, [Y_i - \tilde{\pi}_i(C_0) Y_0] = 1 - \tilde{\pi}_i(C_0)
		\end{split}\end{equation*}

		Since previous relation holds $\forall \epsilon > 0$ and all terms are numbers (no functions), we can freely take the limit for $\epsilon \to 0$.
		$$ 1 - \tilde{\pi}_i(C_0) = \lim_{\epsilon \to 0^{+}} \epsilon \, [Y_i - \tilde{\pi}_i(C_0) Y_0] = 0 \Rightarrow \tilde{\pi}_i(C_0) = f_{i0} = 1$$
		which implies that \gls{mc} is recurrent for lemma \ref{lemma:MC_irreducible_fi0}.
		\end{enumerate}
	\end{proof}
