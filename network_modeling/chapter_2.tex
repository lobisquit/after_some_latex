\chapter{Markov Chains}

\section{Introduction}
A markov process is a process $X_n \text{ or } X_t$ s.t.
\begin{equation}
\prob[X_{n+1} = j | X_0 = i_0, X_1 = i_1, \dots, X_n = i_n] = \prob[X_{n+1} = j | X_n=i_n] = P
\end{equation}
with P as the \textit{one-step probability}. From now on, the notation for that probability
will be $P_{i,j}^{n,n+1}=P_{i,j} \quad \forall n$ if the MC is homogeneous, i.e.
the transition probabilities are stationary.\\
We can suppose $n\ge 0$ and we'll have the \textbf{transition probability matrix}, which will characterize the
process as
\begin{equation} P=\begin{pmatrix}
P_{00} & P_{01} & \cdots  \\
P_{10} & P_{11} & \cdots \\
\vdots & \vdots & \ddots  \\
 \end{pmatrix}
\text{with } P_{i,j}\ge 0 \forall i,j \quad \text{ and } \sum\limits_{j=0}^{+\infty}P_{i,j} = 1 \quad \forall i
\end{equation}
For example
\begin{equation}
\prob[X_3 = i_3 , X_2 = i_2, X_1 = i_1, X_0 = i_0]=P_{i_0}\cdot P_{i_0,i_1}\cdot P_{i_1,i_2}\cdot P_{i_2,i_3}
\end{equation}

The description of a markov process is given from the initial state probability and the transition probability matrix.
% \begin{equation} \begin{split}
% P_{i,j}^{(n)} &= \prob[X_{m+n}=j | X_m=i] \forall m\\
% &=\prob[X_n=j|X_0=i]\stackrel{Tot.Prob.Th}{=}\sum\limits_{k=0}^{+\infty}\prob[X_n=j,X_1=k|X_0=i]\\
% &=\sum\limits_{k=0}^{+\infty}\prob[X_n=j|X_1=k]\cdot \prob[X_1=k|X_0=i]=\sum\limits_{k=0}^{+\infty}P_{i,k}\cdot P_{k,j}^{(n-1)} \\
% &\implies P_{i,j}^{(n)}=\sum\limits_{k=0}^{+\infty}P_{i,k}\cdot P_{k,j}^{(n-1)}\\
% &\text{and doing it for all $i_s$ and $j_s$ we obtain}\\
% &\implies P^{(n)} = P \cdot P^{(n-1)} = P \cdot P\cdot P^{(n-1)} = \dots = P^n \\
% \end{split} \end{equation}

\section{Stability of a process}
Suppose a system where customers arrive independently and are queued in the system before getting served.
Suppose the service time occupies only one slot and all the slots for service time have same duration.
If the queue is empty, that time slot is wasted.
Now let the number of users arriving in the n-th slot be $\xi_n$ and the number of users in
the system in the n-th slot  be $X_n$. Then we have:
 \begin{equation}
 	\begin{split}
 	&\prob[\xi_n=k] = a_k \text{, probability of k arriving users at time n}\\
  	&X_{n+1}=
		\begin{cases}
 			X_n -1 +\xi_n &  \text{if } X_n >0 \\
 			\xi_n 			 	&  \text{if } X_n = 0
 		\end{cases}
 	\end{split}
 \end{equation}

The transition probability matrix:
\begin{equation} P=\begin{pmatrix}
	a_{0} & a_{1} & a_2 & \cdots  \\
	a_{0} & a_{1} & a_2 & \cdots  \\
	0			& a_{0} & a_{1} & \cdots  \\
	0 		& 0			& a_{0} & \cdots  \\
	\vdots & \vdots & \ddots &  \\
\end{pmatrix}
\end{equation}
\begin{definition}[Behavior of a MC]
Let $\rho=\sum\limits_{k=0}^{+\infty} k \cdot a_k$. We say that for
\begin{equation} \rho : \begin{cases}
	>1 & \text{\textbf{unstable }: the system will never be able to serve everybody}\\
	=1 & \text{\textbf{unstable \textit{unless deterministic}}: sooner or later the system will fail}\\
	<1 & \text{\textbf{stable}: the queue tends to be empty}
\end{cases}\end{equation}
\end{definition}
% \section{Poisson processes}
\section{Long Run Behaviour}

\subsection{Steady-state probabilities}

	\begin{definition}[Regular \gls{mc}]
		A regular \gls{mc} is a \gls{mc} with the following property:

		\begin{equation} \lim_{n \to \infty} P_{ij}^{(n)} = \lim_{ n \to \infty} Prob[ X_n=j | X_0 =i] = \pi_j > 0 \quad \forall i, j \end{equation}

	\end{definition}

	This tell us that:
	\begin{enumerate}
		\item Limit \textbf{exists} (not obvious)
		\item Limit is indipendent of the initial state
		\item Limit is \textbf{strictly} positive
	\end{enumerate}

	\begin{theorem}
		For a regular \gls{mc} with states 0, 1, ..., N the limit distribution $\bm\pi = (\pi_0,\pi_1,\cdots,\pi_N)$ is the unique solution of the following system of equations:\\

		$\begin{cases}
			\pi_j = \sum\limits_{k=0}^N \pi_k P_{k j} & \text{for } j = 0,1, \cdots, N \\
			\sum\limits_{k=0}^N \pi_k = 1\\
			\\ \pi_k \ge 0 & \forall k
		\end{cases}$
	\end{theorem}

	\begin{proof} of \textbf{existence}:
		\begin{equation}
  			P_{i j}^{(n)} = \sum\limits_{k=0}^N P_{ik}^{(n-1)} P_{k j}
			\qquad with ~\sum\limits_{k=0}^N P_{ik}^{(n)} = 1 ~\forall n
		\end{equation}
		it's the developing of $\bm P^n = \bm P^{n-1} \bm P$

		Now let's study what happens as $ n \to \infty $:
		\begin{equation}
			\begin{split}
				&\pi_j = \lim_{n \to \infty} P_{ij}^{(n)} = \lim_{n \to \infty} \sum\limits_{k=0}^N P_{ik}^{(n-1)} P_{k j
				} =\\
				&= \sum\limits_{k=0}^N \lim_{n \to \infty} P_{ik}^{(n-1)} P_{k j
				} = \sum\limits_{k=0}^N \pi_k P_{kj}
			\end{split}
		\end{equation}
		Here the limit and the sum can be switched since the sum is finite.
		This shows that the system have a solution.
	\end{proof}

	\begin{proof} of \textbf{uniqueness}:
		Let $X_j$ be a solution, so, by construction, $X_j = \sum\limits_{k=0}^N X_k P_{kj} ~$.

		For a given state $l$, it holds that
		\begin{equation}
				X_l = \sum\limits_{j=0}^N X_j P_{jl} =  \sum\limits_{j=0}^N \left( \sum\limits_{k=0}^N X_k P_{kj} \right) P_{jl} =  \sum\limits_{k=0}^N X_k \sum\limits_{j=0}^N P_{kj} P_{jl} = \sum\limits_{k=0}^N X_k P_{kl}^{(2)}
		\end{equation}

		If we apply this trick again $n$ times we can prove by induction that:
		$X_j$ also satisfy $ X_j = \sum\limits_{k=0}^N X_k P_{kj}^{(n)} $ \quad  $\forall n $

		Now, as $n \to \infty$, we have that
		\begin{equation}
			X_j = \sum\limits_{k=0}^N X_k \pi_j = \pi_j (\sum\limits_{k=0}^N X_k) = \pi_j  \implies
			\text{The solution is unique}
		\end{equation}
	\end{proof}

\subsection{Classes of states}
	\begin{definition}[Accessible State]
		State $j$ is {\bfseries accessible} from state $i$ if $\exists n \geq 0$ such that $P_{ij}^{(n)} > 0$. It can be written ($i \rightarrow j$).
		State it's \textbf{not} accessible if $\forall n \ge 0 \quad P_{ij}^{(n)}=0$
	\end{definition}

	\begin{definition}[Communicant States]
		States $i$ and $j$ are said to {\bfseries communicate}, written ( $ i \leftrightarrow j$ ), if
		% \begin{equation}
			$\begin{cases}
				i \rightarrow j \\
				j \rightarrow i
			\end{cases}$
		% \end{equation}
	\end{definition}

	{\bfseries Proprieties}
	\begin{enumerate}
		\item \textbf{Reflexivity}: \quad $i \leftrightarrow i \quad\text{ and } P_{ii}^{(0)}=1$
		\item \textbf{Symmetry}: \quad if $i \leftrightarrow j$ then $j \leftrightarrow i$
		\item \textbf{Transitivity}: \quad if $i \leftrightarrow j$ and $j \leftrightarrow k \Rightarrow i \leftrightarrow k$
	\end{enumerate}
	This implies that communication between states is an equivalence relation.

	\begin{definition}[Periodicity]
		The period of state i, written $d(i)$, is the GCD (greatest common denominator) of set $S_i = \{ s>0 : P_{ii}^{(s)} >0 \}$.\\
		If $d(i)=1$, the state is said to be \textbf{aperiodic}
	\end{definition}

	\begin{theorem}[Periodicity] Periodicity is a characteristic of groups (called \emph{classes}) of communicating states.
		$$\text{if } i \leftrightarrow j \text{, then } d(i) = d(j)$$
	\end{theorem}

	\begin{proof}
		Given $S_i = \{ s>0 : P_{ii}^{(s)} >0 \}$, and let $n, m >0 : P_{ij}^{(m)} >0, P_{ji}^{(n)} > 0$, we have that $$\forall s \in S_i : P_{ii}^{(s)} > 0$$.
		Now, for total probability theorem, it holds that
		$$P_{jj}^{(n+s+m)} = \sum\limits_{h, k} P_{jh}^{(n)} P_{hk}^{(s)} P_{kj}^{(m)} \geq P_{ji}^{(n)} P_{ii}^{(s)} P_{ij}^{(m)} >0 $$
		$$P_{jj}^{(n+s+m)} >0 \implies n+s+m \in S_j$$
		$$P_{jj}^{(n+2s+m)} \geq P_{ji}^{(n)} P_{ii}^{(s)} P_{ii}^{(s)} P_{ij}^{(m)} >0 \Rightarrow n+2s+m \in S_j$$

		Now let's call $d(j) =$ g.c.d. of $S_j$.

		Since both $n+s+m$ and $n+2s+m$ are integer multiples of $d(j)$, so it is their difference $s$.

		We have proved that $\forall s \in S_i$ is integer multiple of $d(j)$: this implies that $d(j)$ is a common divisor of $S_i$, and moreover it divides the g.c.d. of $S$, $d(i)$. In conclusion, $d(i)$ is an integer multiple of $d(j)$.

		Doing this proof again switching role of $i$ and $j$ we prove that $d(j)$ is an integer multiple of $d(i)$, and so $d(i) = d(j)$.
	\end{proof}
	---

	\begin{definition}[Return Time]
		The random variable $R_i$ is defined as the time it takes to return to $i$ starting from $i$ itself.
	\end{definition}

	So the probability of ever going back to state $i$ can be written as
	$$ f_{ii} = \sum\limits_{n=1}^\infty f_{ii}(n)  = \sum\limits_{n=1}^\infty \prob[R_i=n | X_0=i] $$
	where $f_{ii}(n)$ is the probability of returning to state $i$ in $n$ steps.

	\begin{definition}[Recurrent state]
		a state is recurrent if $f_{ii} = 1$
	\end{definition}
	In other words we say that a state $i$ is recurrent if and only if, after the process starts from state $i$, the probability of its returning to state i after some finite length of time is one.

	\begin{definition}[Transient state]
		a state is transient if  $f_{ii} < 1$
	\end{definition}

	In other words a state $i$ is said to be transient if there exists a non-zero probability of never coming back to it.\footnote{like Angela. Please come back Angela, I love you.}

	\begin{definition}
		$M$ is the number of returns to state $i$, starting from $i$
		$$\exp[M | X_0 = i] = \sum\limits_{k=1}^\infty \prob[M\geq k | X_0 = i] = \sum\limits_{k=1}^\infty f_{ii}^k = \begin{cases}
		\frac{f_{ii}}{1-f_{ii}}, & \text{if transient} \\
		\infty, & \text{if recurrent}
		\end{cases}$$
	\end{definition}

	\begin{definition}[Proper r.v]
		given X a finite value r.v. it is called proper if $$\lim_{x\to \infty} \prob[x\geq a] = 0$$
	\end{definition}

	\begin{definition}[Improper r.v]
		given X a r.v. it is called improper if  $$\lim_{x\to \infty} \prob[x\geq a] = p_\infty \ne 0$$
	\end{definition}

	\begin{theorem}
		$$i  \text{ is recurrent} \iff \sum\limits_{n=1}^\infty P_{ii}^{(n)} = \infty$$
	\end{theorem}
	---

	\begin{proof}
		Let the number of time state i has been visited be $$M = \sum\limits_{n=1}^\infty \mathds{1}\{X_n =i\} $$
		The following swap from expected value and infinite sum is allowed by Fubini's Theorem
    $$ \exp[M | X_0 = i] = \exp\left[\sum\limits_{n=1}^\infty \mathds{1}\{X_n = i\} | X_0 = i\right]
    = \sum\limits_{n=1}^\infty \exp\left[\mathds{1}\{ X_n = i\} | X_0 = i\right] = \sum\limits_{n=1}^\infty P_{ii}^{(n)} $$
		So we have shown that $\exp[M | X_0=i] = \sum\limits_{n=1}^\infty P_{ii}^{(n)}$.\\
		Now, remembering that$$ \exp[M | X_0 = i] = \begin{cases}
		\frac{f_{ii}}{1-f_{ii}}, & \text{if transient} \\
		\infty, & \text{if recurrent}
		\end{cases}$$
		the proof is concluded.
	\end{proof}

	\begin{theorem}
		(not to confuse with theorem on periodicity)
		$$\text{if } i\leftrightarrow j \text{and i is recurrent} \Rightarrow \text{ j is also recurrent}$$
	\end{theorem}
	---
	\begin{proof}
		$$\text{Since } i\leftrightarrow j : \exists m,n \text{ such that } P_{ij}^{(n)} \text{ and } P_{ji}^{(m)} > 0$$
		$$\text{let } r>0 : P_{jj}^{(m+r+n)} = \sum\limits_{r, k} P_{jh}^{(m)} P_{hk}^{(r)} P_{kj}^{(n)} \geq P_{ji}^{(m)}  P_{ii}^{(r)}  P_{ij}^{(n)}$$
		$$\sum\limits_{l=1}^\infty P_{jj}^{(l)} \geq \sum\limits_{r=1}^\infty P_{jj}^{(m+r+n)} \geq \sum\limits_{r=1}^\infty P_{ji}^{(m)}  P_{ii}^{(r)}  P_{ij}^{(n)} = P_{ji}^{(m)} P_{ij}^{(n)} \sum\limits_{r=1}^\infty P_{ii}^{(r)}$$
		but since \begin{itemize}
		\item$i$ is recurrent $\Rightarrow \sum\limits_{n=1}^\infty P_{ii}^{(n)} = \infty$
		\item $P_{ij}^{(n)} \text{ and } P_{ji}^{(m)} > 0$
		\end{itemize}
		$$\sum\limits_{l=1}^\infty P_{jj}^{(l)} = \infty \Rightarrow j \text{ is recurrent}$$
  \end{proof}

	\begin{theorem}[Basic limit theorem on MC]
		Consider an irreducible aperiodic recurrent MC (an aperiodic recurrent class), we have
		$$ \lim_{n\to \infty} P_{ii}^{(n)} = \frac{1}{m_i} = \pi_i = \lim_{n\to\infty} P_{ji}^{(n)} \qquad \forall j$$
	\end{theorem}

	%DA SISTEMARE LA TABELLA
	\begin{center}
		\begin{tabular}{|l*{6}{|c}r}
			\hline
			State $i$ & $f_{ii} = \sum\limits_{n=1}^\infty f_{ii}^{(n)}$  & $\lim\limits_{k \to \infty } \prob[M \geq k | X_0=i]$ & $\exp[n|X_0=i]$ & $m_i = \sum\limits_n f_{ii}^{(n)}$ & $\pi_i = \frac{1}{m_i}$  \\ \hline

			Transient & $<1$ & 0 & $\frac{f_{ii}}{1-f_{ii}}$ & $\infty$ & 0 \\ \hline
			Null recurrent & 1 & 1 & $\infty$ & $\infty$ & 0 \\ \hline
			\begin{tabular}{@{}c@{}}Positive\\recurrent\end{tabular} & 1 & 1 & $\infty$ & $<\infty$ & $>1$ \\ \hline
		\end{tabular}
	\end{center}

	\begin{theorem}
		For an aperiodic positive recurrent class (irreducible MC), $\pi_j$ is the unique solution of
		$\begin{cases}
			\pi_j = \sum\limits_{i=0}^\infty \pi_i P_{ij} \\
			\sum\limits_{i=0}^\infty \pi_i = 1 \\
			\pi_i \geq 0 \quad \forall i
		\end{cases}$
	\end{theorem}

	\begin{proof} The funny thing is that this proof is easier than the book:
		\begin{enumerate}
			\item
			we first want to show that the $\pi_j$ satisfy the system (\textbf{\textit{Existence of the solution}})
			\begin{equation*}
				\begin{split}
					\forall m,n \qquad 1&=\sum\limits_{j=0}^\infty P_{ij}^{(n)} > \sum\limits_{j=0}^m P_{ij}^{(n)}\\
	 			 \lim_{n\to\infty} \sum\limits_{j=0}^m P_{ij}^{(n)} &= \sum\limits_{j=0}^m \pi_j \leq 1 \quad \forall n \\
				 &\implies \sum\limits_{j=0}^\infty \pi_j \leq 1\\
				\end{split}
			\end{equation*}

			\item
			\begin{equation*}
				\begin{split}
					P_{ij}^{(n+m)} &\geq \sum\limits_{k=0}^m P_{ik}^{(m)} P_{kj}^{(n)} \quad \forall n, m, M\\
					\text{as } m \to \infty :\quad \pi_j &\geq \sum\limits_{k=0}^m \pi_k P_{kj}^{(n)}\\
					&\implies  \pi_j \geq \sum\limits_{k=0}^m \pi_k P_{kj}^{(n)}
				\end{split}
			\end{equation*}

			\item
			\begin{equation*}
				\begin{split}
					\sum\limits_{k=0}^\infty \sum\limits_{j=0}^\infty \pi_k P_{kj}^{(n)} &\geq
					\sum\limits_{k=0}^m \sum\limits_{j=0}^\infty \pi_k P_{kj}^{(n)}  \\
					&=(\text{since } \sum\limits_{j=0}^\infty P_{kj}^{(n)} = 1 )\\
					&=\sum\limits_{k=0}^m \pi_k \quad \forall m\\
					\text{suppose } \exists j > 1 : \pi_j &> \sum\limits_{k=0}^\infty \pi_k P_{kj}^{(n)} \\
					\text{then we have that } \sum\limits_{j=0}^\infty \pi_k &> \sum\limits_{j=0}^\infty \sum\limits_{k=0}^\infty \pi_k P_{kj}^{(n)} > \sum\limits_{k=0}^\infty \pi_k \quad \text{ABSURD.} \\
					&\implies \pi_j = \sum\limits_{k=0}^\infty \pi_k P_{kj}^{(n)}
				\end{split}
			\end{equation*}

			\item
			\begin{equation*}
				\begin{split}
			 		\pi_j = \sum\limits_{k=0}^\infty \pi_k P_{kj}^{(n)}, \qquad |P_{kj}^{(n)}| &\leq 1 \quad \forall n,i,k \\
					\text{using an appropriate theorem for sliding the}&\text{ limit inside the infinite sum we have:}\\
					\pi_j = \sum\limits_{k=0}^\infty  \pi_k \lim_{n\to\infty} P_{kj}^{(n)} &= (\sum\limits_{k=0}^\infty \pi_k) \pi_j\\
					\text{and now since $\pi_j>0$}
					&\implies \sum\limits_{k=0}^\infty \pi_k = 1
				\end{split}
			\end{equation*}
			\textbf{\textit{This concludes the existence proof.}}

			\item
			Let's now prove the \textbf{\textit{Uniqueness}} \\
			Suppose $X_j$ is a solution
			\begin{equation}
				\begin{split}
					X_j =
					 \sum\limits_{i=0}^\infty X_i P_{ij} &=
					 \sum\limits_{i=0}^\infty ( \sum\limits_{k=0}^\infty X_k P_{ki} ) P_{ij} \geq
					 \sum\limits_{k=0}^m X_k \\
					 \sum\limits_{i=0}^\infty P_{ki} P_{ij} &= \sum\limits_{k=0}^m X_k P_{kj}^{(2)}
					 \quad \forall m
					 \Rightarrow X_j \geq \sum\limits_{k=0}^\infty X_k P_{kj}^{(n)}\qquad \forall n
				\end{split}
			\end{equation}

			This is the same result of $3^{rd}$ step, so as in $3^{rd}$ step, we can prove by contradiction that this inequality is in fact an equality.

			$$\implies X_j = \sum\limits_{k=0}^\infty X_k P_{kj}^{(n)} \quad \forall n$$
			as $n \to \infty$ we have:

			\begin{equation}
				X_j = (\sum\limits_{k=0}^\infty X_k ) \pi_j \Rightarrow X_j = \pi_j
			\end{equation}

			So it's unique.
		\end{enumerate}
	\end{proof}

	\begin{lemma}
	  If $0 < p_i < 1 ~,~ i=0,1,2.\cdots $, then:
		\begin{equation}\label{limprodpi}
			\lim_{m \to \infty} \prod_{i=0}^{m}(1-p_i) = 0
		\end{equation}
		if and only if
		\begin{equation}\label{pitoinfty}
			\sum\limits_{i=0}^\infty p_i = \infty
		\end{equation}
	\end{lemma}

	\begin{proof}
		\begin{enumerate}
			\item Assume \eq{pitoinfty} is true. \\
				Since the series expansion for $e^{-p_i}$ is an alternating series with terms decreasing in absolute value, we can write:
				\begin{equation}
					1-p_i < 1-p_i + \frac{p_i^2}{2!} - \frac{p_i^3}{3!} + \cdots = ~e^{-p_i} \quad with ~i\ge 0
				\end{equation}
				applying the product to both members we obtain
				\begin{equation}
					\prod_{i=0}^{k-1} (1-p_i) < e^{-\sum\limits_{i=0}^{k-1}p_i}
				\end{equation}
				But by assumption $\sum\limits_{i=0}^\infty p_i = \infty$ hence,
				$$ \lim_{m \to \infty} \prod_{i=0}^{m}(1-p_i) = 0 $$

			\item Let's prove the following inequality:
			$$ \prod_{i=j}^m(1-p_i) > 1-\sum\limits_{i=j}^m p_i \quad \forall m \ge j+1$$
			We can prove it recursively:
			$$(1-p_j)(1-p_{j+1}) = 1-p_j - p_{j+1} + p_jp_{j+1} > 1-p_j - p_{j+1}$$
			Iterating we obtain:
			\begin{eqnarray*}
				\prod_{i=j}^{m+1}(1-p_i) = (1-p_{m+1})\prod_{i=j}^m(1-p_i) > (1-p_{m+1})(1-\sum\limits_{i=j}^m p_i) = \\
				1- \sum\limits_{i=j}^{m+1} p_i + p_{m+1}\sum\limits_{i=j}^m p_i
			\end{eqnarray*}
			Assume now that $\sum\limits_{i=1}^\infty p_i < \infty$, then there must exist an index $j>1$ s.t. $\sum\limits_{i=j}^\infty p_i < 1$. \\
			Then we have:
			$$ \lim_{m \to \infty} \prod_{i=j}^m (1-p_i) > \lim_{m \to \infty} (1-\sum\limits_{i=j}^m p_i) > 0 $$
		\end{enumerate}
	\end{proof}

	\begin{definition}[lesson 22/03/17]
		$$\Pi_{ik}^n(c) = \prob[X_n = k \in c, X_l \notin c, l=1, ... , n-1 | X_0 =i]$$
		$c$ is an aperiodic recurrent class, $i$ is a transient state.
		$$\Pi_i^n(c) = \sum\limits_{k \in c} \Pi_{ik}^n(c); \qquad \Pi_i(c) = \sum\limits_{n=1}^\infty \Pi_i^n(c)$$
	\end{definition}

	\begin{theorem}
		Let $i$ be a transient state and$j \in c$, where $c$ is an aperiodic recurrent class:
		$$\lim_{n\to\infty} P_{ij}^{(n)} = \Pi_i(c) \lim_{n\to\infty} P_{jj}^{(n)} = \Pi_i(c) \Pi_j$$
	\end{theorem}




%\begin{theorem}
%Let $f$ be a function whose derivative exists in every point, then $f$ is
%a continuous function.
%\end{theorem}

%\begin{theorem}[Pythagorean theorem]
%\label{pythagorean}
%This is a theorema about right triangles and can be summarised in the next
%equation
%\[ x^2 + y^2 = z^2 \]
%\end{theorem}

%And a consequence of theorem \ref{pythagorean} is the statement in the next
%corollary.

%\begin{corollary}
%There's no right rectangle whose sides measure 3cm, 4cm, and 6cm.
%\end{corollary}

%You can reference theorems such as \ref{pythagorean} when a label is assigned.

%\begin{lemma}
%Given two line segments whose lengths are $a$ and $b$ respectively there is a
%real number $r$ such that $b=ra$.
%\end{lemma}

%\begin{proof}
%To prove it by contradiction try and assume that the statemenet is false,
%proceed from there and at some point you will arrive to a contradiction.
%\end{proof}
