\chapter{Polarizzazione del campo \textsc{EM}}
Andremo ora ad analizzare come si comporta il campo elettromagnetico nello spazio, in particolare la sua evoluzione e i casi degeneri.

Iniziamo con un'analisi nel dominio del tempo del campo, prestando attenzione alla differenza con il corso di Fisica 2: mentre nell'altro corso si è osservata la polarizzazione del mezzo, ora analizzeremo la polarizzazione dell'onda elettromagnetica.
\begin{esp}
	\ert = & \sum\limits_{n=1}^3 \hat{x}_n^{\prime\prime} \cdot \E_n(\r)\cdot \cos\left[\omega t + \phi_{E_n}(\r)\right] \\
	= & \sum\limits_{n=1}^3 \hat{x}_n^{\prime\prime} \cdot \E_n(\r)\cdot \cos(\omega t)\cdot \cos\left[ \phi_{E_n}(\r)\right] - sin(\omega t)\cdot sin\left[ \phi_{E_n}(\r)\right]\\
	= & \cos(\omega t) \cdot \sum\limits_{n=1}^3 \hat{x}_n^{\prime\prime} \cdot \E_n(\r)\cdot \cos \left[ \phi_{E_n}(\r) \right] \\
	&+\sin(\omega t) \cdot \sum\limits_{n=1}^3 \hat{x}_n^{\prime\prime} \cdot \E_n(\r)\cdot \sin \left[ \phi_{E_n}(\r) \right] \\
\end{esp}

Raggruppando i termini, si ha in conclusione che
\begin{equation} \label{eq:ert}
	\ert = \E^R(\r)\cdot \cos(\omega t) - \E^I(\r)\cdot sin(\omega t)
\end{equation}

\newpage
con il vettore di Steinmetz del campo elettrico scomposto in parte reale e immaginaria
\begin{equation*} \begin{dcases}
	\E^R(\r) &= \sum\limits_{n=1}^3 \hat{x}_n^{\prime\prime} \cdot \E_n(\r)\cdot \cos \left[ \phi_{E_n}(\r) \right] \\
	\E^I(\r) &= \sum\limits_{n=1}^3 \hat{x}_n^{\prime\prime} \cdot \E_n(\r)\cdot sin \left[ \phi_{E_n}(\r) \right]
\end{dcases} \end{equation*}

Possiamo osservare il vettore di Steinmetz si può ottenere direttamente dalla formula del campo.
\begin{equation*} \begin{dcases}
	\e(\r, t=0) = \E^R(\r) \\
	\e \left( \r, t = \frac{\pi}{2\omega} \right) = -\E^I(\r)
\end{dcases} \end{equation*}

\paragraph{Proprietà di polarizzazione del campo}
A partire dall'equazione \ref{eq:ert} possiamo osservare alcune simmetrie del campo $\E$.
\begin{enumerate}
	\item il campo elettromagnetico giace sempre su un piano di polarizzazione, qualunque $t$ si scelga, determinato da $\E^I(\r)$ e $\E^R(\r)$
	\item il campo elettromagnetico traccia una curva chiusa
	\item la curva chiusa è, in generale, un ellisse
\end{enumerate}

Queste affermazioni possono essere provate come segue, scomponendo il campo $\vec{E}$ lungo due versori arbitrari $\hat{x}^\prime$ e $\hat{y}^\prime$ del piano di polarizzazione.
\begin{esp*}
	& \begin{dcases}
		\E^R = E^R_{x^{\prime}} \cdot \hat{x}^{\prime}+E^R_{y^{\prime}} \cdot \hat{y}\prime \\
		\E^I = E^I_{x^{\prime}} \cdot \hat{x}^{\prime}+E^I_{y^{\prime}} \cdot \hat{y}\prime \\
	\end{dcases}
\end{esp*}

\begin{esp*}
	\e &= (E^R_{x^{\prime}}\cdot \hat{x}^{\prime}+ E^R_{y^{\prime}})\cdot \cos(\omega t) - (E^I_{x^{\prime}}\cdot \hat{x}^{\prime}+ E^I_{y^{\prime}})\cdot sin(\omega t)\\
	&= \left[E^R_{x^{\prime}} \cdot \cos(\omega t)-E^I_{x^{\prime}}\cdot sin(\omega t)\right]\cdot \hat{x}^{\prime} + \left[E^R_{y^{\prime}} \cdot \cos(\omega t) -E^I_{y^{\prime}}\cdot sin(\omega t)\right]\cdot \hat{y}\prime \\
\end{esp*}
	o in forma matriciale
\begin{esp}
	\begin{pmatrix}
		e_{x^{\prime}} \\ e_{y^{\prime}}
	\end{pmatrix}
	&=
	\underbrace{\begin{pmatrix}
		E_{x^{\prime}}^R & -E_{x^{\prime}}^I\\
		E_{y^{\prime}}^R & -E_{y^{\prime}}^I
	\end{pmatrix}}_{\text{Deformazione lineare}}
	\cdot
	\underbrace{\begin{pmatrix}
		\cos(\omega t) \\ sin(\omega t)
	\end{pmatrix}}_{\text{circonferenza}}
\end{esp}

Passando quindi alle coordinate cartesiane principali, si ottiene
\begin{esp}
\begin{pmatrix} e_{x} \\ e_{y} \end{pmatrix}
&= \begin{pmatrix}
	a \cdot \cos(\omega t + \Phi) \\
	\pm b \cdot sin(\omega t + \Phi)
\end{pmatrix}
\end{esp}
dove $a,b \in \R^+$ indicano il valore dei semiassi dell'ellisse, mentre il $\pm$ indica il verso \emph{destrorso} o \emph{sinistrorso} della polarizzazione ellittica del campo elettrico. Ci sono ora due casi degeneri principali:
\begin{enumerate}
	\item $a = b$ l'ellisse degenera in una circonferenza, per cui la polarizzazione sarà circolare destrorsa o sinistrorsa;
	\item $\begin{cases}a=0 \\ b \neq 0 \end{cases} \vee \quad \begin{cases} a\neq 0 \\ b = 0 \end{cases}$ l'ellisse degenera in un segmento e la polarizzazione sarà rettilinea verticale o orizzontale.
\end{enumerate}

Queste proprietà si posso ricavare anche con i vettori di Steinmetz.
\begin{esp}
	\e(t) &= \Re[\E\cdot e^{\jmath \omega t}] \quad \text{where } \E \in \C^3 \\
	&=\Re\left[\left(\E^R + \jmath \E^I\right)\cdot\left(\cos(\omega t) + \jmath \sin(\omega t)\right)\right] \\
	&= \E^R \cdot \cos(\omega t) - \E^I\cdot sin(\omega t) \\
	&= a \cdot \cos(\omega t + \Phi)\cdot \hat{x} \pm b \cdot sin(\omega t + \Phi)\cdot \hat{y}
\end{esp}

Ruotando il sistema di riferimento di un angolo $\delta$ la proprità resta valida: è sufficiente applicare la corrispodente trasformazione ai campi considerati.
\begin{equation}
	\begin{pmatrix} E_{x^{\prime}} \\ E_{y^{\prime}} \end{pmatrix} =
	\begin{pmatrix}
		 \cos(\delta) & sin(\delta) \\ -sin(\delta) & \cos(\delta)
	\end{pmatrix} \cdot
		\begin{pmatrix} E_{x} \\ E_{y} \end{pmatrix}
\end{equation}

\section{Rapporto di polarizzazione rettilinea}
Definiamo il rapporto di polarizzazione $p$ nel sistema di assi principali come
\begin{equation}
	p = \jmath \frac{E_y}{E_x} \in \C
\end{equation}
mentre in un sistema di assi generico, si ottiene, in modo analogo alle linee di trasmissione,
\begin{equation}
	p^\prime = \jmath \frac{E_y}{E_x} = \frac{p - \jmath tg(\delta)}{1-\jmath p \cdot tg(\delta)}
\end{equation}
Risulta interessante notare come, per qualsiasi angolo $\delta$ il segno della parte reale di $p^\prime$ non cambi. $p^\prime \in \C$ discrimina infatti, indipendentemente dal sistema di referimento, i diversi tipi di polarizzazione
\begin{equation}\begin{cases}
	\Re[p^\prime] >0 & \text{ellittica destrorsa} \\
	\Re[p^\prime] <0 & \text{ellittica sinistrorsa} \\
	\Re[p^\prime] =0 & \text{rettilinea ``orizzontale''} \\
	\Re[p^\prime] =\infty & \text{rettilinea ``verticale''} \\
	p = p^\prime = 1 & \text{circolare destrorsa} \\
	p = p^\prime = 1 & \text{circolare sinistrorsa}
\end{cases}\end{equation}

% \textbf{\textsc{Nota Bene}} la polarizzazione circolare prevede che $p=p\prime$ sia \textbf{puramente reale} e pari a $\pm1$

$\Re [p\prime]$ si può calcolare come
\begin{esp*}
	p^\prime
	&= \jmath \frac{E_{y^{\prime}}}{E_{x^{\prime}}} \\
	& = e^{\jmath \frac{\pi}{2}} \cdot \left \| \frac{E_{y^{\prime}}}{E_{x^{\prime}}} \right \|
	\cdot e^{j(\Phi_{y^{\prime}}-\Phi_{x^{\prime}})} \\
	&= \left \| \frac{E_{y^{\prime}}}{E_{x^{\prime}}} \right \| \cdot e^{j(\frac{\pi}{2}+\Phi_{y^{\prime}}-\Phi_{x^{\prime}})}
\end{esp*}

e dunque,
\begin{esp*}
	\Re[p^\prime]
	&= \left \| \frac{E_{y^{\prime}}}{E_{x^{\prime}}} \right \| \cdot \underbrace{\cos \left( \frac{\pi}{2} +\Phi_{y^{\prime}}-\Phi_{x^{\prime}} \right)}_{\text{verso di polarizzazione}}
\end{esp*}

%%% Local Variables:
%%% mode: latex
%%% TeX-master: "antenne"
%%% End:
