\chapter{Polarizzazione del campo elettromagnetico}
Andremo ora ad analizzare come si comporta il campo elettromagnetico nello spazio, in particolare la sua evoluzione e i casi degeneri.

Iniziamo con un'analisi nel dominio del tempo del campo, prestando attenzione alla differenza con il corso di Fisica 2: mentre nell'altro corso si è osservata la polarizzazione del mezzo, ora analizzeremo la polarizzazione dell'onda elettromagnetica.
\begin{esp}
  \ert &= \sum\limits_{n=1}^3 \hat{x}_n^{\prime\prime} \cdot \E_n(\r)\cdot cos\left[\omega t + \phi_{E_n}(\r)\right] \\
  &=\ert &= \sum\limits_{n=1}^3 \hat{x}_n^{\prime\prime} \cdot \E_n(\r)\cdot cos(\omega t)\cdot cos\left[ \phi_{E_n}(\r)\right] - sin(\omega t)\cdot sin\left[ \phi_{E_n}(\r)\right]\\
  &=cos(\omega t) \cdot \sum\limits_{n=1}^3 \hat{x}_n^{\prime\prime} \cdot \E_n(\r)\cdot cos \left[ \phi_{E_n}(\r) \right] \\
  &+sin(\omega t) \cdot \sum\limits_{n=1}^3 \hat{x}_n^{\prime\prime} \cdot \E_n(\r)\cdot sin \left[ \phi_{E_n}(\r) \right] \\
  \implies \ert &= \E^R(\r)\cdot cos(\omega t) - \E^I(\r)\cdot sin(\omega t)
&\text{with } \\
\E^R(\r) &= \sum\limits_{n=1}^3 \hat{x}_n^{\prime\prime} \cdot \E_n(\r)\cdot cos \left[ \phi_{E_n}(\r) \right] \\
\E^I(\r) &= \sum\limits_{n=1}^3 \hat{x}_n^{\prime\prime} \cdot \E_n(\r)\cdot sin \left[ \phi_{E_n}(\r) \right]
\end{esp}

We can say that $\e(\r,0) = \E^R(\r)$ if $t=0$ and $\e(\r,\frac{pi}{2\omega}) = -\E^I(\r)$

\textsc{Possiamo quindi affermare che}
\begin{enumerate}
  \item Il campo elettromagnetico giace in un dato punto $\r$ su un piano di polarizzazione, qualunque t si scelga
  \item Il campo elettromagnetico traccia una curva chiusa
  \item La curva chiusa, in genere, è un ellisse.
\end{enumerate}
