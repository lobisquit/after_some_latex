\section{Propagazione guidata}
Le equazioni delle onde piane uniformi trovate sinora si possono quindi applicare anche per la propagazione di onde elettromagnetiche in conduttori, per esempio
\begin{itemize}
  \item \textbf{linee bifilari} come il doppino telefonico
  \item \textbf{linee coassiali}, per esempio  cavo che collega l'antenna al televisore
  \item \textbf{le guide rettangolari}, usate principalmente per applicazioni in cui è necessario trasportare molta potenza (radar o trasmissioni satellitari)
  \item \textbf{le linee a microstriscia} utilizzate per trasportare segnali in circuiti stampati
\end{itemize}
Accenniamo ora come si propaga l'onda in alcune di queste guide, poiché il segnale prima di essere emesso dall'antenna dovrà essere trasportato in una qualche maniera.
\paragraph{Campi nella guida}

I campi elettromagnetici che si propagano in guide d'onda il vettore di propagazione  e l'impedenza d'onda sono:
\begin{esp}
  |\k| &=\beta= \omega \sqrt{\mu \epsilon} \\
  \eta &= \frac{E_o}{H_o} = \sqrt{\frac{\mu}{\epsilon}}
\end{esp}

\paragraph{Linea bifilare}
Le linee di forza del campo elettrico nella linea bifilare sono simili a quelle di un condensatre, ossia $\forall$ punto $\E\perp\H\perp\k$. In figura \#TODO\# sono rappresentate le linee di campo.

\paragraph{Linea coassiale}
In questo caso le linee di campo elettrico sono ortogonali alla circonferenza, mentre quelle di campo magnetico sono concentriche all'asse della linea, come visibile in fig \#TODO\#

\paragraph{Impedenza caratteristica}
Cerchiamo ora di valutare l'impedenza caratteristica della linea e, in particolare, come sfruttarla per ridurre al minimo le perdite.

La tensione e la corrente nella linea coassiale si possono scrivere come
\begin{esp}
  V&= \int_{r_{int}}^{r_{ext}} \E \cdot d\vec{l} \\
  I&= \oint \H \cdot d\vec{l} \\
  &\implies Z_c = \frac{V_o}{I_o} \text{ impedenza caratteristica della linea}
\end{esp}
Definendo inoltre
\begin{itemize}
  \item $l$ l'induttanza per unità di lunghezza $\frac{H}{m}$
  \item $c$ la capacità per unità di lunghezza $\frac{F}{m}$
\end{itemize}
si ottiene che l'impedenza caratteristica risulta
\begin{equation}
  Z_c = \sqrt{\frac{l}{c}} = \underbrace{F_{forma}}_{\text{fattore di forma}} \cdot \sqrt{\frac{\mu}{\epsilon}}
\end{equation}

Possiamo quindi riscrivere le equazioni per la tensione e la corrente nel cavo coassiale come
\begin{esp}
  V(z) &= V_o \cdot e^{-\jmath \beta \cdot z}\\
  I(z) &= I_o \cdot e^{-\jmath \beta \cdot z}
\end{esp}


\chapter{Antenne}
In questo capitolo analizzeremo le antenne, partendo inizialmente con i potenziali elettromagnetici e cercando di modificare le equazioni di Maxwell affinché siano più comode da studiare nei nostri casi. Vedremo quindi l'equazione di Helmoltz e la scelta di Lorentz. Inizieremo quindi a studiare le antenne partendo dal caso del dipolo elementare, ideale e non realizzabile nella realtà, ma utile per fare le prime approssimazioni. Estenderemo quindi il dipolo elementare al dipolo corto per studiare poi le antenne lineari, le schiere e le patch.

Riprendiamo le equazioni di Maxwell scritte con i vettori di Steinmets:
\begin{equation}\begin{cases}
  \rot\E = -\jmath \omega \mu\H(\r) \textcolor[rgb]{0.8,0,0}{+ \M} \\
  \rot\H = \jmath  \omega \epsilon \E + \J\\
  \diverg\H=0\\
  \diverg \E = \frac{\rho}{\epsilon}
\end{cases}\end{equation}
Ora aggiungiamo un termine $\M$ fittizio, affinché le equazioni risultino simmetriche e si possano scrivere, in seguito, le equazioni per le sorgenti equivalenti. ($\M$ non può esistere, in quanto non esiste il monopolo magnetico)

\subsubsection{Potenziali elettromagnetici}
$\forall \A , \diverg(\rot \A) = 0$ per quanto trovato nel paragrafo dell'identità notevole (\ref{sec:id-not}). Definiamo quindi $\A$  \textbf{potenziale vettore magnetico} il campo per cui $\H = \frac{1}{\mu} \cdot \rot \A$.
\paragraph{Teorema di Helmoltz}
Un campo vettoriale $\A$ è univocamente definito se sono fissati $\rot \A$ e $\diverg \A$
----------------

Dobbiamo dunque definire in maniera intelligente $\diverg \A$:
\begin{esp}
  \rot\E &= - \jmath \omega \cdot \mu \frac{1}{\mu} \rot \A \quad \Leftrightarrow \quad \rot\left(\rot \E + \jmath \omega \A\right)=0 \\
  \forall \Phi& \text {, scalare, } \rot(\diverg\Phi)=0\\
  \E +& \jmath \omega \A = -\nabla\Phi \quad \implies \E = -\jmath \omega \A - \nabla \Phi  \\
  \implies &[\Phi] = V
\end{esp}
Definiamo quindi $\Phi$ potenziale scalare elettrico associato ad $\A$. Analizziamo ora la seconda equazione di Maxwell
\begin{esp}
  \rot \H &= \jmath \omega \epsilon_c \cdot \E + \J \\
  \mu \cdot \rot\left(\frac{1}{\mu} \rot \A\right) &= \jmath \omega \mu \epsilon_c \cdot \left(-\jmath \omega \A - \nabla \Phi \right) + \J \mu  \\
  -\nabla^2\A + \diverg(\nabla \A) = \omega^2 \mu \epsilon_c \A - \jmath \omega \mu \epsilon_c \nabla \Phi + \mu \J \\
  \underbrace{\nabla^2 \A + \omega^2 \mu \epsilon_c \A}_{\parbox[c]{2cm}{\text{Struttura dell'eq. di Helmoltz}}} &= -\mu\J + \diverg(\nabla\A) + \jmath \omega \mu \epsilon_c \nabla \Phi \\
  &=\underbrace{-\mu\J}_{\parbox[c]{2cm}{\text{termine noto: presenza sorgenti}}} + \diverg\left(\underbrace{\nabla \A + \jmath \omega \mu \epsilon_c \Phi}_{\parbox[c]{2cm}{\text{complicazione? scelgo il valore di $\A$}}}\right)\\
\end{esp}
\paragraph{Scelta di Lorentz}
Poiché abbiamo ancora un grado di libertà per definire $\A$, possiamo scegliere
\begin{equation}
  \diverg \A = -\jmath \omega \mu \epsilon_c \Phi
\end{equation}
Otteniamo quindi l'\textbf{equazione di Helmoltz non omogenea}:
\begin{equation}
  \nabla^2\A + \omega^2 \mu\epsilon_c\A =-\mu \J
\end{equation}
Dobbiamo quindi determinare il valore di $\Phi$, cosa che faremo sfruttando la terza legge di Maxwell
\begin{esp*}
  \diverg\E &= \frac{\rho}{\epsilon} \quad \implies\quad \diverg\left(-\jmath \omega \A - \nabla \Phi  \right)= \frac{\rho}{\epsilon}\\
  \implies & \jmath \omega \A + \nabla \Phi= -\frac{\rho}{\epsilon}  \\
  &\jmath\omega\left(-\jmath \omega\mu\epsilon_c\Phi\right) + \nabla^2\Phi = - \frac{\rho}{\epsilon}\\
  &\text{Otteniamo quindi l'equazione scalare di Helmoltz non omogenea}
  \omega^2\mu\epsilon_c\Phi + \nabla^2 \Phi = -\frac{\rho}{\epsilon}
\end{esp*}
\begin{esp}\label{eq:helmolts-lorentz}
  \nabla^2\A + \omega^2\mu\epsilon_c\A &= -\mu\J \\
  \nabla^2\Phi + \omega^2\mu\epsilon_c\Phi &= -\frac{\rho}{\epsilon}
\end{esp}
Poiché le equazioni di Helmoltz trovate sono lineari, utilizzeremo spesso il principio di sovrapposizione degli effetti

\section{Dipolo elementare}
