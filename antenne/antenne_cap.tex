\section{Propagazione guidata}
Le equazioni delle onde piane uniformi trovate sinora si possono quindi applicare anche per la propagazione di onde elettromagnetiche in conduttori, per esempio
\begin{itemize}
  \item \textbf{linee bifilari} come il doppino telefonico
  \item \textbf{linee coassiali}, per esempio  cavo che collega l'antenna al televisore
  \item \textbf{le guide rettangolari}, usate principalmente per applicazioni in cui è necessario trasportare molta potenza (radar o trasmissioni satellitari)
  \item \textbf{le linee a microstriscia} utilizzate per trasportare segnali in circuiti stampati
\end{itemize}
Accenniamo ora come si propaga l'onda in alcune di queste guide, poiché il segnale prima di essere emesso dall'antenna dovrà essere trasportato in una qualche maniera.
\paragraph{Campi nella guida}

I campi elettromagnetici che si propagano in guide d'onda il vettore di propagazione  e l'impedenza d'onda sono:
\begin{esp}
  |\k| &=\beta= \omega \sqrt{\mu \epsilon} \\
  \eta &= \frac{E_o}{H_o} = \sqrt{\frac{\mu}{\epsilon}}
\end{esp}

\paragraph{Linea bifilare}
Le linee di forza del campo elettrico nella linea bifilare sono simili a quelle di un condensatre, ossia $\forall$ punto $\E\perp\H\perp\k$. In figura \#TODO\# sono rappresentate le linee di campo.

\paragraph{Linea coassiale}
In questo caso le linee di campo elettrico sono ortogonali alla circonferenza, mentre quelle di campo magnetico sono concentriche all'asse della linea, come visibile in fig \#TODO\#

\paragraph{Impedenza caratteristica}
Cerchiamo ora di valutare l'impedenza caratteristica della linea e, in particolare, come sfruttarla per ridurre al minimo le perdite.

La tensione e la corrente nella linea coassiale si possono scrivere come
\begin{esp}
  V&= \int_{r_{int}}^{r_{ext}} \E \cdot d\vec{l} \\
  I&= \oint \H \cdot d\vec{l} \\
  &\implies Z_c = \frac{V_o}{I_o} \text{ impedenza caratteristica della linea}
\end{esp}
Definendo inoltre
\begin{itemize}
  \item $l$ l'induttanza per unità di lunghezza $\frac{H}{m}$
  \item $c$ la capacità per unità di lunghezza $\frac{F}{m}$
\end{itemize}
si ottiene che l'impedenza caratteristica risulta
\begin{equation}
  Z_c = \sqrt{\frac{l}{c}} = \underbrace{F_{forma}}_{\text{fattore di forma}} \cdot \sqrt{\frac{\mu}{\epsilon}}
\end{equation}

Possiamo quindi riscrivere le equazioni per la tensione e la corrente nel cavo coassiale come
\begin{esp}
  V(z) &= V_o \cdot e^{-\jmath \beta \cdot z}\\
  I(z) &= I_o \cdot e^{-\jmath \beta \cdot z}
\end{esp}


\chapter{Antenne}
In questo capitolo analizzeremo le antenne, partendo inizialmente con i potenziali elettromagnetici e cercando di modificare le equazioni di Maxwell affinché siano più comode da studiare nei nostri casi. Vedremo quindi l'equazione di Helmoltz e la scelta di Lorentz. Inizieremo quindi a studiare le antenne partendo dal caso del dipolo elementare, ideale e non realizzabile nella realtà, ma utile per fare le prime approssimazioni. Estenderemo quindi il dipolo elementare al dipolo corto per studiare poi le antenne lineari, le schiere e le patch.

Riprendiamo le equazioni di Maxwell scritte con i vettori di Steinmets:
\begin{equation}\begin{cases}
  \rot\E = -\jmath \omega \mu\H(\r) \textcolor[rgb]{0.8,0,0}{+ \M} \\
  \rot\H = \jmath  \omega \epsilon \E + \J\\
  \diverg\H=0\\
  \diverg \E = \frac{\rho}{\epsilon}
\end{cases}\end{equation}
Ora aggiungiamo un termine $\M$ fittizio, affinché le equazioni risultino simmetriche e si possano scrivere, in seguito, le equazioni per le sorgenti equivalenti. ($\M$ non può esistere, in quanto non esiste il monopolo magnetico)

\subsubsection{Potenziali elettromagnetici}
$\forall \A , \diverg(\rot \A) = 0$ per quanto trovato nel paragrafo dell'identità notevole (\ref{sec:id-not}). Definiamo quindi $\A$  \textbf{potenziale vettore magnetico} il campo per cui $\H = \frac{1}{\mu} \cdot \rot \A$.
\paragraph{Teorema di Helmoltz}
Un campo vettoriale $\A$ è univocamente definito se sono fissati $\rot \A$ e $\diverg \A$
----------------

Dobbiamo dunque definire in maniera intelligente $\diverg \A$:
\begin{esp} \label{eq:AE}
  \rot\E &= - \jmath \omega \cdot \mu \frac{1}{\mu} \rot \A \quad \Leftrightarrow \quad \rot\left(\rot \E + \jmath \omega \A\right)=0 \\
  \forall \Phi& \text {, scalare, } \rot(\diverg\Phi)=0\\
  \E +& \jmath \omega \A = -\nabla\Phi \quad \implies \E = -\jmath \omega \A - \nabla \Phi  \\
  \implies &[\Phi] = V
\end{esp}
Definiamo quindi $\Phi$ potenziale scalare elettrico associato ad $\A$. Analizziamo ora la seconda equazione di Maxwell
\begin{esp}
  \rot \H &= \jmath \omega \epsilon_c \cdot \E + \J \\
  \mu \cdot \rot\left(\frac{1}{\mu} \rot \A\right) &= \jmath \omega \mu \epsilon_c \cdot \left(-\jmath \omega \A - \nabla \Phi \right) + \J \mu  \\
  -\nabla^2\A + \diverg(\nabla \A) = \omega^2 \mu \epsilon_c \A - \jmath \omega \mu \epsilon_c \nabla \Phi + \mu \J \\
  \underbrace{\nabla^2 \A + \omega^2 \mu \epsilon_c \A}_{\parbox[c]{2cm}{\text{Struttura dell'eq. di Helmoltz}}} &= -\mu\J + \diverg(\nabla\A) + \jmath \omega \mu \epsilon_c \nabla \Phi \\
  &=\underbrace{-\mu\J}_{\parbox[c]{2cm}{\text{termine noto: presenza sorgenti}}} + \diverg\left(\underbrace{\nabla \A + \jmath \omega \mu \epsilon_c \Phi}_{\parbox[c]{2cm}{\text{complicazione? scelgo il valore di $\A$}}}\right)\\
\end{esp}
\paragraph{Scelta di Lorentz}
Poiché abbiamo ancora un grado di libertà per definire $\A$, possiamo scegliere
\begin{equation}
  \diverg \A = -\jmath \omega \mu \epsilon_c \Phi
\end{equation}
Otteniamo quindi l'\textbf{equazione di Helmoltz non omogenea}:
\begin{equation}
  \nabla^2\A + \omega^2 \mu\epsilon_c\A =-\mu \J
\end{equation}
Dobbiamo quindi determinare il valore di $\Phi$, cosa che faremo sfruttando la terza legge di Maxwell
\begin{esp*}
  \diverg\E &= \frac{\rho}{\epsilon} \quad \implies\quad \diverg\left(-\jmath \omega \A - \nabla \Phi  \right)= \frac{\rho}{\epsilon}\\
  \implies & \jmath \omega \A + \nabla \Phi= -\frac{\rho}{\epsilon}  \\
  &\jmath\omega\left(-\jmath \omega\mu\epsilon_c\Phi\right) + \nabla^2\Phi = - \frac{\rho}{\epsilon}\\
  &\text{Otteniamo quindi l'equazione scalare di Helmoltz non omogenea}\\
  &\omega^2\mu\epsilon_c\Phi + \nabla^2 \Phi = -\frac{\rho}{\epsilon}
\end{esp*}
\begin{esp}\label{eq:helmolts-lorentz}
  \nabla^2\A + \omega^2\mu\epsilon_c\A &= -\mu\J \\
  \nabla^2\Phi + \omega^2\mu\epsilon_c\Phi &= -\frac{\rho}{\epsilon}
\end{esp}
Poiché le equazioni di Helmoltz trovate sono lineari, utilizzeremo spesso il principio di sovrapposizione degli effetti

\section{Dipolo elementare}
Partiamo analizzando il dipolo elementare, indicando la rappresentazione di un punto P attraverso il vettore posizione $\r$
\begin{esp}
  \r &= (x,y,z) = (|\r|,\theta,\phi) = (r,\theta,\phi) \\
  \J &= I \cdot \delta(x)\delta(y)\cdot f(z)\cdot \hat{z}
\end{esp}
Dobbiamo quindi risolvere l'equazione di Helmoltz non omogenea:
\begin{esp*}
  \nabla^2 \A + \omega^2 \mu\epsilon_c \A &= -\mu \J \quad \text{con } \A = (A_x,A_y,A_z)\\
  \begin{cases}
    \nabla^2 \A_{x,y} + \omega^2 \mu\epsilon_c\A_{x,y} =0 \\
    \nabla^2 \A_{z} + \omega^2 \mu\epsilon_c\A_{z} =-\mu\J_z
  \end{cases} \\
  \implies & A_x=A_y=0 \parbox[c]{5cm}{\text{ è una soluzione possibile e più semplice}}\\
  \forall \r &= 0 \implies \nabla^2 \A_{z} + \omega^2 \mu\epsilon_c\A_{z} =0
\end{esp*}
Supponiamo ora che $\sigma=0 \implies \epsilon_c = \epsilon \in \R$
\begin{equation*}
  \nabla^2 \A_{z} + \omega^2 \mu\epsilon\A_{z} =0
\end{equation*}

Le soluzioni dovranno avere simmetria sferica
\begin{esp*}
  &A_z(r,\theta,\phi) =A_z(r) = \frac{f(r)}{r} \\
  &\implies \nabla^2 A_z = \frac{1}{r^2} \deriv{}{r}\left(r^2 \deriv{A_z(r)}{r}\right) \\
  &\implies  \frac{1}{r^2} \deriv{}{r}\left(r^2 \deriv{\frac{f(r)}{r}}{r}\right) + \omega^2 \mu\epsilon\frac{f(r)}{r}\\
  & \frac{1}{r} \deriv{}{r}\left(r^2 \deriv{\frac{f'(r)\cdot r - f}{r^2}}{r}\right) + \omega^2 \mu\epsilon\frac{f(r)}{r}\\
  &\frac{1}{r} \cdot \left(f\prime\prime(r)\cdot r + f\prime(r)\cdot 1 - f\prime(r) \right) + \omega^2\mu\epsilon f(r) = 0\\
\end{esp*}
dove $f\prime(r) = \deriv{f(r)}{r}$ e $f\prime\prime =\deriv[2]{f(r)}{r}$
Concludiamo quindi che
\begin{equation*}
  f\prime\prime(r) + \omega^2\mu\epsilon f(r) = 0
\end{equation*}
Integrando otteniamo
\begin{equation}
  f(r) = c \cdot e^{-\jmath \beta r} + d \cdot e^{\jmath \beta r}
\end{equation}
con $c,d \in \C$ costanti di integrazione, $\beta = \omega\sqrt{\mu\epsilon}$.

$A_z(r)$ risulta quindi essere
\begin{equation}
  A_z(r) = \frac{c}{r}\cdot e^{-\jmath \beta r} +\frac{d}{r}\cdot e^{\jmath \beta r}
\end{equation}
che non è più un'onda piana.

Nel dominio temporale, ora troviamo che:
\begin{equation}
  a_z(r,t) = \Re \left[A_z(r)\cdot e^{\jmath \omega t}\right] \propto cos(\omega t \pm \beta r)
\end{equation}
che rende evidente come non ci sia dipendenza rispetto $\theta$ e $\phi$. Questo implica che l'onda si propaga radialmente. Il $\pm$ indica la presenza di un'onda progressiva e, eventualmente, una regressiva, la quale si annulla se non vi sono ostacoli.

Definiamo ora la velocità di fase: $v_f = \frac{\omega}{\beta}$
Analizzando $A_z$ otteniamo che
\begin{itemize}
  \item $|A_z| = \left|\frac{c}{r}\right|$ la superficie equiampiezza è una sfera di raggio r
  \item $\measuredangle A_z = \beta r$ la superficie equifase è una sfera di raggio r
\end{itemize}
ossia l'onda è di tipo sferico.

integrando nel volume della sfera otteniamo
\begin{esp*}
  \int_V \nabla^2 A_z \cdot dV + \omega^2 \mu \epsilon \int_V A_z \cdot dV &= -\mu \int_V J_z \cdot dV \\
  \int_V \diverg(\nabla A_z) \cdot dV + \omega^2 \mu \epsilon \int_V A_z \cdot dV &= -\mu \int_V I \cdot \delta(x)\delta(y)\cdot F(z) \cdot dV \\
  \text{ con } F(z) &=
  \begin{cases}
    1 & |z| \le \frac{\Delta z}{2} \\ 0 &\text{altrove}
  \end{cases}
\end{esp*}
Attraverso il teorema della divergenza possiamo continuare
\begin{esp*}
  \int_{S_V} \nabla A_z \cdot \r \cdot dS + \omega^2 \mu \epsilon \int_V A_z \cdot dV &= -\mu \int_V I \cdot \delta(x)\delta(y)\cdot F(z) \cdot dV \\
  \nabla A_z = \hr \cdot \deriv{A_z}{r} &= \hr \cdot c \left(\frac{-\jmath \beta \ejbr -\ejbr }{r^2}\right) \\
  c \cdot \int_{S_V} \left(-\jmath\beta\frac{\ejbr}{r} -\frac{\ejbr}{r} \right) \hr \cdot \hr dS + \int_V \oqme A_z dV &= -\mu \int_V I \cdot \delta(x)\delta(y)\cdot F(z) \cdot dV \\
  \A_z(r) = \frac{\mu\cdot I \cdot \Delta z}{4\pi}\cdot \frac{\ejbr}{r} \cdot \hz
\end{esp*}
Dove $c=\frac{\mu\cdot I \cdot \Delta z}{4\pi}$, calcolato attraverso il teorema dei residui.

Otteniamo quindi le equazioni per il csmpo elettrico e magnetico
\begin{esp}\label{eq:campoDE}
  \H &= \frac{1}{\mu}  \cdot \rot \A \stackrel{=}{conti} \frac{I \cdot \Delta z}{4\pi} \cdot \left(\frac{\jmath \beta}{r} + \frac{1}{r^2}\right)\cdot\ejbr \cdot sin(\theta)\hphi\\
  \E &= -\jmath\omega\A - \nabla \left(\frac{\diverg\A}{-\jmath\omega\mu\epsilon}\right) \stackrel{=}{conti}\\
  &= \eta \frac{I \cdot \Delta z}{4\pi} \cdot\left(\frac{\jmath \beta}{r} + \frac{1}{r^2} + \frac{1}{\jmath\beta r^3}\right)\cdot\ejbrp \cdot sin(\theta)\hth +
  \eta \frac{I \cdot \Delta z}{2\pi} \cdot\left( \frac{1}{r^2} + \frac{1}{\jmath\beta r^3}\right)\cdot\ejbr \cdot cos(\theta)\hr
\end{esp}

L'andamento della soluzione del campo elettrico permette di individuare la regione di campo lontano, in particolare con le approssimazioni per r sufficientemente grande, si ottiene:
\begin{esp*}
  \frac{\jmath\beta}{r} &= \frac{\jmath 2\pi}{\lambda \cdot r} \quad \lambda r \gg  \lambda \implies \text{regione campo lontano}\\
  \frac{1}{\jmath\beta r^3} = \frac{\lambda}{\jmath 2 \pi r^3} \to 0 \\
  \H &\approx \frac{I \cdot \Delta z}{4\pi} \cdot \frac{\jmath\beta}{r} \cdot \ejbr sin(\theta)\hphi\\
  \E &\approx \eta \cdot \frac{I \cdot \Delta z}{4\pi} \cdot \jmath\beta\cdot \frac{\ejbr{r}}{r} sin(\theta) \cdot \hth
\end{esp*}

Possiamo quindi scrivere le condizioni di campo lontato:
\begin{equation}\begin{cases}
  E_\theta \approx \jmath\beta\eta\cdot\frac{I \cdot \Delta z}{4\pi} \cdot \frac{\ejbr}{r} \cdot sin(\theta) = \jmath \eta\cdot\frac{I \cdot \Delta z}{2\lambda} \cdot \frac{\ejbr}{r} \cdot sin(\theta) \\
  E_\theta \approx \jmath\beta\cdot\frac{I \cdot \Delta z}{4\pi} \cdot \frac{\ejbr}{r} \cdot sin(\theta) = \jmath \cdot\frac{I \cdot \Delta z}{2\lambda} \cdot \frac{\ejbr}{r} \cdot sin(\theta) \\
\end{cases}\end{equation}

Possiamo quindi trovare una relazione tra l'impedenza d'onda e il campo EM, attraverso
\begin{equation}
  \frac{E_\theta}{H_\Phi} = \eta \in \R
\end{equation}
L'onda è quindi localmente piana.

Il vettore di Poynting risulta quindi
\begin{esp*}
  \S = \frac{\E \times\H^*}{2} &= \frac{E_\theta \cdot \hth \times H_\Phi \cdot \hphi}{2} = \frac{E_\theta\cdot H_\Phi}{2}\cdot \hth \times \hphi \\
  &= \frac{E_\theta\cdot H_\Phi}{2} \cdot \hr = \frac{\eta|I|^2 \Delta z^2}{8\lambda^2\r^2} \sin^2(\theta)\cdot \hr \\
  \frac{\eta |I|^2}{8} \cdot \left(\frac{\Delta z}{\lambda}\right) \cdot \frac{sin^2(\theta)}{r^2} \cdot \hr = \Re[\P]
\end{esp*}
dove $\Re[\P]$ è il flusso di potenza attiva
