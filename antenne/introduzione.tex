\chapter{Introduzione}
\section{Sistemi di riferimento}
Nel corso verranno utilizzate due sistemi di coordinate, \emph{cartesiano} e \emph{sferico} basate su versori differenti.

Le due rappresentazioni di un medesimo punto $P$, definito del vettore posizione $\r$, saranno le seguenti.
\begin{esp}
	P&=(x,y,z) = \r = x \cdot \hat{x} + y \cdot \hat{y} +z \cdot \hat{z}
	& \text{ con\quad} x\perp y \perp z \perp x \\
	P&=(r,\theta,\phi) = r \cdot \hat{r} = r \cdot \hat{r}(\theta,\phi)
		& \text{ con\quad} r \perp \theta \perp \phi \perp r
	\label{eq:riferimento}
\end{esp}

\section{Operatori differenziali spaziali}
Introduciamo qui gli operatori differenziali da applicare ad un generico vettore $\A\left(\r\right)=\left(A_x(\r),A_y(\r),A_z(\r)\right)$ nello spazio $\mathbb{R}^3$.
\begin{description}
	\item \textbf{Vettore Nabla}
	$ \nabla = \left(\frac{\partial}{\partial x},\frac{\partial}{\partial y},\frac{\partial}{\partial z}\right)$

	Nabla viene, impropriamente, definito come un vettore di operatori: le loro diverse applicazioni al vettore $\vec{A}$ genera le seguenti operazioni.

	\item \textbf{Rotore} $\rot$

	Calcolando il ``determinante'' della matrice con il metodo di Laplace partendo dalla prima riga, si ottiene la formula del rotore.
	\begin{esp}
		\rot \A(\r) &=
		\begin{vmatrix}
			\hat{x} &\hat{y} &\hat{z} \\
			\frac{\partial}{\partial x} & \frac{\partial}{\partial y} & \frac{\partial}{\partial z} \\
			A_x & A_y & A_z
		\end{vmatrix} \\
		&= \left(\frac{\partial A_z}{\partial y} - \frac{\partial A_y}{\partial z}\right) \hat{x} -
		\left(\frac{\partial A_z}{\partial x} - \frac{\partial A_x}{\partial z}\right) \hat{y} +
		\left(\frac{\partial A_y}{\partial x} - \frac{\partial A_x}{\partial y}\right) \hat{z}
	\end{esp}

	\item \textbf{Divergenza} $\diverg$

	\begin{equation}
		\diverg \vec{A} = \frac{\partial A_x}{\partial x} + \frac{\partial A_y}{\partial y} + \frac{\partial A_z}{\partial z}
	\end{equation}

	\item \textbf{Gradiente} $\nabla$

	Dato $\Phi(\r)$ uno scalare, il suo gradiente risulta
	\begin{equation}
		\nabla \Phi = \left(\frac{\partial \Phi}{\partial x} , \frac{\partial \Phi}{\partial y}, \frac{\partial \Phi}{\partial z}\right)
	\end{equation}

	\item \textbf{Laplaciano scalare} $\nabla^2$

	Data una funzione scalare $\Phi(\r)$, il suo laplaciano è definito come
	\begin{equation}
		\nabla^2 \Phi = \frac{\partial^2 \Phi}{\partial x^2} + \frac{\partial^2 \Phi}{\partial y^2} + \frac{\partial^2 \Phi}{\partial z^2}
	\end{equation}
\end{description}

\section{Teoremi notevoli}
Mostreremo ora gli enunciati di due teoremi che verranno utilizzati per risolvere le equazioni di Maxwell: il teorema di Stokes (o \emph{del rotore}) e il teorema di Gauss (o \emph{del flusso}).

\subsection{Teorema di Stokes}
Data $\vec{l}$ una circuitazione, $S_l$ la superficie della circuitazione, $\hat{n}$ il versore normale alla superficie	 e $\vec{A}$ il vettore del campo attraverso la superficie $S_l$, si ha:
\begin{equation}
	\oint_{\vec{l}} \vec{A} \cdot \de \vec{l} = \int_{S_l}\rot \vec{A} \cdot \hat{n} \de S_l
\end{equation}

\subsection{Teorema di Gauss (del flusso)}
Dato $V$ un volume, $\vec{A}$ un campo e $S$ la superficie del volume e $\hat{n}$
il versore normale alla superficie, si ha:
\begin{equation}
	\int_V \diverg \vec{A} \de V = \int_S \vec{A}\cdot \hat{n} \de S
\end{equation}

\section{Equazioni di Maxwell}
Si elencano ora le quattro equazioni di Maxwell nel dominio del tempo e si ricaveranno le due equazioni ``alla divergenza'' dalle due ``al rotore''.

\begin{equation}\begin{dcases}
	\rot \ert = -\deriv{\brt}{t} \\
	\rot \hrt = \jt(\r,t) + \deriv{\drt}{t} \\
	\diverg \d(\r,t) = \rho \\
	\diverg \b(\r,t) = 0
\end{dcases}\label{eq:maxwell}\end{equation}

I vettori $\e$, $\h$, $\d$, $\b$, $\jt$ sono tutti rappresentabili nella forma
$$\vec{v}(\r,t) = \left(v_x(\r,t), v_y(\r,t), v_z(\r,t)\right)$$
e sono definiti dal punto di vista fisico come segue
\begin{itemize}
	\item $\ert$ è il vettore campo elettrico $\left[\frac{V}{m}\right]$
	\item $\brt$ è il vettore induzione magnetica $[T]$
	\item $\hrt$ è il vettore campo magnetico $[A \cdot m]$
	\item $\drt$ è il vettore spostamento elettrico $\left[\frac{C}{m^2}\right]$
	\item $\jt(\r, t)$ è il vettore densità totale di corrente elettrica $\left[\frac{A}{m^2}\right]$
\end{itemize}

\subsubsection{Identità vettoriale e ultime due equazioni di Maxwell} \label{sec:id-not}
L'identità vettoriale $\diverg (\rot \vec{A}) = 0$ permette di ricavare le equazioni alle divergenze dalle equazioni ai rotori.

La seconda si ricava immediatamente come
\begin{equation}\begin{split}
	& \diverg (\rot \ert) = \diverg \left(-\deriv{\brt}{t}\right)=0 \\
	& \implies \deriv{\diverg \b}{t}=0 \implies \diverg \b \text{ è costante nel tempo.}\\
\end{split}\end{equation}

Poiché non esiste il monopolo magnetico, si ottiene che $\diverg\b =0$.

Per la prima equazione il passaggio è invece
\begin{esp}
	\diverg(\rot \h) &= \diverg \jt + \deriv{\diverg \d}{t}=0 \\
\end{esp}

Per ricavare l'equazione sulla divergenza di $\d$ sfruttiamo, insieme al teorema sul flusso di $\jt$, la \emph{legge di conservazione della carica elettrica}.

\begin{esp}
	& \begin{dcases}
		\int_{S_v}\jt \cdot \hat{n} \de S_v &= -\deriv{Q_v}{t} = -\deriv{}{t} \int_V \rho \cdot \de V = \int_V \deriv{\rho}{t} \de V \\
		\int_{S_v}\jt \cdot \hat{n} \de S_v &= \int_V \diverg \jt \de V = - \int_V \deriv{\rho}{t} \de V
	\end{dcases} \\
	& \implies \diverg \jt = -\deriv{\rho}{t} \\
\end{esp}

E dunque,
\begin{esp}
	\deriv{}{t} \left(\diverg \d - \rho \right) &=0 \implies
	\begin{cases}
		\diverg \d - \rho & \text{è costante nel tempo} \\
		\diverg \d = \rho & \text{si riconosce come la legge di Gauss}
	\end{cases}
\end{esp}

\subsubsection{Leggi del legame materiale}
La risoluzione delle equazioni di Maxwell porta ad aver 6 equazioni, derivanti dalle prime due equazioni (vettoriali) di Maxwell in 15 incognite ($\b , \e , \h , \d , \jt $ tutte $\in \R^3$).

Si può ridurre il numero delle incognite attraverso le \emph{relazioni costitutive} o \emph{leggi del legame materiale}.

Viene prima di tutto scomposta la densità di corrente totale in due componenti: la prima è la corrente autoindotta (incognita) mentre la seconda è la corrente impressa dai generatori, nota.
\begin{equation}
	\jt = \vec{j}_{COND} + \vec{j}
\end{equation}

Nel vuoto si ha inoltre che
\begin{equation}
	\begin{cases}
		\b = \mu_0 \cdot \h \\
		\d = \epsilon_0 \e \\
		\vec{j}_{COND} = 0 \implies \jt \text{ è nota} \\
	\end{cases}
	\label{eq:bh-de-j}
\end{equation}
dove
\begin{itemize}
	\item $\mu_0 = 4\cdot \pi \cdot 10^{-7} \left[\frac{H}{m}\right]$ è la costante di \emph{permeabilità
	magnetica del vuoto}
	\item $\epsilon_0 = 8.85 \cdot 10^{-12} \left[\frac{F}{m}\right]$ è la costante di \emph{permermittività
	dielettrica del vuoto}
\end{itemize}

\subsubsection{Densità di corrente impressa}
Si definisce $V_s$ come il volume delle sorgenti, dove tutta la corrente è confinata.
\begin{equation} \vec{j}(\r,t)
	\begin{cases}
		\neq 0 & \forall \r \in V_s \\
		=0 & \text{altrimenti}
	\end{cases}
	\label{eq:correnti}
\end{equation}

Grazie alle condizioni descritte in \eqref{eq:bh-de-j} e \eqref{eq:correnti}, nel vuoto
le equazioni di Maxwell si possono riscrivere come
\begin{equation}\begin{dcases}
	\rot \ert = -\mu_0 \deriv{\hrt}{t} \\
	\rot \hrt = \epsilon_0 \deriv{\ert}{t}\\
	\diverg \hrt = 0 \\ \diverg\ert = \frac{\rho}{\epsilon_0}
\end{dcases}\label{eq:maxwell-vuoto}\end{equation}

Le incognite $\ert, \hrt$ sono quindi le uniche incognite del \emph{campo elettromagnetico}.
Si può notare che se uno dei campi è statico la sua derivata si annulla e perciò l'andamento dell'altro ne risulta indipendente.

\smallbreak
\textsc{Nota } Per semplificare la notazione, d'ora in poi si ometterà la variabile spaziale $\vec{r}$ e temporale $t$ dall'espressioni dei vettori considerati.
\smallbreak

Applicando il rotore alle prime due equazioni di Maxwell nel vuoto, si ha che
\begin{equation}\begin{split}
	\rot\left(\rot\e\right) &= -\mu_0 \deriv{}{t} \left(\rot\h\right) = -\mu_0\epsilon_0 \deriv[2]{\e}{t} \\
	\rot\left(\rot\h\right) &= -\epsilon_0 \deriv{}{t} \left(\rot\e\right) = -\mu_0\epsilon_0 \deriv[2]{\h}{t} \\
\end{split}\end{equation}

Sfruttando l'identità vettoriale $\rot\rot\vec{A} = -\nabla^2 + \nabla(\diverg\vec{A})$ si ottiene
\begin{equation}\begin{split}
	-\nabla^2\e + \nabla(\diverg\e) &= - \mu_0\epsilon_0 \deriv[2]{\e}{t} ~~ \Rightarrow ~~ \nabla^2\e = \mu_0\epsilon_0\deriv[2]{\e}{t}\\
	-\nabla^2\h + \nabla(\diverg\b) &= - \mu_0\epsilon_0 \deriv[2]{\h}{t} ~~ \Rightarrow ~~ \nabla^2\h = \mu_0\epsilon_0\deriv[2]{\h}{t}\\
\end{split}\end{equation}
Dove la prima equazione suppone che non ci siano cariche superficiali ($\rho=0$).
Abbiamo quindi ottenuto sei equazioni in sei incognite, che conducono a un sistema risolvibile.

\paragraph{Analisi dimensionale}

Per semplicità, consideriamo solo la componente lungo l'asse $x$ del campo elettrico
\begin{equation}
	\left[\nabla^2 e_x\right] = \left[\mu_0\epsilon_0\right]	\cdot \left[\deriv[2]{e_x}{t}\right] \\
	\frac{[e]}{m^2} = [\mu_0\epsilon_0] \cdot \frac{[e]}{s^2} \implies \frac{1}{\sqrt{[\mu_0\epsilon_0]}} = \frac{m}{s}
\end{equation}
Questa quantità è omogenea ad una velocità: dalla teoria delle onde emerge infatti che essa è la velocità dell'onda nel vuoto. D'ora in avanti definiremo
$c_0 = 1 / \sqrt{\mu_0\epsilon_0} \simeq 3 \cdot 10^8 m/s$ come la velocità della luce nel vuoto.

A seconda che le sorgenti siano o meno nella regione di spazio considerata, possiamo inoltre discriminare il campo elettromagnetico in

\begin{description}
	\item [campo di propagazione] se $\vec{J_s}=\vec{0}$
	\item [campo di radiazione] se $\vec{J_s}\neq\vec{0}$
\end{description}

\section{Campo in spazio non vuoto}
Analizziamo ora la propagazione nel mezzo, supponendo che la legge che lega $\b$ ad $\h$ e $\d$ ad $\e$ sia analoga a quella nel vuoto, ovvero $\b=\mu\cdot\h$ e $\d=\epsilon\cdot\e$ con
\begin{itemize}
	\item $\mu$ è la costante di \emph{permeabilità	 magnetica del mezzo}
	\item $\epsilon$ è la costante di \emph{permermittività	 dielettrica del mezzo}
\end{itemize}
Supponiamo inoltre che le correnti impresse siano $\vec{j}_i$ e che le correnti di conduzione rispettino la legge di Ohm, per cui $\vec{J}_{COND} = \sigma ~ \e$, con $\sigma$ la \emph{costante di conducibilità elettrica del mezzo}.

Elenchiamo ora le \emph{proprietà del mezzo} che andremo a studiare, considerandole \emph{localmente} vere per i mezzi reali che incontreremo.
\begin{itemize}
	\item \textbf{linearità}
	\item \textbf{isotropia}
	\item \textbf{omogeneità} ossia $\epsilon, \mu, \sigma$ rimangono costanti nello spazio considerato
	\item \textbf{tempo-invarianza} ossia le costanti del materiale non variano al variare del tempo
	\item \textbf{non dispersività} per cui le costanti di propagazione non variano in frequenza
\end{itemize}

D'ora in poi considereremo solo mezzi \emph{dielettrici}, mezzi per cui $\mu \simeq \mu_0$, in cui quindi le equazioni di Maxwell saranno analoghe a quelle del vuoto, a meno delle costanti $\epsilon, \mu_0$ e $\sigma$.

\section{Studio delle equazioni di Maxwell in regime sinusoidale}
Studieremo ora le proprietà del campo elettromagnetico nel caso in cui l'onda sia di tipo sinusoidale: introdurremo i vettori di \emph{Steinmetz} e vedremo come un mezzo possa variare la sua attitudine al campo alle diverse frequenze.

Ci poniamo in condizioni di banda stretta, supponendo perciò che le trasmissioni avvengano su una banda di frequenze $\Delta f$ centrata attorno ad una portante $f$: l'analisi limitata alla sola frequenza $f$ è una buona approsimazione dell'onda, a patto che essa sia stretta ($\Delta f \ll f$).
I segnali a banda larga verranno poi approssimati come sovrapposizione di molti segnali a banda stretta, linearità del mezzo permettendo.

Il campo elettrico ora può essere scritto come
\begin{esp}
	\ert &= \sum\limits_{n=1}^3 \hat{x}_n^{\prime\prime} \cdot E_n(\r)\cdot \cos\left[\omega t + \Phi_{E_n}(\r)\right]\\
\end{esp}
con
\begin{itemize}
	\item $E_n (\r) \in \R$ pari al massimo del campo nella componente $n$
	\item $\Phi_{E_{n}} \in [0;2\pi]\subset \R$ pari alla fase del campo nella componente $n$
\end{itemize}

Date queste ipotesi, introduciamo ora i \emph{vettori complessi}, o di \emph{Steinmetz}, quali estensione tridimensionale del concetto di fasore complesso: utilizzeremo i valori massimi (invece che valori efficaci, usati durante il corso di elettrotecnica).

\subsubsection{Fasore sulla linea di trasmissione}

La tensione lungo la linea di trasmissione che lavora a pulsazione $\omega$ si può descrivere con il fasore $V \in \mathbb{C}^3$.
\begin{equation}
	v(t,x) = |V| \cos(\omega t + \Phi_0) = \Re\left[V \cdot e^{\jmath \omega t}\right]
\end{equation}

Per un certo $\E(\r) = \left(E_x(\r);E_y(\r);E_z(\r)\right) \in \C^3$ potremo scrivere in modo analogo $\ert = \Re\left[\E(\r)\cdot e^{\jmath \omega t}\right]$.

Avremo quindi
\begin{equation*}
	\E(\r) = \sum\limits_{n=1}^3 \hat{x}_n^{\prime\prime} E_n(\r)\cdot e^{\jmath \Phi_{E_n}(\r)}
\end{equation*}
\begin{equation*}\begin{split}
	\ert &= \Re\left[\sum\limits_{n=1}^3 \hat{x}_n^{\prime\prime} E_n(\r)\cdot e^{\jmath \Phi_{E_n}(\r)}\cdot e^{\jmath \omega t}\right]\\
	&=\sum\limits_{n=1}^3 \hat{x}_n^{\prime\prime} E_n(\r)\cdot \Re\left[e^{\jmath \Phi_{E_n}(\r)}\cdot e^{\jmath \omega t}\right]\\
	&=\sum\limits_{n=1}^3 \hat{x}_n^{\prime\prime} E_n(\r)\cdot cos[\omega t + \Phi_{E_n}(\r)]
\end{split}\end{equation*}
Allo stesso modo si ottengono le uguaglianze per il campo magnetico e per le correnti.

Cerchiamo ora di scrivere le equazioni di Maxwell utilizzando i vettori di Steinmetz. Data la trasformazione presentata in precedenza, ovvero
\begin{esp}
	\rot \e &= \rot \Re\left[\E(\r) \cdot e^{\jmath\omega t}\right] = \Re\left[\rot\E(\r) \cdot e^{\jmath\omega t}\right]\\
\end{esp}

si può ricavare che
\begin{esp}
	\rot\e &\leftrightarrow \rot\E \\
	\diverg\e &\leftrightarrow \diverg\E \\
	\deriv{\e}{t} & \leftrightarrow \jmath \omega \E\quad \text{(trasformata di Fourier)} \\
\end{esp}

\subsection{Equazioni di Maxwell in regime sinusoidale}
Le equazioni di Maxwell, nel caso del regime sinusoidale si possono riscrivere come
\begin{equation}\begin{dcases}
	\rot\E(\r) = -\jmath \omega \mu\H(\r) \\
	\begin{aligned}[b]
		\rot\H(\r) &= \jmath	\omega \epsilon \E(\r) + \J(\r) + \sigma\E \\
		&= \J(\r) +(\sigma + \jmath \omega \epsilon) \cdot \E(\r) \\
		&= \J(\r) +\jmath \omega \epsilon_c \cdot \E(\r) \\
	\end{aligned} \\
	\diverg\H(\r)=0\\
	\diverg \E(\r) = \frac{\rho_T(\r)}{\epsilon}
\end{dcases}\end{equation}
dove $\rho$ varia in regime sinusoidale, come le altre grandezze fisiche in gioco, e $\epsilon_c = \epsilon -\jmath\sigma / \omega$ è detta \emph{costante di permittività dielettrica complessa}.

Andiamo ad analizzare ora i diversi tipi di mezzi al variare di $\epsilon_c$.

\begin{definition}[Angolo di perdita]
	Considerando soltanto materiali dielettrici, vale che $\Re(\epsilon_c) = \epsilon$, $\Im(\epsilon_c) = -\frac{\sigma}{\omega}$.

	Sia $\J_{COND} = \sigma \E$ è invece la densità di corrente di conduzione e $\J_{SPOST}$ la densità di corrente di spostamento.
	Definiamo quindi l'angolo di perdita come
	\begin{equation}
		\delta = \arctg-\frac{\Im(\epsilon_c)}{\Re(\epsilon_c)} = \arctg\frac{\sigma}{\omega \epsilon}
		\end{equation}
\end{definition}

Nella seguente tabella definiamo i diversi materiali a seconda di $\sigma$ e $\delta$.
\begin{table}[h] \label{tab:conductors}
\centering
\begin{tabular}{llll}
	$\sigma$ & $\delta$ & tipo di materiale \\
	\hline \\
	0 & 0 & isolante perfetto (come il vuoto) \\
	$\approx 0$ & $\approx 0$ & buon isolante (come l'aria) \\
	$\gtreqless \omega \epsilon $ & & conduttore \\
	$\gg \omega \epsilon$ & $\approx \frac{\pi}{2}$ & buon conduttore (tra cui i metalli) \\
	$\approx +\infty$ & $\frac{\pi}{2}$ & conduttore perfetto o ideale (PEC)
\end{tabular}
\end{table}

%%% Local Variables:
%%% mode: latex
%%% TeX-master: "antenne"
%%% End:
