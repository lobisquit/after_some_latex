\chapter{Introduzione}
\section{Sistemi di riferimento}
Nel corso verranno utilizzate principalmente due sistemi di riferimento: quello
con coordinate \textbf{cartesiane} e quello con coordinate \textbf{sferiche}.
Le rappresentazioni di un punto \textit{\textbf{P}} nelle due coordinate saranno le seguenti:
\begin{equation}
  P=(x,y,z) = \r = x \cdot \hat{x} + y \cdot \hat{y} +z \cdot \hat{z}  \quad \r\text{ è il vettore posizione} \\
  x \bot y \bot z \bot x \\
  P=(r,\theta,\phi) = r \cdot \hat{r} = r \cdot \hat{r}(\theta,\phi)\\
  r \bot \theta \bot \phi \bot r
  \label{eq:riferimento}
\end{equation}

\section{Operatori differenziali (spaziali)}
Elenchiamo ora gli operatori che utilizzeremo, per poi definirli meglio in delle loro sottosezioni.
\begin{description}
  \item[Operatore Nabla] $ \nabla = \left(\frac{\partial}{\partial x},\frac{\partial}{\partial y},\frac{\partial}{\partial z}\right)$
  \item[Rotore] $\rot$
  \item[Divergenza] $\diverg$
  \item[Gradiente] $\nabla$
  \item[Laplaciano] $\nabla^2$
\end{description}
Iniziamo a descrivere cosa succede a un generico vettore $\vec{A\left(\r\right)}=\left(A_x(\r),A_y(\r),A_z(\r)\right)$.
\subsection{Rotore}
\begin{equation} \begin{split}
  \rot \vec{A(\r)} &=
  \begin{vmatrix}
    \hat{x} &\hat{y} &\hat{z} \\
    \frac{\partial}{\partial x} & \frac{\partial}{\partial y} & \frac{\partial}{\partial z} \\
    A_x & A_y & A_z
  \end{vmatrix} \\
  &= \left(\frac{\partial A_z}{\partial y} - \frac{\partial A_y}{\partial z}\right)\cdot \hat{x} -
  \left(\frac{\partial A_z}{\partial x} - \frac{\partial A_x}{\partial z}\right)\cdot \hat{y} +
  \left(\frac{\partial A_y}{\partial x} - \frac{\partial A_x}{\partial y}\right)\cdot \hat{z}
\end{split}\end{equation}

\subsection{Divergenza}
\begin{equation}
  \diverg \vec{A} = \frac{\partial A_x}{\partial x} + \frac{\partial A_y}{\partial y} + \frac{\partial A_z}{\partial z}
\end{equation}
\subsection{Gradiente}
Dato $\Phi(\r)$ uno scalare, il suo gradiente risulta
\begin{equation}
  \nabla \Phi = \left(\frac{\partial \Phi}{\partial x} ; \frac{\partial \Phi}{\partial y}; \frac{\partial \Phi}{\partial z}\right)
\end{equation}
\subsection{Laplaciano}
Dato $\Phi(\r)$ uno scalare, il suo laplaciano risulta
\begin{equation}
  \nabla^2 \Phi = \left(\frac{\partial^2 \Phi}{\partial x^2} + \frac{\partial^2 \Phi}{\partial y^2} + \frac{\partial^2 \Phi}{\partial z^2}\right)
\end{equation}

\section{Teoremi importanti}
Mostreremo ora due teoremi che verranno utilizzati per risolvere le equazioni di Maxwell:
il teorema di \textbf{Stokes} (o \textit{del rotore}) e il teorema di \textbf{Gauss} (del flusso).
Si mostrerà solo il risultato, senza enunciarlo nè dimostrarlo.
\subsection{Teorema di Stokes}
Data $\vec{l}$ una circuitazione, $S_l$ la superficie della circuitazione, $\hat{n}$
 il versore normale alla superficie  e $\vec{A}$ il vettore del campo attraverso la superficie $S_l$, si ha:
\begin{equation}
  \oint_{\vec{l}} \vec{A} \cdot d\vec{l} = \int_{S_l}\rot \vec{A} \cdot \hat{n} d S_l
\end{equation}

\subsection{Teorema di Gauss (del flusso)}
Dato $V$ un volume, $\vec{A}$ un campo e $S$ la superficie del volume e $\hat{n}$
il versore normale alla superficie, si ha:
\begin{equation}
  \int_V \diverg \vec{A} dV = \int_S \vec{A}\cdot \hat{n} dS
\end{equation}

\section{Equazioni di Maxwell}
Si elencano ora le quattro equazioni di Maxwell nel dominio del tempo e, dall'identità
vettoriale della divergenza del rotore, si ricaveranno le ultime due equazioni dalle prime due.

\begin{equation}\begin{cases}
  \rot \ert = -\deriv{\brt}{t} \\
  \rot \hrt = \Jt(\r,t) + \deriv{\drt}{t} \\
  \diverg \d = \rho \\
  \diverg \b = 0
\end{cases}\label{eq:maxwell}\end{equation}

I vettori $\e$, $\h$, $\d$, $\b$ sono tutti rappresentabili nella forma
$$\vec{v}(\r,t) = \left(v_x(\r,t);v_y(\r,t);v_z(\r,t)\right)$$
Definiamo ora cosa rappresentano i vettori sopra elencati:
\begin{description}
  \item[$\ert$ ] è il vettore campo elettrico $\left[\frac{V}{m}\right]$
  \item[$\brt$] è il vettore induzione magnetica $[T]$
  \item[$\hrt$] è il vettore campo magnetico $[A \cdot m]$
  \item[$\drt$] è il vettore spostamento elettrico $\left[\frac{C}{m^2}\right]$
  \item[$\Jt$] è il vettore densità totale di corrente elettrica $\left[\frac{A}{m^2}\right]$
\end{description}

\subsubsection{Identità vettoriale e ultime due equazioni di Maxwell}
L'identità vettoriale $\diverg (\rot \vec{A}) = 0$ permette di ricavare
dalle equazioni ai rotori, le equazioni alle divergenze. In particolare si trova che:
\begin{equation}\begin{split}
  \diverg (\rot \ert) &= \diverg \left(-\deriv{\brt}{t}\right)=0 \quad \Leftrightarrow \deriv{\diverg \b}{t}=0\\
  \implies & \diverg \b \text{ è costante nel tempo.}\\
  \diverg(\rot \h) &= \diverg \Jt + \deriv{\diverg \d}{t}=0
\end{split}\end{equation}

Poiché non esiste il monopolo magnetico, si ottiene che $\diverg\b =0$.\\
Per valutare il risultato della divergenza del rotore del campo magnetico, abbiamo bisogno della
\textbf{legge di conservazione della carica elettrica}, che porta come risultato
\begin{equation}
  
\end{equation}
