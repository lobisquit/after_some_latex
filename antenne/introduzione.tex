\chapter{Introduzione}
\section{Sistemi di riferimento}
Nel corso verranno utilizzate principalmente due sistemi di riferimento: quello
con coordinate \textbf{cartesiane} e quello con coordinate \textbf{sferiche}.
Le rappresentazioni di un punto \textit{\textbf{P}} nelle due coordinate saranno le seguenti:
\begin{equation}
  P=(x,y,z) = \r = x \cdot \hat{x} + y \cdot \hat{y} +z \cdot \hat{z}  \quad \r\text{ è il vettore posizione} \\
  x \bot y \bot z \bot x \\
  P=(r,\theta,\phi) = r \cdot \hat{r} = r \cdot \hat{r}(\theta,\phi)\\
  r \bot \theta \bot \phi \bot r
  \label{eq:riferimento}
\end{equation}

\section{Operatori differenziali (spaziali)}
Elenchiamo ora gli operatori che utilizzeremo, per poi definirli meglio in delle loro sottosezioni.
\begin{itemize}
  \item \textbf{Operatore Nabla} $ \nabla = \left(\frac{\partial}{\partial x},\frac{\partial}{\partial y},\frac{\partial}{\partial z}\right)$
  \item \textbf{Rotore} $\rot$
  \item \textbf{Divergenza} $\diverg$
  \item \textbf{Gradiente} $\nabla$
  \item \textbf{Laplaciano} $\nabla^2$
\end{itemize}
Iniziamo a descrivere cosa succede a un generico vettore $\vec{A}\left(\r\right)=\left(A_x(\r),A_y(\r),A_z(\r)\right)$.
\subsection{Rotore}
\begin{equation} \begin{split}
  \rot \vec{A(\r)} &=
  \begin{vmatrix}
    \hat{x} &\hat{y} &\hat{z} \\
    \frac{\partial}{\partial x} & \frac{\partial}{\partial y} & \frac{\partial}{\partial z} \\
    A_x & A_y & A_z
  \end{vmatrix} \\
  &= \left(\frac{\partial A_z}{\partial y} - \frac{\partial A_y}{\partial z}\right)\cdot \hat{x} -
  \left(\frac{\partial A_z}{\partial x} - \frac{\partial A_x}{\partial z}\right)\cdot \hat{y} +
  \left(\frac{\partial A_y}{\partial x} - \frac{\partial A_x}{\partial y}\right)\cdot \hat{z}
\end{split}\end{equation}

\subsection{Divergenza}
\begin{equation}
  \diverg \vec{A} = \frac{\partial A_x}{\partial x} + \frac{\partial A_y}{\partial y} + \frac{\partial A_z}{\partial z}
\end{equation}
\subsection{Gradiente}
Dato $\Phi(\r)$ uno scalare, il suo gradiente risulta
\begin{equation}
  \nabla \Phi = \left(\frac{\partial \Phi}{\partial x} ; \frac{\partial \Phi}{\partial y}; \frac{\partial \Phi}{\partial z}\right)
\end{equation}
\subsection{Laplaciano}
Dato $\Phi(\r)$ uno scalare, il suo laplaciano risulta
\begin{equation}
  \nabla^2 \Phi = \left(\frac{\partial^2 \Phi}{\partial x^2} + \frac{\partial^2 \Phi}{\partial y^2} + \frac{\partial^2 \Phi}{\partial z^2}\right)
\end{equation}

\section{Teoremi importanti}
Mostreremo ora due teoremi che verranno utilizzati per risolvere le equazioni di Maxwell:
il teorema di \textbf{Stokes} (o \textit{del rotore}) e il teorema di \textbf{Gauss} (del flusso).
Si mostrerà solo il risultato, senza enunciarlo nè dimostrarlo.
\subsection{Teorema di Stokes}
Data $\vec{l}$ una circuitazione, $S_l$ la superficie della circuitazione, $\hat{n}$
 il versore normale alla superficie  e $\vec{A}$ il vettore del campo attraverso la superficie $S_l$, si ha:
\begin{equation}
  \oint_{\vec{l}} \vec{A} \cdot d\vec{l} = \int_{S_l}\rot \vec{A} \cdot \hat{n} d S_l
\end{equation}

\subsection{Teorema di Gauss (del flusso)}
Dato $V$ un volume, $\vec{A}$ un campo e $S$ la superficie del volume e $\hat{n}$
il versore normale alla superficie, si ha:
\begin{equation}
  \int_V \diverg \vec{A} dV = \int_S \vec{A}\cdot \hat{n} dS
\end{equation}

\section{Equazioni di Maxwell}
Si elencano ora le quattro equazioni di Maxwell nel dominio del tempo e, dall'identità
vettoriale della divergenza del rotore, si ricaveranno le ultime due equazioni dalle prime due.

\begin{equation}\begin{cases}
  \rot \ert = -\deriv{\brt}{t} \\
  \rot \hrt = \jt(\r,t) + \deriv{\drt}{t} \\
  \diverg \d = \rho \\
  \diverg \b = 0
\end{cases}\label{eq:maxwell}\end{equation}

I vettori $\e$, $\h$, $\d$, $\b$ sono tutti rappresentabili nella forma
$$\vec{v}(\r,t) = \left(v_x(\r,t);v_y(\r,t);v_z(\r,t)\right)$$
Definiamo ora cosa rappresentano i vettori sopra elencati:
\begin{itemize}
  \item \textbf{$\ert$]}è il vettore campo elettrico $\left[\frac{V}{m}\right]$
  \item \textbf{$\brt$} è il vettore induzione magnetica $[T]$
  \item \textbf{$\hrt$} è il vettore campo magnetico $[A \cdot m]$
  \item \textbf{$\drt$} è il vettore spostamento elettrico $\left[\frac{C}{m^2}\right]$
  \item \textbf{$\jt$} è il vettore densità totale di corrente elettrica $\left[\frac{A}{m^2}\right]$
\end{itemize}

\subsubsection{Identità vettoriale e ultime due equazioni di Maxwell}
L'identità vettoriale $\diverg (\rot \vec{A}) = 0$ permette di ricavare
dalle equazioni ai rotori, le equazioni alle divergenze. In particolare si trova che:
\begin{equation}\begin{split}
  \diverg (\rot \ert) &= \diverg \left(-\deriv{\brt}{t}\right)=0 \quad \Leftrightarrow \deriv{\diverg \b}{t}=0\\
  \implies & \diverg \b \text{ è costante nel tempo.}\\
  \diverg(\rot \h) &= \diverg \jt + \deriv{\diverg \d}{t}=0
\end{split}\end{equation}

Poiché non esiste il monopolo magnetico, si ottiene che $\diverg\b =0$.\\
Per valutare il risultato della divergenza del rotore del campo magnetico, abbiamo bisogno della
\textbf{legge di conservazione della carica elettrica}, che porta come risultato
\begin{equation}\begin{split}
  \int_{S_v}\jt \cdot \hat{n} d S_v &= -\deriv{Q_v}{t} = -\deriv{}{t} \int_V \rho(\r,t) \cdot dV = \int \deriv{\rho(\r,t)}{t} dV\\
  \int_V \diverg \jt dV &= - \int \deriv{\rho}{t} dV \Leftrightarrow \diverg \jt = -\deriv{\rho}{t} \\
  \deriv{}{t} \left(\diverg \d - \rho \right) &=0 \implies
  \begin{cases}
    \diverg \d - \rho & \text{costante nel tempo} \\
    \diverg \d = \rho & \text{legge di Gauss}
  \end{cases}
\end{split}\end{equation}
La risoluzione delle equazioni di Maxwell porta ad aver 6 equazioni, derivanti
dalle prime due equazioni (vettoriali) di Maxwell in 15 incognite ($\b , \e , \h , \d , \jt $ tutte $\in \R^3$)

Si possono ridurre le incognite attraverso le \textbf{relazioni costitutive} o
\textbf{leggi del legame materiale}. In particolare si riesce ad ottenere
\begin{equation}
  \jt = \vec{j}_{COND} + \vec{j}
\end{equation}
dove la prima è la corrente autoindotta (incognita) e la seconda è la corrente impressa, nota.

Nel vuoto si ha inoltre che:
\begin{equation}
  \b = \mu_0 \cdot \h \quad \d = \epsilon_0 \e \\
  \vec{j}_{COND} = 0 \implies \jt \text{ è nota} \\
  \label{eq:bh-de-j}
\end{equation}
dove
\begin{itemize}
  \item $\mu_0 = 4\cdot \pi \cdot 10^{-7} \left[\frac{H}{m}\right]$ è la costante di \textbf{permeabilità
  magnetica del vuoto}
  \item $\epsilon_0 = 8.85 \cdot 10^{-12} \left[\frac{F}{m}\right]$ è la costante di \textbf{permermittività
  dielettrica del vuoto}
\end{itemize}

\textsc{Densità di corrente impressa}
Dato $V_s$ il volume delle sorgenti, si ha che la densità di corrente elettrica impressa è:
\begin{equation} \vec{j}(\r,t)=
  \begin{cases}
    \neq 0 & \forall \r \in V_s \\
    =0 & \text{altrimenti}
  \end{cases}
  \label{eq:correnti}
\end{equation}

Grazie alle condizioni descritte in \eqref{eq:bh-de-j} e \eqref{eq:correnti}, nel vuoto
le equazioni di Maxwell si possono riscrivere come
\begin{equation}\begin{cases}
  \rot \ert = -\mu_0 \deriv{\hrt}{t} \\
  \rot \hrt = \epsilon_0 \deriv{\ert}{t}\\
  \diverg \hrt = 0 \\ \diverg\ert = \frac{\rho}{\epsilon_0}
\end{cases}\label{eq:maxwell-vuoto}\end{equation}

Le incognite $\{\ert, \hrt\}$ sono le incognite del \textbf{campo elettromagnetico}.
Inoltre si ha che se uno dei due campi è statico, l'altro risulta indipendente.

\textsc{Nota:} d'ora in poi si scriverà $\e$ invece che $\ert$ per semplicità.
Applicando ora il rotore alle prime due equazioni di Maxwell nel vuoto, si ha:
\begin{equation}\begin{split}
  \rot\left(\rot\e\right) &= -\mu_0 \deriv{}{t} \left(\rot\h\right) = -\mu_0\epsilon_0 \deriv[2]{\e}{t} \\
  \rot\left(\rot\h\right) &= -\epsilon_0 \deriv{}{t} \left(\rot\e\right) = -\mu_0\epsilon_0 \deriv[2]{\h}{t} \\
\end{split}\end{equation}

Sfruttando l'identità vettoriale $\rot\rot\vec{A} = -\nabla^2 + \nabla(\diverg\vec{A})$ si ottiene
\begin{equation}\begin{split}
  -\nabla^2\e + \nabla(\diverg\e) &= - \mu_0\epsilon_0 \deriv[2]{\e}{t} \Leftrightarrow \nabla^2\e = \mu_0\epsilon_0\deriv[2]{\e}{t}\\
  -\nabla^2\h + \nabla(\diverg\b) &= - \mu_0\epsilon_0 \deriv[2]{\h}{t} \Leftrightarrow \nabla^2\h = \mu_0\epsilon_0\deriv[2]{\h}{t}\\
\end{split}\end{equation}
Dove la prima equazione suppone che non ci siano cariche superficiali ($\rho=0$)
Abbiamo quindi ottenuto sei equazioni in sei incognite, che conducono a un sistema risolvibile.

\textbf{\textsc{Analisi dimensionale}}\\
Consideriamo solo la componente lungo l'asse x del campo elettrico
\begin{equation}
  \left[\nabla^2 e_x\right] = \left[\mu_0\epsilon_0\right]  \cdot \left[\deriv[2]{e_x}{t}\right] \\
  \frac{[e]}{m^2} = [\mu_0\epsilon_0] \cdot \frac{[e]}{s^2} \implies \frac{1}{\sqrt{[\mu_0\epsilon_0]}} = \frac{m}{s}
\end{equation}
Abbiamo quindi scoperto la velocità del moto ondoso nel vuoto. D'ora in avanti definiremo
$c_0 = \frac{1}{\sqrt{\mu_0\epsilon_0}}\stackrel{\approx}{conti} 3 \cdot 10^8 m/s$.
Possiamo inoltre definire il campo elettromagnetico come \textbf{campo di}
\begin{description}
  \item \textbf{[propagazione} se $\vec{J_s}=\vec{0}$
  \item \textbf{radiazione} se $\vec{J_s}\neq\vec{0}$
\end{description}

\section{Campo in spazio non vuoto}
Nel mezzo, supponiamo che la legge che lega $\b$ ad $\h$ e $\d$ ad $\e$ sia analoga a quella nel vuoto.
In particolare $\b=\mu\cdot\h$ e $\d=\epsilon\cdot\e$ dove
\begin{itemize}
  \item $\mu ]$ è la costante di \textbf{permeabilità  magnetica del mezzo}
  \item $\epsilon$ è la costante di \textbf{permermittività  dielettrica del mezzo}
\end{itemize}
Supponiamo inoltre che le correnti impresse siano $\vec{j_i}$ e che le correnti di conduzione siano
$\vec{J_{COND}} = \sigma \cdot \e$, definendo $\sigma$ la \textbf{costante di conducibilità elettrica
del mezzo}, affinché le correnti di conduzione rispettino la legge di Ohm.

Elenchiamo ora le \textbf{proprietà del mezzo} che andremo a studiare, considerandole vere almeno per volumi
di spazio sufficientemente piccoli.
\begin{itemize}
  \item \textbf{linearità}
  \item \textbf{isotropia}
  \item \textbf{omogeneità}  ossia $\epsilon, \mu, \sigma$ rimangono costanti nello spazio considerato
  \item \textbf{tempo-invarianza} ossia le costanti non variano al variare del tempo (temperatura costante)
  \item \textbf{non dispersività}  le costanti di propagazione non variano in frequenza
\end{itemize}

D'ora in poi considereremo \textbf{solo mezzi dielettrici}, mezzi per cui $\mu \approx \mu_0$, per cui
le equazioni di Maxwell rimarranno invariate nel mezzo e le costanti saranno $\epsilon, \mu_0$ e $\sigma$

\section{Studio delle equazioni di Maxwell in regime sinusoidale}
Studieremo ora le proprietà del campo elettromagnetico nel caso in cui l'onda sia di tipo sinusoidale:
introdurremo i vettori di \textbf{\textit{Steinmetz}} e vedremo come un mezzo possa variare la sua
attitudine al campo alle diverse frequenze.

Supponiamo quindi, nel caso generale di lavorare a pulsazione $\omega = 2\pi f$ con $[f] = Hz = s^{-1}$ frequenza.
Nonostante le trasmissioni avvengano su una banda di frequenze($\Delta f$), se consideriamo
di lavorare in banda stretta ($\Delta f \ll f$), l'analisi alla frequenza portante è una buona
approsimazione per tutta l'analisi. Nel momento in cui andremo a studiare segnali a
banda larga, li vedremo come sovrapposizione di molti segnali a banda stretta.

Il campo elettrico ora può essere scritto come:
\begin{equation}\begin{split}
  \ert &= \sum\limits_{n=1}^3 \hat{x}_n^{\prime\prime} \cdot E_n(\r)\cdot cos\left[\omega t + \Phi_{E_n}(\r)\right]\\
  \text{con}& \\
  E_n (\r) &\in \R \text{ pari al massimo del campo nella componente n} \\
  \Phi_{E_{n}} &\in [0;2\pi]\subset \R \text{ pari alla fase del campo nella componente n} \\
\end{split}\end{equation}

Introduciamo ora la rappresentazione con i \textbf{vettori complessi} o rappresentazione di \textbf{Steinmetz}
come estensione del concetto di fasore complesso: utilizzeremo i valori massimi (invece che valori efficaci,
usati durante il corso di elettrotecnica) e il fasore sarà un vettore.\
\textsc{Fasore sulla linea di trasmissione}\\
Vediamo ora come estendere il concetto di fasore al nostro campo elettromagnetico analizzando
la tensione lungo la linea di trasmissione che lavora a pulsazione $\omega$
\begin{equation}
  v(t,x) = cos(\omega t + \Phi_0) = \Re\left[V \cdot e^{\jmath \omega t}\right]
\end{equation}
Con il campo elettrico $\E(\r) = \left[E_x(\r);E_y(\r);E_z(\r)\right] \in \C^3$ possiamo
scrivere $\ert = \Re\left[\E(\r)\cdot e^{\jmath \omega t}\right]$. Avremo quindi
\begin{equation}\begin{split}
  \E(\r) &= \sum\limits_{n=1}^3 \hat{x}_n^{\prime\prime} E_n(\r)\cdot e^{\jmath \Phi_{E_n}(\r)} \\
  \ert &= \Re\left[\sum\limits_{n=1}^3 \hat{x}_n^{\prime\prime} E_n(\r)\cdot e^{\jmath \Phi_{E_n}(\r)}\cdot e^{\jmath \omega t}\right]\\
  &=\sum\limits_{n=1}^3 \hat{x}_n^{\prime\prime} E_n(\r)\cdot \Re\left[e^{\jmath \Phi_{E_n}(\r)}\cdot e^{\jmath \omega t}\right]\\
  &=\sum\limits_{n=1}^3 \hat{x}_n^{\prime\prime} E_n(\r)\cdot cos[\omega t + \Phi_{E_n}(\r)]
\end{split}\end{equation}
Analogamente si ottengono le uguaglianze per il campo magnetico e per le correnti.

Cerchiamo ora di scrivere le equazioni di Maxwell utilizzando i vettori di Steinmets:
\begin{equation}\begin{split}
  \rot \e &= \rot \Re\left[\E(\r) \cdot e^{\jmath\omega t}\right] = \Re\left[\rot\E(\r) \cdot e^{\jmath\omega t}\right]\\
  &\text{otteniamo quindi che:}\\
  \rot\e &\rightarrow \rot\E \quad \text{corrispondenza biunivoca} \\
  \diverg\e &\rightarrow \diverg\E \quad \text{corrispondenza biunivoca} \\
  \deriv{\e}{t} & \rightarrow \jmath \omega \E\quad \text{trasformata di Fourier} \\
\end{split}\end{equation}

\subsection{Equazioni di Maxwell in regime sinusoidale}
Le equazioni di Maxwell, nel caso del regime sinusoidale si possono riscrivere come
\begin{equation}\begin{cases}
  \rot\E(\r) = -\jmath \omega \mu\H(\r) \\
  \rot\H(\r) = \jmath  \omega \epsilon \E(\r) + \J(\r) + \sigma\E = \J(\r) +(\sigma + \jmath \omega \epsilon) \cdot \E(\r) = \J(\r) +\jmath \omega \epsilon_c \cdot \E(\r) \\
  \diverg\H(\r)=0\\
  \diverg \E(\r) = \frac{\rho_T(\r)}{\epsilon} & \rho \text{ varia in regime sinusoidale}
\end{cases}\end{equation}
Con $\epsilon_c = \epsilon + \frac{-\jmath\sigma}{\omega}$ \textbf{costante di permittività
dielettrica complessa}

Andiamo ad analizzare ora i diversi tipi di mezzi al variare di $\epsilon_c$
Sapendo che $\Re(\epsilon_c) = \epsilon$ e  $\Im(\epsilon_c) = -\frac{\sigma}{\omega}$,
che $\J_{COND} = \sigma \E$ la densità di corrente di conduzione e $\J_{SPOST}$ la densità
di corrente di spostamento, definiamo
\begin{definition}[Angolo di perdita]
  Definiamo l'angolo di perdita come
  \begin{equation}
    \delta = arctg\left(-\frac{\Im(\epsilon_c)}{\Re(\epsilon_c)}\right) = arctg\left(\frac{\sigma}{\omega \epsilon}\right)
    \end{equation}
\end{definition}

Nella seguente tabella definiamo i diversi materiali a seconda di $\sigma$ e $\delta$.
\begin{table}[h]
\centering
\label{tab:conductors}
\begin{tabular}{llll}
$\sigma$                      & materiale & $\delta$                & tipo di materiale                  \\
0                             & vuoto     & 0                       & isolante perfetto                  \\
$\approx 0$                   & aria      & $\approx 0$             & buon isolante                      \\
$\gtreqless \omega \epsilon $ &           &                         & conduttore                         \\
$\gg \omega \epsilon$         & metalli   & $\approx \frac{\pi}{2}$ & buon conduttore                    \\
$\approx +\infty$             &           & $\frac{\pi}{2}$         & conduttore perfetto o ideale (PEC)
\end{tabular}
\end{table}
