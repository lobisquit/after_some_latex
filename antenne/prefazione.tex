\chapter{Prefazione}
Allo studente di \emph{Antennas and Wireless propagation}, vorremo scrivere qualche nota introduttiva su questa dispensa. Come prima cosa, la stessa si basa sugli appunti delle lezioni del corso, frequentato nell'A.A. 2016/17 quando ancora era in Italiano; questo significa che ci potrebbero essere variazioni nell'approfondimento di alcune parti e altre ancora potrebbero non essere presenti.


In questa dispensa non è presente il materiale fornito dal docente Santagiustina attraverso le slides, in particolare:
\begin{itemize}
  \item Normativa italiana in ambito di radioprotezione da campi EM
  \item Temperatura di rumore, già vista nel corso di Telecomunicazioni e presente comunque nelle slides
  \item Antenne ad apertura
  \item Seminari presenti durante il corso
\end{itemize}


Per alcuni tipi di antenne, per esempio quelle ad elica in modo assiale e normale, le Log-periodiche a dipoli, le biconiche o le bow-tie il docente ha fatto solo degli accenni, per cui si rimanda al libro \emph{Antenna Theory and Design,3\textsuperscript{rd} Ed., Warren L. Stutzman  and Gary A. Thiele, 2013, Wiley}.


Per gli esercizi, questa dispensa è stata creata con lo scopo di preparare allo studio della teoria, per cui non vi è presente alcun esercizio.
