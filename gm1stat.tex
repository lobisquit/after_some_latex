\subsection{G/M/1 queue limiting distribution}
We know that the limiting distribution is the solution of the following system of equations:
\begin{equation*}
 \begin{split}
	 \begin{cases}
		 \pi_k = \sum_i \pi_i p_{ik} \\
		 \sum_i \pi_i = 1
	 \end{cases}
 \end{split}
\end{equation*}
In our case this is equal to:
$$\pi_x = \sum_{i = k-1}^{+\infty} \pi_i \int_0^{+\infty}e^{-\mu t} \frac{(\mu t)^{i+1-k}}{(i+1-k)!}dG(t) \hspace{1cm} \text{for } k \geq 1$$
We try to verify the guessing $\pi_k=c\beta^k$ (geometric distribution).
\begin{equation*}
 \begin{split}
		 c\beta^k &= \sum_{i = k-1}^{+\infty} c\beta^i \int_0^{+\infty}e^{-\mu t} \frac{(\mu t)^{i+1-k}}{(i+1-k)!}dG(t) \\
		 &= c \int_0^{+\infty}e^{-\mu t}\beta^{k-1}\sum_{i = k-1}^{+\infty}\frac{(\beta \mu t)^{i+1-k}}{(i+1-k)!}dG(t) \\
		 &= c \int_0^{+\infty}e^{-\mu t}\beta^{k-1}e^{\beta \mu t}dG(t)
 \end{split}
\end{equation*}
Therefore $\beta^k = \int_0^{+\infty}e^{-\mu(1-\beta)t}\beta^{k-1}dG(t)$\\
and
\begin{equation*}
 \begin{split}
		  \beta &= \int_0^{+\infty}e^{-\mu(1-\beta)t}dG(t)\\
			&= \E[e^{-\mu(1-\beta)t}]\\
			&\triangleq A(\beta)
 \end{split}
\end{equation*}
We have to solve $\beta = A(\beta)$:
\begin{itemize}
	\item $A(0) = \int_0^{+\infty}e^{-\mu(1-\beta)t}dG(t) > 0$
	\item $A(1) = 1 $
	\item $A^{'}(\beta)=\int_0^{+\infty}e^{-\mu(1-\beta)t}(\mu t)dG(t) > 0$, therefore $A(\beta)$ is a monotonic increasing function
	\item $A^{''}(\beta)=\int_0^{+\infty}e^{-\mu(1-\beta)t}(\mu t)^2dG(t) > 0$, therefore $A(\beta)$ is a concave function
\end{itemize}
%%GRAPH MISSING
If 2 is $A(\beta)$ there is no conclusion (no intersection). If, instead, 1 is $A(\beta)$ there is one intersection, that intersection is the solution and the chain is positive recurrent with $\pi_k = (1-\beta^*)(\beta^*)^k$.\\
The third possibility is $\beta^* = 1$ and the curve is tangent. Coming from below, $A^{'}(1)>1$, therefore $\frac{\mu}{\lambda}>1$. We know from queing theory that the condition $\lambda < \mu$ is a sufficient condition for the stability of the chain. In this example, we are not able to find a solution if $\lambda > \mu$, in principle there should be no conclusion, but we know there is no solution from queing theory.
